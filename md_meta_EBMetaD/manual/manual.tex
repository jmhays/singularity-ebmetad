\documentclass[12pt,fleqn]{report}
\usepackage{color}
\usepackage{multicol}
\usepackage{makeidx}
\usepackage{bm}
\makeindex

\usepackage[linkbordercolor={0 0 1}]{hyperref}
\definecolor{light-gray}{gray}{0.95}

\newcommand{\keyword}[1]{\index{Keywords!{\tt #1}} {\tt #1}}
\newcommand{\cv}[1]{\index{Collective variables!{#1}}
\index{{#1}|see{Collective variables, {#1}}} }
\newcommand{\plumed}{{\tt PLUMED}}


\newcommand{\esempio}[1]{
\vspace{10pt}
\begin{flushright}
\colorbox{light-gray}{
   \begin{minipage}{13cm}
       \scriptsize{
{\fontfamily{phv} \fontseries{b}
 \selectfont Example. \\
 \fontseries{m} \selectfont #1 } }
\end{minipage}}
\end{flushright}
\vspace{20pt}
}



\begin{document}

%% -----------------------------------------------------------------------------------------
\begin{titlepage}
\vspace{4cm}
\begin{flushleft}
{
 { \fontencoding{OT1}\fontfamily{phv} \fontshape{i} \fontseries{b} \Huge{PLUMED User's Guide}} \\ \vspace{.5cm}
 { \fontencoding{OT1}\fontfamily{phv} \fontshape{sl} \selectfont \Large{A portable plugin for free-energy calculations \\ with  molecular dynamics}}
}
\rule{12cm}{4pt}
\end{flushleft}
\vspace{10cm}


\begin{flushright}
 \fontencoding{OT1}\fontfamily{phv} \fontseries{b} \fontshape{i} \large{Version 1.3.0 -- Nov 2011}
\end{flushright}
%% -----------------------------------------------------------------------------------------



\end{titlepage}

\tableofcontents

\chapter{Introduction}
\label{ch.introduction}

\section{What is \plumed?} 

 \plumed\cite{plumed} \ is a plugin for free-energy calculations in molecular
systems.  It works with some of the most popular classical molecular dynamics (MD) codes,
such as GROMACS \cite{Hess:2008p11450}, NAMD \cite{NAMD}, DL\_POLY \cite{dlpoly} , LAMMPS \cite{lammps}
and the  SANDER module in AMBER \cite{Case:2006p16366}.
It also works with the very fast CUDA/GPU MD code ACEMD \cite{acemd} and,
more recently, it has been extended to work with \emph{ab initio} MD codes,
such as Quantum-ESPRESSO \cite{giannozzi09jpcm} and CPMD.

Free-energy calculations can be performed as a function of
many order parameters, with a particular focus on biological problems, and using
state-of-the-art methods such as metadynamics \cite{metad}, umbrella sampling \cite{torrie-valleau,wham1,wham2} and
Jarzynski-equation based steered MD \cite{jarzynski,Crooks98}.  

Here is a brief outline of this guide:
\begin{itemize}
\item In this chapter we give  an overview
of the features and restrictions of  the current release of \plumed.
\item In the second chapter we describe the
procedure for installing the plugin and  testing the correct installation.
\item The third chapter explains how \plumed \ can be used to perform free-energy calculations, 
setting up the input file and analyzing the output.
\item The fourth chapter contains  a list of collective variables (CVs) which are implemented
and allow a wide variety of physical and chemical problems to be addressed.
\item The fifth chapter is dedicated to postprocessing. It explains how to reconstruct the
free-energy profile from the output of a metadynamics run and how to extract the CV values from MD trajectories.
%\item (??) the fifth chapter describes the overall structure of the plugin code, and how new collective variables
 %can be implemented in it.
\end{itemize}

\section{Supported codes}
\label{supported}
 \plumed \ works as an add-on to some of the most popular MD codes:
NAMD, GROMACS, SANDER, DL\_POLY, Quantum-ESPRESSO, ACEMD, LAMMPS and CPMD.
These codes are not distributed with the \plumed \ package, but they must be obtained separately.
\vspace{0.3cm}

NAMD 2.6/2.7/2.8   \\    \url{http://www.ks.uiuc.edu/Research/namd/}

\vspace{0.3cm}

GROMACS 4.0/4.5.x    \\   \url{http://www.gromacs.org/}

\vspace{0.3cm}

AMBER 10/11            \\     \url{http://ambermd.org/}

\vspace{0.3cm}

DL\_POLY 2.20     \\  \url{http://www.cse.scitech.ac.uk/ccg/software/DL\_POLY/}

\vspace{0.3cm}

Quantum-ESPRESSO 4.3.2   \\  \url{http://www.quantum-espresso.org/}

\vspace{0.3cm}

ACEMD 1.2    \\  \url{http://multiscalelab.org/acemd}

\vspace{0.3cm}

LAMMPS 27-Oct-2011 \\ \url{http://lammps.sandia.gov}

\vspace{0.3cm}

CPMD 3.15.1 \\ \url{http://www.cpmd.org}

\vspace{0.3cm}
Note that at the moment of this release of \plumed, only those specific versions of the listed codes
have been tested.
Moreover, some other codes are available through the Python interface provided by Rosa Bulo.
\vspace{0.3cm}

Amsterdam Density Functional (ADF) \\ \url{http://www.scm.com/}

\vspace{0.3cm}

and all the codes supported by 

\vspace{0.3cm}

Atomic Simulation Environment (ASE) \\ \url{https://wiki.fysik.dtu.dk/ase/}

\vspace{0.3cm}

Additionally there exist some porting to  

\vspace{0.3cm}

FHI-AIMS \\ \url{https://aimsclub.fhi-berlin.mpg.de/}

\vspace{0.3cm}

which is not directly mantained by  the \plumed\ developer team. Additional information have to be 
retrieved by the developers of AIMS themselves.  





\section{Features}
\label{sec.overview}

\plumed \ can perform several different types of calculation:
\begin{itemize}
\item Metadynamics with a large variety of CVs \cite{metad};
\item Well-tempered metadynamics \cite{Barducci:2008};
\item Multiple walkers metadynamics \cite{multiplewalkers};
\item Combined parallel tempering and metadynamics \cite{bussi_xc};
\item Bias-exchange metadynamics \cite{piana};
\item Reconaissance metadynamics \cite{tribello-recon};
\item Steered MD;
\item Umbrella sampling;
\item Adiabatic biased molecular dynamics \cite{ballone};
\item Commitment analysis.
\end{itemize}

\section{New in version 1.3}
\plumed  \ version  1.3 presents several new features, including new collective variables and 
support to new codes. Among these:
\begin{itemize}
\item Reconaissance Metadynamics \cite{tribello-recon};
\item Driven adiabatic free energy dynamics (d-AFED, contributed by Michel Cuendet)
\item Definition of CVs as Polynomial combination of CVs and Function on CVs;
\item New INTERVAL keyword to limit a region to be sampled using Metadynamics (contributed by Alessandro Laio)
\item Better scalability with Gromacs;
\item New collective variables: Projection on PCA eigenvectors (contributed by Ludovico Sutto), SPRINT topological variable and gyration tensor based CVs;
\item New tool: reweight well-tempered metadynamics simulations \cite{Bonomi:2009p17105};
\item Added support to ACEMD 1.2, Lammps (27-oct-2011), Quantum Espresso 4.3.2, CPMD 3.15.1, NAMD 2.8, Gromacs 4.5.x and Amber11;
\item Python interface (contributed by Rosa Bulo);
\item New tool to perform bias-exchange simulations via the linux shell with any MD code;
\item Bugs have been fixed and the manual has been improved thanks to the feedbacks of many users.
\end{itemize}

A full list including also the bug fixed in the current release can be found in the CHANGES file 
distributed with the package.

\section{Restrictions}

The current release of \plumed \ has a few restrictions: 
\begin{itemize}
\item Parallel tempering plus metadynamics and bias-exchange metadynamics are
available only in the GROMACS version (however a tool allows bias-exchange simulations via linux shell with any MD engine); 
\item The patched version of GROMACS cannot perform hamiltonian (lambda) replica
      exchange;
\item The \keyword{ENERGY} collective variable is available only for GROMACS, AMBER and DL\_POLY and cannot be used with multiple time step algorithm;
\item Only orthorhombic cells are supported in ACEMD and DRIVER;
\item Only orthorhombic and truncated octahedron cells are supported in AMBER;
\item Only Car-Parrinello and Born-Oppenheimer simulations are supported in CPMD;
\item Only NVT simulations are supported because the contribution to the virial is not calculated.

\end{itemize}


\section{The \plumed \ package}
The plugin package has the following directory structure:
\begin{itemize}
\item {\verb common_files }.  The directory  containing all the basic routines.
\item { \verb tests }. A variety of examples of different CVs and free-energy methods provided
with topology and input files for the supported codes.
These examples, combined with a script adapted from CP2K \cite{VandeVondele:2005p10650}, work also as a regtest for the plugin.
\item {\verb patches }. A collection of patches to interface \plumed \ with different codes.
\item {\verb utilities }. A few small utilities:
\begin{itemize}
\item {\verb sum_hills } is a post-processing program which reads the {\verb HILLS } file produced by the plugin
in a metadynamics simulation and returns the free energy by summing the Gaussians that have been
deposited.
\item {\verb driver }
is a tool that calculates  the value of selected CVs along a MD trajectory.
It requires a PDB file, a trajectory in DCD format and a file with the same syntax of the \plumed \ input file.
\item {\verb standalone } is a program to run \plumed \ as a standalone code.
\item {\verb plumedat.sh } is a shell script to extract information from \plumed\ output files.
\item{\verb reweight } is a tool to calculate the unbiased probability distribution of variables
             other than the CVs in well-tempered metadynamics simulations
\item{\verb python_interface  } a set of tools that enables to use \plumed\ with Python (hey, a Python which is plumed with feathers is a really weird animal!).
\item{\verb bias-exchange } a set of tools that allow to use \plumed\ to perform bias-exchange simulations via the linux shell using any MD engine.
\end{itemize}
\item {\verb manual }. This manual.
\item {\verb ACEMD }. It contains the files needed to compile the plugin for ACEMD.
\item {\verb tutorial }. It contains the resources of the \plumed \ tutorial 2010 (\url{http://sites.google.com/site/plumedtutorial2010/})
\item {\verb recon_src }. It contains the source files specific for the Reconaissance Metadynamics.
\end{itemize}

\section{Online resources}
\label{sec.online}
You can find more information on the web: 

\begin{center}
\url{http://www.plumed-code.org}
\end{center}
For any questions, please subscribe to our mailing list:

\begin{center}
\href{mailto:plumed-users@googlegroups.com}{plumed-users@googlegroups.com}
\end{center}

\section{Credits}
\plumed \ has been developed  by 
Massimiliano Bonomi, Davide Branduardi, Giovanni Bussi, Carlo Camilloni, 
Davide Provasi, Paolo Raiteri,
Davide Donadio, Fabrizio Marinelli, Fabio Pietrucci, 
 Francesco Luigi Gervasio and others.
However, this work would not have been possible without the joint effort of many people. 
Among these, we would like to thank (in alphabetical order):
Alessandro Barducci, Anna Berteotti, Rosa Bulo, Matteo Ceccarelli, Michele Ceriotti, Paolo Elvati,  Antonio Fortea-Rodriguez, 
Alessandro Laio, Matteo Masetti, Fawzi Mohamed, Ferenc Molnar, Gabriele Petraglio, Jim Pfaendtner and Federica Trudu. 
Francesco Marini is kindly acknowledged for his technical support and Joost
VandeVondele for  permission to use his regtest script.

Some \plumed \ users have also contributed to the implementation
of new features and the debugging of old bugs. Among these, we would like to thank:
Toni Giorgino, Marcello Sega, Emmanuel Autieri, Gareth Tribello, Andrea Coletta, Ludovico Sutto, Layla Martin-Samos,
Walter Rocchia, Alessio Lodola, Luigi Capoferri, Michel Cuendet, Jiri Vymetal, Fahimeh Baftizadeh, and Katsumasa Kamiya.

\section{Citing \plumed}
\label{sec.citing}
You may wish to cite the following reference if you have utilized \plumed \ in your work:

M.~Bonomi, D.~Branduardi, G.~Bussi, C.~Camilloni, D.~Provasi, P.~Raiteri, 
D.~Donadio, F.~Marinelli, F.~Pietrucci, R.A.~Broglia and M.~Parrinello.

\href{http://dx.doi.org/10.1016/j.cpc.2009.05.011}{PLUMED: a portable plugin for free-energy calculations with molecular dynamics},
 Comp.~Phys.~Comm. 2009 vol. 180 (10) pp. 1961-1972. 

%\href{http://arxiv.org/abs/0902.0874}{arXiv:0902.0874}.

\section{License}

\plumed \ is free software: you can redistribute it and/or modify 
it under the terms of the GNU Lesser General Public License as published by 
the Free Software Foundation, either version 3 of the License, or 
(at your option) any later version. 
\plumed \ is distributed in the hope that it will be useful,
but WITHOUT ANY WARRANTY; without even the implied warranty of
MERCHANTABILITY or FITNESS FOR A PARTICULAR PURPOSE.  See the 
GNU Lesser General Public License for more detail. 
You should have received a copy of the GNU Lesser General Public License
along with {\tt PLUMED}.  If not, see  \url{http://www.gnu.org/licenses/}.

%% ================================================================================================

\chapter{Installation}
\label{sec.install}
The plugin  installation requires the molecular
dynamics code to be recompiled after the calls to the plugin routines 
have been inserted at the appropriate places in the original program.  
For a list of the supported molecular dynamics codes see Sec.~\ref{supported}.

All the basic plugin routines are contained in the {\tt common\_files} folder of
the \plumed \ distribution package.
The insertions are automatically performed by a series of
scripts provided in the {\tt patches} directory.
\\
In the following we will briefly describe the procedure for applying these patches
to the different MD codes supported by \plumed.
We will refer to the root directory of the distribution package as {\tt PLUMED\_root}.

\section{Compiling \plumed}

A similar procedure can be followed for all the supported codes except ACEMD.
When code-specific procedures are needed, we state it explicitely.
A different proceedure is required for ACEMD (see below).

Different patches are available for different code versions. The name
of the suitable patch is {\tt plumedpatch\_CODENAME\_CODEVERSION.sh}.
If the patch corresponding to the exact CODEVERSION that you are using is not
available, choose the closest match.
In the following, we shall indicate with {\tt plumedpatch.sh} the proper patch script.
\begin{itemize}
\item Extract the source code from its archive and then move into its root directory.
\item Configure the code as usual (not necessary for DL\_POLY)
\item NAMD and LAMMPS only: In the plumedpatch script, modify the myarch variable to match your architecture.
\item Set the environmental variable {\tt plumedir} to point to {\tt PLUMED\_root}.
\item Copy or link the {\tt plumedpatch.sh} file from the {\tt patches} folder to the current directory.
\item Execute the script: "{\tt ./plumedpatch.sh -patch} ".
\item DL\_POLY-only: copy the proper Makefile from {\tt build/} to {\tt srcmod/} (or {\tt src/} in older versions)
\item CPMD only: you may need to change variables {\tt cxx, cxxflag, extraflag} in the plumedpatch script
depending on your computer architecture: see the examples below.
\item Compile the code as usual.
\end{itemize}

Further details can be found in the {\tt patches/README} file.

\esempio{This is the procedure for compiling the serial version of AMBER 10 with \plumed \ using g95
in the Bourne shell. \vspace{10pt} \\
{\tt tar zxf AMBER10.tgz \\
cd amber10/ \\
export plumedir="PLUMED\_root" \\
cp \$plumedir/patches/plumedpatch\_sander\_10.sh . \\
cd src/ \\
./configure g95 \\
cd .. \\
./plumedpatch\_sander\_10.sh -patch \\
}

{\tt make}

}

\esempio{This is the procedure for compiling the serial version of GROMACS, using the GNU compilers. \vspace{10pt} \\
{\tt tar zxf gromacs-4.0.5.tar.gz \\
cd gromacs-4.0.5 \\
export plumedir="PLUMED\_root" \\
cp \$plumedir/patches/plumedpatch\_gromacs\_4.0.4.sh . \\
CC=gcc CXX=g++ ./configure \\
./plumedpatch\_gromacs\_4.0.4.sh -patch \\
make \\
make install
}
}

\esempio{This is the procedure for compiling NAMD on an Intel Mac using the GNU g++ compiler and the FFTW. \vspace{10pt} \\
{\tt tar zxf NAMD\_2.6\_Source.tar.gz \\
cd NAMD\_2.6\_Source \\
export plumedir="PLUMED\_root" \\
cp \$plumedir/patches/plumedpatch\_namd\_2.6.sh . \\
./config fftw MacOSX-i686-g++ \\
}

Edit {\tt ./plumedpatch\_namd\_2.6.sh} by setting the  {\tt myarch} variable to {\tt MacOSX-i686-g++}.\\ 

{\tt
./plumedpatch\_namd\_2.6.sh -patch\\
cd MacOSX-i686-g++ \\
make 
}
}

\esempio{
This is a sample procedure for compiling the scalar version of DL\_POLY\_2.20 with the gfortran compiler in the Bourne Shell environment.\\ \\

{\tt tar zxf dl\_poly\_2.20.tar.gz \\
cd dl\_poly\_2.20\\
export plumedir=PLUMED\_root\\
cp \$plumedir/patches/plumedpatch\_dlpoly\_2.20.sh . \\
./plumedpatch\_dlpoly\_2.20.sh -patch\\
cp build/MakeSEQ srcmod/Makefile\\
cd srcmod\\
make gfortran
}}

\esempio{
This is a sample procedure for compiling the openmpi version of LAMMPS with the mpic++ 
compiler in the Bourne Shell environment. Please, note that the LAMMPS tarball 
comes with the download date but unless major changes are done in the host code, the 
patching procedure will stay unchanged and the latest available patching file has to be 
used. \\ \\
{\tt
tar xvf lammps-15Jan10.tar \\
cd lammps-15Jan10/ \\
}
\\ Edit {\tt ./src/MAKE/Makefile.openmpi} to suit your machine. \\ \\
{\tt
export plumedir="PLUMED\_root" \\
cp \${plumedir}/patches/plumedpatch\_lammps\_15-01-2010.sh ./ \\
}
\\ Edit {\tt ./plumedpatch\_lammps\_15-01-2010.sh} to set myarch to {\tt openmpi}. \\ \\
{\tt
./plumedpatch\_lammps\_15-01-2010.sh -patch \\
cd src \\
make openmpi \\
}
}

\esempio{
This is a sample procedure for compiling CPMD-3.15.1 under different computer architectures.\\ \\
{\tt
tar zxvf cpmd-v3\_15\_1.tgz \\
cd CPMD
} \\ \\ 
Generate a valid Makefile for your machine using the script {\tt ./mk\_config.sh}, e.g.: \\ 
{\tt 
./mk\_config.sh PC-GFORTRAN > Makefile \\ \\
export plumedir="PLUMED\_root" \\
cp \${plumedir}/patches/plumedpatch\_cpmd-3.15.1.sh .
} \\ \\
If necessary, modify the variables {\tt cxx} (a valid C++ compiler), {\tt cxxflag} (the corresponding flag to avoid exception handling), 
and {\tt extraflag} (see below) in the script {\tt plumedpatch\_cpmd-3.15.1.sh}, e.g.
for GNU compilers: \\
{\tt cxx="mpiCC" \\ cxxflag="-fno-exceptions" } \\ \\
for IBM machines like AIX-power7: \\
{\tt cxx="xlc\_r -c" \\ cxxflag="-qnoeh" } \\ \\
for IBM BlueGene: \\
{\tt cxx="bgxlc++\_r -c -O" \\ cxxflag="-qnoeh" } \\ \\
for PGI compilers: \\
{\tt cxx="pgCC" \\ cxxflag="--no\_exceptions" } \\ \\
To successfully compile on some IBM architectures you may also need to set one or both of the following extra flags: \\
{\tt extraflag="-DPLUMED\_CPMD\_NOUNDERSCORE -DPLUMED\_AIXXLF"} \\ \\
Finally, patch and compile:\\ 
{\tt ./plumedpatch\_cpmd-3.15.1.sh -patch \\
make }
}

\subsection{Compiling the ACEMD plugin with \plumed}
\plumed \ works with ACEMD 1.2.
As the source code for ACEMD is not available, \plumed \ 
will work as a plug-in. For this reason the compilation
will be slightly different from that of the other codes.
Moreover, given the unavailability of source code,
this version of \plumed \ will {\bf not be supported} by the \plumed \ 
development team. 

\begin{itemize}
\item Extract the source code from its archive.
\item Set the environmental variable {\tt plumedir} to point to {\tt PLUMED\_root}. 
\item Use {\tt make} to compile the plugin inside the {\tt plumedir/ACEMD} folder. 

\esempio{
This is a sample procedure for compiling the ACEMD plug-in in the Bourne Shell environment.\\ \\

{
export plumedir=PLUMED\_root\\
cd \$plumedir ACEMD \\
make 
}}
\end{itemize}
Once the plumed.so is compiled, copy it where you will run ACEMD.
Add the following lines to the ACEMD input file:
\begin{verbatim}
pluginload testplug ./plumed.so
pluginarg testplug input META_INP
pluginarg testplug boxx xx 
pluginarg testplug boxy yy   
pluginarg testplug boxz zz 
pluginfreq 1
\end{verbatim}

where META\_INP is the \plumed \ input file and xx,yy,zz are the box dimensions.
At this time the support of the ACEMD plugin will be provided by the ACEMD
developers. Please note that only orthorhombic cells are available in ACEMD 1.2.

% {\tt setenv mydir "/Users/chicco/Programs/md\_meta/"}

%
%substituting  the path where common files for metadynamics are placed, {\tt \$PLUMEDDIR}.
%For the 2.16 version only, {\tt mydir} can be modified directly in the {\tt metapatch\_dlpoly\_2.16.sh} script.
%\item execute the script, {\tt .\/metapatch\_2.16.sh -patch} (for the 2.16 release) or 
% {\tt .\/metapatch\_2.19.sh -patch} (for the 2.19  release)
%\item copy the desired makefile from the {\tt build/} directory into the {srcf90/} directory; 
%notice that with some compilers (e.g. g95) the option \\

%{\tt -fno-second-underscore} \\

%must be added in 
% the compilation flag to allow the correct linking of C and Fortran routines
%\item build as usual
%\end{itemize}

\section{Including reconnaissance metadynamics}
\label{sec:recon_compile}

The reconnaissance metadynamics routines (see section \ref{sec:recon_metad} included in \plumed\ make use of the C$++$ standard library and lapack.  On some machines these libraries are not installed.  Hence, \plumed\ has been written so that by default no attempt to compile reconnaissance metadynamics is made.  Consequentially, if you wish to perform reconnaissance metadynamics you must patch \plumed\ slightly differently.

To include reconnaissance metadynamics in your patched MD code is simply a matter of, during the patching procedure, telling \plumed\ the location of the C standard library, the location of lapack and the location of a c++ compiler.  This is done by making some small modifications to the plumedpatch file.  If this file is opened using a text editor you will notice at least one of the following two lines: \\ \\
%
{\verb RECON_LIBS= } \\
{\verb RECON_CPP= }
\\

If at least one of these lines is not present then reconnaissance metadynamics is not implemented for the particular code you are patching.  The first of these two lines tells \plumed\ the locations of the lapack and c standard libraries.  If the space after the equal sign is left blank \plumed\ patches in its default fashion and no attempt is made to compile the reconnaissance metadynamics routines.  By contrast if there is any character after the equals sign \plumed\ will endeavor to compile the reconnaissance metadynamics routines.  For a correct compilation the {\verb RECON_LIBS} variable should indicate the locations of the lapack (i.e. -llapack) and c standard libraries (i.e. -lstdc$++$).  The  {\verb RECON_CPP= } variable should be set equal to a c$++$ compiler (e.g. g$++$).   

\esempio{To include the reconnaissance metadynamics routines in the \plumed\ compilation the use must change the values of {\tt RECON\_LIBS} and {\tt RECON\_CPP} in the plumedpatch file: \vspace{10pt} \\
RECON\_LIBS="-llapack -lstdc++" \\
RECON\_CPP="g$++$" \\
} 

There are some differences between the procedures for patching \plumed +  reconnaissance metadynamics in the different MD codes between the different codes.  This is because some of the codes already have lapack included or are compiled with a c++ compiler.  These differences are described in table \ref{tab:recon_table}:

\begin{table}
\centering
\begin{tabular}{ r c c c c } 
Code & Reconnaissance & C++ compiler & Lapack & lstc++    \\
& implemented &  required & required & required \\[0.1cm] \hline
NAMD  & yes & no & yes & yes \\
GROMACS & yes & yes & no & yes \\
AMBER   & no &  - & - & -       \\
DL\_POLY & yes & yes & yes & yes \\
Q-ESPRESSO & yes & yes & no & yes \\
ACEMD & no & - & - & - \\
LAMMPS & no & - & - & - \\ 
CPMD & yes & yes & no & yes \\ \hline
\end{tabular}
\caption{ Details on compiling \plumed\ compilation with the various codes}
\label{tab:recon_table}
\end{table} 

\section{Testing the installation}
\label{sec.regtest}

Once the installation process has been completed successfully, 
the user is encouraged to test  the chosen MD package for any problems.
The {\tt tests} directory contains regtest scripts for the different MD codes.
These also serve as a regularity test in case the user implements his own modifications and 
as a basic illustration of the capabilities  of the plugin.
The user should edit the test script,
setting up the path where the test suite is found and giving the location of the executable.

For GROMACS users only. Please note that:
\begin{itemize}
\item The tests for GROMACS are designed for and should be executed with the
double-precision version of the code;
\item Biasxmd, ptmetad, dd and pd are designed for the parallel version of GROMACS.
The user should specify in the test script the location of the parallel executable and the version
of GROMACS used. These tests will fail if the parallel version of
    GROMACS has not been compiled. 
\end{itemize}

\esempio{In the case of NAMD, the regtest script is called 
{\tt  do\_regtest\_namd.sh}. Here, the user should modify the {\tt dir\_base} and the {\tt \tt namd\_prefix}: \vspace{10pt} \\
dir\_base=/Programs/md\_meta/tests/namd \\
 namd\_prefix=/chicco/bin/namd\_plugin/namd2. \\

Regtest scripts for the other programs require an identical procedure.

}

The script executes a list of tests and then compares the results
with the outcome of previously run simulations.
Please note that when the script is run for the 
first time it produces the reference. Finally, the script produces a summary of the results.
The test result can be one of the following:
\begin{itemize}
\item 'OK' if the results match those of a previous run precisely. The execution time is also given.
\item'NEW' if they have not been executed previously. The reference result is generated
      automatically in this run. Tests can also be 'NEW' if they have been reset, \emph{i.e.} been newly
      added to the TEST\_FILES\_RESET files.
\item 'RUNTIME FAILURE' if they stopped unexpectedly (e.g. core dump, or stop).
\item 'WRONG RESULT' if they produce a result that deviates from an old reference.
\end{itemize}
  
  The last two options generally mean that a bug has been introduced, which requires investigation.
  Since regtesting only yields information relative to a previously known result, it is most useful
  to do a regtest before and after you make changes.

\section{Back to the original code}

At any time the user may want to ''unpatch'' the MD code and revert back to
the original distribution. 
To do so, the user should go to the directory where the \plumed \ patch has been copied and type {\tt ./plumedpatch -revert}.  

\section{The Python interface to \plumed}
Starting from version 1.3 \plumed\ has also a Python interface contributed by Rosa Bulo. 
It is possible to find it in the  {\verb utilities/python_interface } directory.   
Inside it you might find a {\tt README} file and a {\tt Makefile} .
After copying (or linking) all the {\verb *.c } and  {\verb *.h } files from the {\verb common_files } directory 
you can immediately do:\\ 
{ \tt make \tt gnu  }
\\
if you want to compile with gnu compilers. Intel compilers are also available.
This creates a file that is {\tt libplumed.so } that will be loaded runtime by Python.
You need to add the directory where this file resides into the {\tt LD\_LIBRARY\_PATH} so that 
Python will be able to find it and  load it runtime. The extension {\tt amber} is only reflecting the fact that 
the interface is built by exploiting the simple AMBER interface. Therefore the limitiations regarding AMBER in the functionalities
apply also here.  

For MacOSX users the  compiler flags should be changed accordingly. For gnu compilers it requires to change 
the linking flags from \\
{\tt -shared -Wl,-soname,libplumed.so}\\
 to\\
 {\tt  -bundle -flat\_namespace -undefined suppress}.

Next, in the directory\\
 {\verb utilities/python_interface/pylib }\\
 you can find the file {\tt plumedamberlib.py  } which 
is the actual Python module that will deal of loading {\tt libplumed.so } and create the suitable Python interface.  
This provides the {\tt  init\_metadyn }  and {\tt cv\_calculation  } methods that one can practically use through Python.  

A practical use of this shown by the Python script\\
 {\tt /utilities/python\_interface/test/run\_water\_amber.py }\\
 where a
water molecule is simply displaced and the collective coordinates are updated at each step.
Notably, this interface is embodied in the Python tools provided with the Amsterdam Density Functional code 
(also called ADF, see \url{http://www.scm.com/ }).  

Additionally Rosa provided an interface with the powerful {\tt Atomic Simulation Environment} ( see \url{https://wiki.fysik.dtu.dk/ase/} ). 
This enable \plumed\ to be interfaced with plenty of DFT and ab-initio codes.
Some examples are present in the directory {\tt utilities/python\_interface/test } and are {\tt ASEmd\_plumed\_emt\_water.py } and
{\tt ASEnvt\_plumed\_emt\_water.py }. 

Many thanks to Rosa for the contribution! 


%% ================================================================================================
\chapter{Running free-energy simulations}
\label{ch.running}
\label{ch.input}

In this chapter we describe how to activate \plumed \
and how to create the correct \plumed  \ input file for a specific type of free energy calculation.
The typical output of these calculations is also explained in detail.

\section{How to activate \plumed}

\plumed \ input is contained in one single file, named {\tt plumed.dat} by default, which defines the
CVs, the type of run to be performed and the parameters for the bias potential generation. 
\begin{itemize}
\item The users of NAMD, SANDER and DL\_POLY can instruct the code to parse the \plumed  \ input file by setting the
\plumed \ variable to {\tt on} (or 1 for SANDER) in the MD input file. It is also possible to change the default name
for the \plumed \ input file by setting the {\tt plumedfile} variable in the MD input;
\item GROMACS users should activate \plumed \ on the command line by specifying the flag {\tt -plumed}
followed by the input file name. The extension of such a file must be {\tt .dat};
\item Quantum-ESPRESSO users should activate \plumed \ on the command line by specifying the flag {\tt -plumed}.
The name of the \plumed \ input file is hardcoded as {\tt plumed.dat}; 
\item In LAMMPS, \plumed \ is activated using the fix {\tt plumed}. The input file is specified by  {\tt plumedfile},
the output file by {\tt outfile}; 
\item In CPMD, \plumed \ is activated on the command line by specifying the flag {\tt -plumed}
after the CPMD input file and the pseudopotential directory. The \plumed \  input file name is {\tt plumed.dat}.
\end{itemize}

\esempio{
A typical SANDER input file for a metadynamics calculation: \\ \\
{\tt 
METADYNAMICS TEST \\
\&cntrl\\
 imin=0, irest=0, ntx=1, ig=71278,\\ 
 nstlim=1001, dt=0.0002,\\
 ntc=1, ntf=1,\\
 ntt=3, gamma\_ln=5,\\ 
 tempi=300.0, temp0=300.0,\\
 ntpr=200, ntwx=0,\\
 ntb=0, igb=0,\\
 cut=999., \\
 plumed=1 , plumedfile='plumed.dat'\\
/
}
}
\esempio{
The NAMD and DL\_POLY input file  should contain,
in addition to the usual keywords defining the run, the following
lines; \\ \\ 
{\tt 
 plumed            on \\
 plumedfile       plumed.dat 
}}
\esempio{
To perform a free-energy calculation with GROMACS, \plumed \ must be activated on the command line: \\ \\
{\tt mdrun -plumed plumed.dat \dots}
}
\esempio{
To perform a free-energy calculation with Quantum-ESPRESSO, \plumed \ must be activated on the command line: \\ \\
{\tt pw.x -plumed \dots}
}

\esempio{
The LAMMPS input file should contain, in addition to the usual keywords defining the run, 
the following line; \\ \\
{\tt
 fix             3 all plumed plumedfile plumed.dat outfile metaout.dat 
}}


\esempio{
To perform a free-energy calculation with LAMMPS the file  should contain,
in addition to the usual keywords defining the run, the following command: \\ \\
{\tt fix ID all plumed plumedfile plumed.inp outfile plumed.out} \\ \\
where ID is the user assigned fix name.
}

\esempio{
To perform a free-energy calculation with CPMD, \plumed \ must be activated on the command line: \\ \\
{\tt cpmd.x input . -plumed} \\ \\
where {\tt input} is the CPMD input file, "." is the current directory (if pseudopotentials are located elsewhere, substitute with the right location),
and the \plumed \ input file is called {\tt plumed.dat}.
}

\section{The input file}
In the following sections we describe the syntax used in the \plumed \ input file.
The commands contained in this file can be divided in two groups according to the
functionality required.
A first group of commands defines the type of simulation (metadynamics, steering
and umbrella sampling, replica exchange methods) and the parameters needed for
the chosen algorithm. These are described in this chapter.  A second part  defines 
the degrees of freedom on which the algorithms operate, the so-called 
collective variables or CVs. The details of such commands are described in the
next chapter.

As a general rule, each setting is defined by a principal keyword that must be
placed at the beginning of the line, in upper case. Additional input pertaining
to the setting can be specified on the same line, using additional keywords
that can be added in any order. A line can be continued to the next one by adding
a backslash or an ampersand as a last character in the line.
Three kinds of keyword may exist: the \emph{directives}, 
which must be placed at the beginning of a line and define the argument of the line, the
\emph{keywords}, which specify the attributes of the different fields in the line
and \emph{flags}, which simply turn a given option on or off.

\esempio{The following is an example of a complete \plumed  \ input file. In this case
the input defines a well-tempered metadynamics run with two CVs; the first 
is the distance between two atoms, and the second a dihedral angle. \vspace{10pt} \\
{\tt \# general options  \\
HILLS HEIGHT 0.1 W\_STRIDE 100 \\
WELLTEMPERED SIMTEMP 300 BIASFACTOR 10 \\

\# print each 50 time units and add a time offset of 20.0 to COLVAR \\
PRINT W\_STRIDE 50  T\_OFFSET 20.0 \\
\\
\# definition of CVs \\

NOTE distance between hydrogens \\
DISTANCE LIST 13 46 SIGMA 0.35 \\

NOTE torsional angle \\
TORSION LIST 1 4 65 344 SIGMA 0.1 \\
\\
\# wall on the CV\\
UWALL CV 1 LIMIT  15.0 KAPPA 100.0 EXP 4.0 EPS 1.0 OFF 0.0 \\
\\
ENDMETA
}}

The \keyword{ENDMETA} directive is the last line read from  the \plumed \ input file and
any line added after this keyword will be ignored.
\\
The symbol \keyword{\#} allows the user to comment any line in the input file. The directive
\keyword{NOTE} allows the user to place comments which are copied to the \plumed \ log file.\\
The \keyword{PRINT} allows to dump a file called {\tt COLVAR} (see more in \ref{sec.control})
which is written each {\tt W\_STRIDE} timesteps and (optional) a time offset can be specified through 
the {\tt T\_OFFSET} keyword. To append the colvar values to an existing  {\tt COLVAR} file,
the flag {\tt APPEND} can be used. 

\section{A note on units}
\label{sec.units}
\index{Units}

The values in  the \plumed \  input file are read in the internal units for the
MD engine.  For {\tt DL\_POLY} the energy in input can be in different
user-specified units; to be consistent, the values in the \plumed \ input file 
must be in the same units  specified in the {\tt FIELD} file.

\section{Metadynamics}
\label{sec.control}

The metadynamics algorithm applies additional forces to a standard molecular
dynamics simulation \cite{metad,gerv-laio09review,bard-bon-parr11review}. In this case,  the \plumed \ input file must
contain the definition of at least one CV (see chapter
\ref{ch.generalcv} for the required syntax), and the \keyword{HILLS} keyword which
defines the details of the bias potential (see section \ref{ssc.bias}).
In this case the biasing potential will be calculated and applied to the
microscopic degrees of freedom during the run.  
Before discussing the additional optional commands (section \ref{ssc.bias}), we give
a brief overview of the file produced in a typical metadynamics run.

\subsection{Typical output}

\index{Output files, Metadynamics} 
Standard metadynamics will produce, in addition to the
usual output files generated by the MD engine, a file called {\tt COLVAR} that
contains the following data:
\begin{itemize}
\item The first column contains the time step;
\item The following $d$ columns contain 
the values of the CV(s);
\item Two additional columns contain the value $V(s,t)$ of the bias
potential at the given point and time, and the potential due to the 
walls (if defined).
\end{itemize}
Since \plumed\ 1.2, a more flexible format for {\tt COLVAR} has been introduced. Here, the first line
of the {\tt COLVAR} file is a header. The line begins with \#, so as to be ignored by plotting
programs such as {\tt gnuplot} and {\tt xmgrace}. A small script
\index{plumedat.sh}
{\tt plumedat.sh} to interpret this file is provided
in the utilities.

Another file, called {\tt HILLS}, contains the information of the biasing
potential needed to estimate the free-energy, and to restart metadynamics runs. 
The bias potential is given by:
\begin{equation}\label{bias}
V(\vec{s},t) = \sum_{ k \tau < t} W(k \tau)
\exp\left(
-\sum_{i=1}^{d} \frac{(s_i-s_i^{(0)}(k \tau))^2}{2\sigma_i^2}
\right).
\end{equation}
Specifically,
\begin{itemize}
\item The first column contains the time step at which the contribution to the bias potential
was added, $\tau$, $2\tau$, etc.;
\item The following $d$ columns contain the values $\vec{s}^{(0)}(t)$ 
specifying the position of the centroid of the Gaussian;
\item The next $d$ columns contain the  values $\vec{\sigma}$  specifying the 
Gaussian width  along the different directions in the CV space;
\item The last  column but one contains the value $W$ of the Gaussian height;
\item The last column contains the bias factor defined in well-tempered metadynamics.
\end{itemize}

Please refer to section \ref{sc.sumhills} for a description of how the potential
in equation \ref{bias} can be computed from the {\tt HILLS} file.

It is important to note that a  metadynamics run should typically start from an
equilibrated system.  The equilibration protocol can be applied without
resorting to  the \plumed \ input file. However, in order to gain important
information concerning the behavior of the run and the tuning of the parameters
of the metadynamics biasing potential, it is useful to monitor the behavior of
the CVs during the equilibration run.
To this aim, just skip the \keyword{HILLS} directive:
in this way, only the {\tt COLVAR} file will be generated.

\subsection{Bias potential}
\label{ssc.bias}
The \keyword{HILLS} directive is used to define the details of the biasing
potential. It must be followed by the definition of the weight
factor $W$, preceded by the keyword {\tt HEIGHT} (see the note on
units in section \ref{sec.units}). Also the frequency at which the Gaussians
are deposited and written into the {\tt HILLS} file must be set using
{\tt  W\_STRIDE} followed by the number of MD steps between two successive depositions.  
The Gaussian width must be set in the proper CV line using the
keyword  {\tt SIGMA}.
\\
Notice that the {\tt HILLS} directive is not compatible with the {\tt COMMITMENT} directive.
\esempio{The following line switches on metadynamics with Gaussian height of 0.1 (in energy unit of the MD code)
and deposition stride of 1000 MD steps. \vspace{10pt} \\ 
{ \tt HILLS HEIGHT 0.1 W\_STRIDE 1000 } 
}

\subsection{Well-tempered metadynamics}
The {\tt WELLTEMPERED} directive activates well-tempered metadynamics \cite{Barducci:2008} which, 
by rescaling the Gaussian weight factor, guarantees the theoretical convergence of metadynamics.
In the well-tempered algorithm, the rate at which the bias potential is added is decreased
during the simulation proportionally to $e^{-V({s},t)/\Delta T}$,
where $\Delta T$ is a characteristic energy:
\begin{equation}
V({s},t)= \sum_{t'=0,\tau_G,2\tau_G,\dots}^{t'<t} W e^{-V({s}({q}(t'),t')/\Delta T} \exp\left(
-\sum_{i=1}^{d} \frac{(s_i({q})-s_i({q}(t'))^2}{2\sigma_i^2}
\right),
\label{meta_eq}
\end{equation}
where $W=\tau_G \omega$ is the height of a single Gaussian.
\\
For a given temperature of the system $T$ (specified with the  keyword {\tt SIMTEMP}), the
CVs are sampled at a fictitious higher temperature $T+\Delta T$ determined by the bias factor $(T+\Delta T)/T$. The user must specify this bias factor using the keyword {\tt BIASFACTOR}.

\esempio{
The following commands define a well-tempered metadynamics run, in which the CVs are sampled 
at the higher temperature of 3000 K. The initial Gaussian height is 0.1 (in energy unit of the MD code) and the
deposition stride is 1000 MD steps.
\vspace{10pt} \\ 
 {\tt HILLS  HEIGHT 0.1 W\_STRIDE 1000  \\
 WELLTEMPERED SIMTEMP 300 BIASFACTOR 10} }

It should be noted that, in the case of well-tempered metadynamics,
in the output printed on the {\tt HILLS} file
the Gaussian height is rescaled using the bias factor.
This is done in order to directly obtain the free energy (and not the bias), 
when summing all the Gaussians deposited during the run.


%
\subsection{Restarting a metadynamics run}

In order to restart a metadynamics run, the flag {\tt RESTART} must be added on the line of the directive {\tt HILLS}. 
It allows a metadynamics simulation to be restarted after
an interruption or after a run has finished. The {\tt HILLS} files will be read
at the beginning of the simulation and the bias potential applied to the
dynamics. Note that the presence of the {\tt RESTART} flag only affects
the metadynamics part of the simulation, and thus the usual procedure for 
restarting a MD run must be followed. This depends on the
particular MD engine used and can be found in the relative documentation.
\esempio{The following is an example of input file for restarting a metadynamics simulation.  \vspace{10pt} \\ 
{ \tt HILLS RESTART HEIGHT 0.1 W\_STRIDE 1000 }}

In case of well-tempered metadynamics, the Gaussians height is rescaled in input according to the
bias factor. This is done assuming that the sum of the Gaussians stored in the {\tt HILLS} file
is an estimate of the (negative) free energy landscape. Since the estimate of the free energy
is in principle independent from the choice of the bias factor, it is correct to restart a
well-tempered simulation with a different bias factor or even restart from a non-well-tempered simulation.

%When the bias factor is changed during the simulation, the Gaussians
%read at restart are rescaled according to the new bias factor.
%In order to use the old ones, the keyword \keyword{READ\_OLD\_BF} must be specified.

%\esempio{The following is an example of input file for restarting 
%a well-tempered metadynamics simulation after changing the bias factor
%using the old one when reading the Gaussians deposited. \vspace{10pt} \\
%{ \tt HILLS RESTART HEIGHT 0.1 W\_STRIDE 1000 \\
%WELLTEMPERED SIMTEMP 300.0 BIASFACTOR 10.0 READ\_OLD\_BF }
%}


\subsection{Using {\tt GRID}} \index{Grid for metadynamics}
\label{grid}
Normally the additional forces of metadynamics are calculated every MD step by summing
the contribution coming from the Gaussians deposited up to this point, following Eq.~\ref{meta_eq}. 
As the simulation goes on, the computational time spent in the evaluation of these 
forces becomes larger and larger and eventually comparable with the time needed to 
calculate the contribution of the force field itself.
This effect is particularly  visible when the system  simulated is small or when using a simplified
coarse grained potential.

A possible solution is to store an array containing the current value of the bias potential (and of the
derivatives with respect to the CVs) on a grid.
In this way the computational cost of metadynamics becomes constant all over the simulation and corresponds
to the cost of evaluating a single Gaussian function on the whole grid with a frequency given by the
stride between subsequent hills.
This approach is similar to that proposed in Ref.~\cite{babi+08jcp}, but has the advantage that
the grid spacing is independent on the Gaussian width.

This operation can be demanding if the number of collective variable and/or the number of grid bins is high.
However, the cost of adding the Gaussian on the grid can be substantially reduced 
taking into account that this function is almost zero outside a characteristic range determined by the Gaussian sigma.
In \plumed \ this interval is calculated once (or at every modification of sigma) and used to build a smaller sub-grid on which 
the potential is updated.

In order to use the grid, the directive \keyword{GRID} must be added together with the keyword {\tt CV} to specify
the collective variable, {\tt MIN} and {\tt MAX} to fix  the CV interval, {\tt NBIN} for the number of bins
and the flag {\tt PBC} if the CV is periodic. 

\esempio{In this example we run metadynamics using only one CV, a distance between 
two atoms, and we put the bias potential on a grid of 200 bins in the interval between 0.0 and 10.0. \vspace{10pt} \\
{\tt
HILLS HEIGHT 0.1 W\_STRIDE 1000 \\
PRINT W\_STRIDE 50  \\
DISTANCE LIST 13 46 SIGMA 0.35 \\
GRID CV 1 MIN 0.0 MAX 10.0 NBIN 200 \\
ENDMETA
}}

Special labels can be used in the definition of the interval with {\tt MIN} and {\tt MAX}, such
as {\tt -pi, +pi, +2pi, -2pi, pi, 2pi}. These labels may be particularly useful with the CVs {\tt ANGLE} or
{\tt TORSION}.

\esempio{In this example we run metadynamics using a dihedral angle 
 and putting the bias potential on a periodic grid of 200 bins in the interval between -pi and pi. \vspace{10pt} \\
{\tt
HILLS HEIGHT 0.1 W\_STRIDE 1000 \\
PRINT W\_STRIDE 50  \\
TORSION LIST 13 15 17 19 SIGMA 0.35 \\
GRID CV 1 MIN -pi MAX +pi NBIN 200 PBC \\
ENDMETA
}}
As in standard metadynamics, a {\tt HILLS} file containing the list of Gaussians deposited is produced.
This file is needed for restarting a metadynamics simulation also when using \keyword{GRID}.
\\
The bias potential in a generic point is calculated by a polynomial interpolation
which has the proper values of the function and of its derivatives in $2^{d}$ points, where $d$ is the
number of collective variables. The forces are then obtained as the analytical derivatives of the bias.
You can switch off the use of splines by using
the directive \keyword{NOSPLINE}. In this case, the bias and forces are simply taken at a close grid point (this requires
a much denser grid).
\\
%To estimate the error due to the approximation of the bias potential on the grid,
%the keyword {\tt DEBUG\_GRID}  must be used. The file {\tt DEBUG\_GRID} will be produced
%with the value of the potential and the forces calculated both with the grid technique and summing analytically
%all the Gaussians. In order to use this feature, the keyword {\tt ENABLE\_UNTESTED\_FEATURES}
%must be on.

Please also note that:
\begin{itemize}
\item {\tt GRID} must be activated (or switched off) on ALL the CVs; 
\item {\tt GRID} can be used together with multiple walkers metadynamics, bias-exchange and
      parallel tempering metadynamics;
\item For a correct calculation of the potential and forces, the bin size must be smaller than
     half the Gaussian sigma. If a larger size is used, the code will stop. 
\item If the simulation goes out of the grid, the code will stop. Please increase {\tt MIN} or {\tt MAX} 
          and restart metadynamics.
\end{itemize}

\subsubsection{Writing a GRID to file}

The directive \keyword{WRITE\_GRID} allows to save on a file the grid on which the bias
potential is stored and the relative forces. 
The keyword \keyword{W\_STRIDE} can be used to specify the writing stride and
\keyword{FILENAME} the name of the file. Since saving the entire grid on file may take some 
time, a reasonable writing stride should be used. 

\esempio{The following command controls a metadynamics calculation using as CV the distance
between two atoms. The bias potential is stored on a grid and saved to the file {\tt bias.dat} every 
100000 steps. \vspace{10pt} \\ 
{ \tt HILLS HEIGHT 0.1 W\_STRIDE 1000 \\
  DISTANCE LIST 10 12 SIGMA 3.0 \\
  GRID CV 1 MIN 0.0 MAX 10.0 NBIN 10 \\
  WRITE\_GRID FILENAME bias.dat W\_STRIDE 100000}} 

The file on which the grid is saved has a specific format. 
A header contains information about the presence of force data,
the number and type of CVs, 
the grid dimension and boundaries and whether the CVs are periodic 
or not. The rest of the file contains for each grid point the value of the bias
potential and forces (i.e. minus the derivative of the potential
with respect to the CVs). 

\esempio{The header states that force data are on file, only 
one CV is present (a distance, see Tab. \ref{id_cv} for a legend). 
The grid is made of 10 bins ranging from 0 to 10 in unit of the CV.
This CV is not periodic.\vspace{10pt} \\
{ \tt 
\#! FORCE 1 \\
\#! NVAR 1 \\
\#! TYPE 1 \\
\#! BIN 10 \\
\#! MIN 0.000000 \\
\#! MAX 10.000000 \\
\#! PBC 0 \\
 0.000000  0.998838  -0.016081 \\
 1.000000  0.960195  0.091229 \\
 2.000000  0.825978  0.170252 \\
 3.000000  0.635806  0.201699 \\
 4.000000  0.437950  0.187594 \\
 5.000000  0.269942  0.145621 \\
 6.000000  0.148888  0.096861 \\
 7.000000  0.073484  0.055971 \\
 8.000000  0.032454  0.028326 \\
 9.000000  0.012826  0.012620 \\
 10.000000  0.004536  0.004967}}

This file can be used to plot the free energy resulting from a metadynamics
simulation instead of summing the Gaussians stored on the {\tt HILLS} file with the
post-processing code {\verb sum_hills }. When using well-tempered metadynamics, please
remember that the bias potential does not compensate exactly the underlying 
free energy \cite{Barducci:2008}. In this case, the potential written on file must be rescaled 
accordingly.

\subsubsection{Reading a GRID from file}

The bias potential stored on file can be used to restart a metadynamics simulation
by specifying the directive \keyword{READ\_GRID} and the file name with \keyword{FILENAME}.

\esempio{The following command controls a metadynamics calculation using as CV the distance
between two atoms. The initial bias potential is read from the file {\tt bias.dat}. \vspace{10pt} \\
{ \tt HILLS HEIGHT 0.1 W\_STRIDE 1000 \\
  DISTANCE LIST 10 12 SIGMA 3.0 \\
  GRID CV 1 MIN 0.0 MAX 10.0 NBIN 10 \\
  READ\_GRID FILENAME bias.dat}}

Please also note that:
\begin{itemize}
\item The CV number and type in the header must be consistent
with what declared in the \plumed \ input file;
\item The number of bins and the boundaries in the header
can be different from what declared in the \plumed \ input file.
In this case the parameters of the  \plumed \ input file will
be used and the grid present on file will be interpolated to fit
the new dimensions; 
\item If force data are not present on file, they will be
calculated from the bias potential using finite differences;
\item {\tt READ\_GRID} is not compatible with the {\tt MULTIPLE\_WALKERS} directive.  
\end{itemize}

Restarting a metadynamics run from a grid written on file
is fully compatible with the standard restart by reading a
{\tt HILLS} file created in a previous run.  

\esempio{The following command controls a metadynamics calculation using as CV the distance
between two atoms. The initial bias potential is read from the file {\tt bias.dat}. 
To this initial bias, the Gaussians deposited on the {\tt HILLS} file are added. \vspace{10pt} \\
{ \tt HILLS RESTART HEIGHT 0.1 W\_STRIDE 1000 \\
  DISTANCE LIST 10 12 SIGMA 3.0 \\
  GRID CV 1 MIN 0.0 MAX 10.0 NBIN 10 \\
  READ\_GRID FILENAME bias.dat}}

\subsection{Multiple walkers}
\label{ssc.multiplewalkers}
The {\tt MULTIPLE\_WALKERS} directive sets the multiple walkers \cite{multiplewalkers} running mode. 
All the processes must have the same CVs, in the same order.
Each process will write its own bias potential in the
directory specified by the {\tt HILLS\_DIR} keyword and in a file
named {\tt HILLS.0}, {\tt HILLS.1}, etc etc. Multiple processes must be
launched independently and must point to the same directory {\tt HILLS\_DIR}, so that each one will
contribute to the total bias potential.

The keyword {\tt NWALKERS} sets the total number of walkers. This can be set to
a value greater than the actual number of processes running and it can be
increased during the run. {\tt ID} determines a unique id of the walker,  starting from 0.
The {\tt R\_STRIDE} keyword sets the stride (in time steps) at which each individual bias 
potential is updated by reading all the {\tt HILLS} files contained in {\tt HILLS\_DIR}.
\esempio{The following command controls a multiple walkers calculation using a maximum
of 10 walkers. Gaussians are added every 1000 steps and updated every 5000 steps. \vspace{10pt} \\ 
{ \tt HILLS HEIGHT 0.1 W\_STRIDE 1000 
 \\
 MULTIPLE\_WALKERS  HILLS\_DIR /scratch/HILLS R\_STRIDE 5000 NWALKERS 10 ID 0}}


\subsection{Monitoring a collective variable without biasing it}
The directive \keyword{NOHILLS} can be used to monitor a CV 
during a metadynamics run without applying a bias on it. 
The keyword {\tt CV} must be specified to select the CV to be monitored. 
This directive must be used for metadynamics simulations with Bias-Exchange.
\esempio{In this example we run metadynamics using only one CV, a distance between 
two atoms. During the simulation we monitor the behavior of another CV, the dihedral
defined by a set of four atoms, without putting a bias on it.  \vspace{10pt} \\
{\tt
HILLS HEIGHT 0.1 W\_STRIDE 1000 \\
PRINT W\_STRIDE 50  \\
DISTANCE LIST 13 46 SIGMA 0.35 \\
TORSION LIST 1 4 65 344 SIGMA 0.1 \\
NOHILLS CV 2\\
ENDMETA
}}
As an alternative you may also avoid the SIGMA value in the CV. This will be understood
as a directive that no hills must be put on this CV.
The previous example therefore becomes:
\vspace{10pt} \\
\esempio{
\\
{\tt
HILLS HEIGHT 0.1 W\_STRIDE 1000 \\
PRINT W\_STRIDE 50  \\
DISTANCE LIST 13 46 SIGMA 0.35 \\
TORSION LIST 1 4 65 344  \\
ENDMETA
}
}
where the \keyword{NOHILLS} keyword for CV 2 has been omitted as implicitly 
included by the missing \keyword{SIGMA} parameter.

\vspace{1cm}

\subsection*{Avoiding systematic errors at the boundaries}

Metadynamics is often used to reconstruct the free energy as a function
of intrinsically limited CVs (like e.g. \keyword{COORD} and \keyword{ALPHABETA}, see chapter \ref{ch.generalcv})
or articially limited CVs (using for example an external potential,
see section \ref{ext_pot}). In this case the finite-width Gaussians used in
metadynamics can induce systematics errors at the boundaries. These
errors become more and more severe with time, and are due to the fact
that a sum of a finite number of Gaussians can not reproduce a free-energy
profile with large or infinite derivative. However, \plumed \ 
implements two different procedures aimed at avoiding the onset of
these systematic errors, activated respectively by the keyword \keyword{INTERVAL}
and by the keyword \keyword{INVERT}.


\subsection{Defining an interval}

%The keyword \keyword{INTERVAL} activates a fast algorithm that effectively limits the exploration range along a single collective variable with a minimal perturbation of the resulting FES. The region is selected by using {\verb LOWER_LIMIT } and {\verb UPPER_LIMIT }, these limits must be chosen inside the physical meaningful region in order to work. Given a region, the Gaussians will be added both within as outside the boundaries, but the forces will be evaluated only if the system is inside the interval. The result is that the system will not accumulate Gaussians on the borders in the trial to pass over and the sampling will be automatically limited inside the defined region.  The reconstructed FES will be meaningful only inside the interval region. The \keyword{INTERVAL} keyword can be used only on a single CV.

With the keyword \keyword{INTERVAL} one changes the metadynamics algorithm setting
the bias force equal to zero beyond a boundary \cite{baftizadeh}. If, for example,
metadynamics is performed on a CV $s$ and one is interested only
to the free energy for $s>s_{w}$, the history dependent
potential is still updated according to Eq.~\ref{bias}, 
but the metadynamics force is set to zero for $s<s_{w}$:
\[\frac{\partial V_{G}\left(s,t\right)}{\partial s}=0\ \ \ \forall s<s_{w}\]
Notice that Gaussians
are added also if $s<s_{w}$, as the tails of these Gaussians influence
$V_{G}$ in the relevant region $s>s_{w}$. In this way,
the force on the system in the region $s>s_{w}$ comes from both metadynamics
and the force field, in the region $s<s_{w}$ only from the latter.
This approach allows obtaining a history-dependent bias potential
$V_{G}$ that fluctuates around a stable estimator, equal
to the negative of the free energy far enough from the boundaries~\cite{baftizadeh}.

\begin{itemize}
\item In order to obtain a parallel growing $V_{G}$, one should place the interval
limit $s_{w}$ in a region where the free energy derivative is not
large. For example, if one uses the keyword with a coordination number
CV (see Section~\ref{coordnum}), that is intrinsically limited to $s>0$, the
correct choice for $s_{w}$ is not 0, but a number of the order
of the width of the Gaussians ($\sigma$ in Eq.~\ref{bias}).
\item This remedy has the advantage of being robust and parameter-free,
but works only for one-dimensional biases.
\item If in the region $s<s_{w}$ the system has a free energy minimum,
the \keyword{INTERVAL} keyword should be used together with a soft wall at $s_{w}$
(keyword \keyword{LWALL}, see section~\ref{ext_pot}), namely a time-indipendent external
potential that prevents the system from going there and remaining
trapped in the minimum. An example is provided below.
\end{itemize}

\esempio{The following input file contains the interval condition in presence of soft walls: \vspace{10pt} \\
{\tt
HILLS HEIGHT 0.1 W\_STRIDE 500 \\
COORD LIST <g1> <g2> NN 8 MM 12 D\_0 0 R\_0 0.25 SIGMA 0.15 \\
g1->\\
26 27\\
g1<-\\
g2->\\
29 30\\
g2<-\\
INTERVAL CV 1 LOWER\_LIMIT 0.2 UPPER\_LIMIT 2.8 \\
UWALL CV 1 LIMIT 2.8 KAPPA 3000.0 \\
LWALL CV 1 LIMIT 0.2 KAPPA 3000.0 \\
ENDMETA \\
}
}

\subsection{Inversion condition}
The \keyword{INVERT} keyword activates the inversion condition near predefined boundaries {\verb LIMIT1, \verb LIMIT2 }\cite{marinell-crespo10}.
To simplify the notation, we here assume the CVs space is one-dimensional 
with the boundary at $s=0$. The inversion condition ensure that near $s=0$ the bias potential satisfy the following relation:
\begin{equation}
V(-s,t)\approx2V(0,t)-V(s,t)\label{eq:inversion}\end{equation}
This property ensures that, at stationary conditions, the history-dependent
potential is approximately linear close to the boundary, but it does
not impose the value of its derivative, that is iteratively determined
by the thermodynamic bias. In practice this is achieved by adding extra Gaussians out
of the boundaries according to the following roles:

\begin{itemize}
\item An interval centered in $0$ is chosen, whose width $\chi_{1}$ is of
the order of $\sigma$ (Gaussian width for the selected CV; \keyword{SIGMA}).
\item If $s<\chi_{1}$ another Gaussian centered in $-s$ and with the same
width and height is added.
\item If $ s > \chi_{1}$ another Gaussian centered in $-s$ and with the same width is added.
In this case, the height of the extra Gaussians depends
on $V$ and is given by:
 \begin{equation}\label{boundary}
   w  =   \left[   2V(0,t) - V(-s,t) - V(s,t) \right]  y(s), 
\end{equation}
 where $y(s) =   1 / \left[ 1 + \left( s/(\chi_{2} \chi_{1}) \right) ^{10}\right] $ and $\chi_{2}>\chi_{1}$
\end{itemize}
The second factor in Eq. (\ref{boundary}) is approximately one for $\mid s \mid< \chi_{2} \chi_{1}$ and goes to zero for $ \mid s \mid > \chi_{2} \chi_{1}$. This
ensures that $V$ goes smoothly to zero out of the boundaries. 
$\chi_{2}$$\chi_{1}$ thus defines the width of the inversion interval and $\chi_{2}$ is a scaling factor regulated by the keyword {\verb INVERSION }.
$\chi_{1}$ is regulated by the keyword {\verb REFLECTION } ($\chi_{1}$ is in $\sigma$ units).
The keyword {\verb MAXHEIGHT } defines the largest $\mid w \mid$ in gaussian height units and it regulates the speed of variation of $V$ out of the boundaries.
The \keyword{INVERT} keyword can be used in presence of an external potential that limits the CVs space exploration (see section \ref{ext_pot}) like in the example below;  
\esempio{The following input file contains the inversion condition in presence of soft walls: \vspace{10pt} \\
{\tt
HILLS HEIGHT 0.1 W\_STRIDE 500 \\
COORD LIST <g1> <g2> NN 8 MM 12 D\_0 0 R\_0 0.25 SIGMA 0.15 \\
g1->\\
26 27\\
g1<-\\
g2->\\
29 30\\
g2<-\\
INVERT CV 1 REFLECTION 1.6 INVERSION 6 MAXHEIGHT 4 LIMIT1 0.2 LIMIT2 2.8 \\
UWALL CV 1 LIMIT 2.8 KAPPA 3000.0 \\
LWALL CV 1 LIMIT 0.2 KAPPA 100000.0 \\
ENDMETA \\
}
}
or if an intrinsically limited CV is used (see chapter \ref{ch.generalcv}):

\esempio{The following input file contains the inversion condition in presence of an intrinsically limited CV, such as ALPHABETA: \vspace{10pt} \\
{\tt
HILLS HEIGHT 0.1 W\_STRIDE 250 \\
ALPHABETA NDIH 1 SIGMA 0.05 \\
5 12 14 23 -3.14159 \\
INVERT CV 1 REFLECTION 1.6 INVERSION 6 MAXHEIGHT 4 LIMIT1 0.0 LIMIT2 1.0 \\
ENDMETA \\
}
}

If \keyword{INVERT} is used in presence of soft walls it is recommended to use KAPPA larger than $KT/\sigma^{\mathtt{EXP}}$, where $\mathtt{EXP}$ is the exponent defined for the wall (see section \ref{ext_pot}). 
Much smaller values for KAPPA in fact do not efficiently prevent the exploration of CVs regions that are located at more than $\sigma$ beyond the bounduaries and this may lead to systematics errors.
It is worth to note that {\verb LIMIT1 } and {\verb LIMIT2 } must not be in unphysical regions of the CVs space, i.e.
they must be visited during the simulation. The present implementation does not remove systematics errors for multidimensional
metadynamics in regions of the CVs space that are near crossing boundaries (e.g. if LIMIT1=0 for CV 1 and LIMIT1=0 for CV 2 there still can
be systematic error near (0,0) ). The values for the {\verb REFLECTION }, {\verb INVERSION } and {\verb MAXHEIGHT } keywords chosen in the examples are 
also their default values (if not specified the program selects automatically their default values). They were chosen in other to minimize the free energy error 
at the boundaries for several different test cases. Further information can be found in ref.\cite{marinell-crespo10}. 


\section{Running in parallel}

Simulations of large systems can often accelerated by using parallel machines.
The behavior of \plumed \ in parallel simulations is dependent on the host code:
\begin{itemize}
\item NAMD: the present version of \plumed \ is fully compatible with NAMD for parallel simulations. However, since
\plumed \ runs on the first processor only, the computational effort required to evaluate the collective variables and their
derivatives, as well as the history dependent potential in metadynamics simulations, should be kept
lower than $1/N_p$ of the total computational effort, where $N_p$ is the number of processors.
Thus, pay attention to heavy variables and metadynamics simulations with many hills.
\item GROMACS, DL\_POLY, AMBER and LAMMPS: the present version of \plumed \  is fully compatible with these codes for parallel simulations.
Consider the following notes:
\begin{itemize}
\item The coordinates of the particles involved in collective variables are replicated over all nodes
at every step. If you don't want to slow down your simulation, minimize the number of involved atoms.
\item The computational effort required to evaluate the collective variables and their
derivatives is replicated on all processors, thus is effectively not scaling. Pay attention to
heavy variables (such as coordination numbers with long lists).
\item The computational effort required to evaluate the history dependent potential
in metadynamics simulations is spread over processors, thus should scale linearly.
\item With GROMACS and domain decomposition, pay attention to periodic boundary conditions (see Section.~\ref{PBC}).
\item With DL\_POLY, use the patched version of MakePAR to compile \plumed.
\end{itemize}
\end{itemize}

\section{Replica exchange methods}

When combined with GROMACS (both version 4.0 and 4.5), \plumed \  can perform replica--exchange
simulations coupled with metadynamics in two different ways:
parallel tempering metadynamics (PTMetaD)~\cite{bussi_xc,camilloni_protG} and
bias-exchange metadynamics (BE-META)~\cite{piana}.

\subsection{Parallel tempering metadynamics}

Parallel tempering metadynamics \index{Parallel tempering metadynamics}
(currently implemented only for the GROMACS engine) is selected using the
\keyword{PTMETAD} directive. 

To run parallel tempering simulations with metadynamics, one has to follow the
standard GROMACS procedure for parallel tempering (see GROMACS manual). A
binary topology {\tt .tpr} file must be prepared 
for each replica, while only one plugin input file is required.

\esempio{The following input file defines a parallel-tempering metadynamics: \vspace{10pt} \\
{\tt 
\# switching on metadynamics and Gaussian  parameters \\
HILLS HEIGHT 0.1 W\_STRIDE 500 \\
\# switching on parallel tempering \\
PTMETAD \\
\# instruction for CVs printout \\
PRINT W\_STRIDE 50 \\
\# the CV: radius of gyration \\
RGYR LIST <CA> SIGMA 0.1 \\
CA-> \\
20 22 26 30 32 \\
CA<- \\
\# end of the input \\
ENDMETA  }
}

The {\tt PTMETAD} directive switches on parallel tempering.  
All replicas have the same CVs, in this particular case 
the radius of gyration defined by the group of atoms {\tt <CA> }. 

The Gaussian height set by the keyword {\tt HEIGHT} is
automatically rescaled with temperature, following $W_i=W_0\frac{T_i}{T_0}$,
where $i$ is the index of a replica and $T_i$  its temperature. 

Similarly, the simulation temperature needed to use the well-tempered algorithm
(which, for non-parallel simulations is set with the \keyword{SIMTEMP} keyword)
is here taken directly from the GROMACS input at each replica.
As a result, the value of $\Delta T$ for the well-tempered algorithm is rescaled across the replicas.

The plugin will produce one {\tt COLVAR } file and one {\tt HILLS } file for each replica.

\subsection{Bias exchange simulations}
\label{sec.beparallel}

Bias exchange simulations \index{Bias exchange simulations} 
must be run using the \keyword{BIASXMD} directive. 
Only the GROMACS engine currently has this feature implemented within a single parallel run, 
whereas the bias-exchange tool in the directory {\tt utilities} allows to perform bias-exchange simulations
with any MD engine via the linux shell (see Section~\ref{sec.betool}).

The procedure for running bias-exchange simulations is similar to the one described
earlier for parallel tempering (i.e., add {\tt -multi nrep -replex nexch} to the mdrun command line, where
{\tt nrep} is the number of replicas and {\tt nexch} is the number of steps between attempting exchanges).  
One plugin input and one binary
topology file must be provided for each replica.  These files must be named
with the replica index (e.g., {\tt plumed0.dat}, {\tt plumed1.dat}, ..., {\tt md0.tpr}, {\tt md1.tpr}, ...).
Each plugin input file must contain the directive \keyword{BIASXMD}
and the same list of CVs, in the same order.
Each replica usually has only one (or few) of the CVs active: use a \keyword{NOHILLS} directive
for each of the inactive CVs (or do not specify \keyword{SIGMA} for them).
It is also possible to have replicas with all CVs inactive: such unbiased replicas can exchange 
with biased ones, jumping in this way beyond free-energy barriers, 
but since they are not subject to the hills they tend to populate the free-energy basins according to
equilibrium statistics.

\esempio{The following two input files define a bias-exchange metadynamics run: \vspace{10pt} \\
{\tt 
\# --- input file plumed0.dat ---\\
\# switching on metadynamics and Gaussian  parameters \\
HILLS HEIGHT 0.1 W\_STRIDE 500 \\
\# switching on bias-exchange \\
BIASXMD \\
\# instruction for CVs printout \\
PRINT W\_STRIDE 50 \\
\# instruction for labeling of COLVAR and HILLS files \\
HILLS\_LABEL A \\
\# the CVs: dihedral angles \\
TORSION LIST  1  5 11 13 SIGMA 0.314 \\
TORSION LIST 11 13 15 21 SIGMA 0.314 \\
\# in this replica we bias only CV 1: \\
NOHILLS CV 2\\
\# end of the input \\
ENDMETA \\ 
\vspace{10pt}\\
\# --- input file plumed1.dat ---\\
\# switching on metadynamics and Gaussian  parameters \\
HILLS HEIGHT 0.1 W\_STRIDE 500 \\
\# switching on bias-exchange \\
BIASXMD \\
\# instruction for CVs printout \\
PRINT W\_STRIDE 50 \\
\# instruction for labeling of COLVAR and HILLS files \\
HILLS\_LABEL B \\
\# the CVs: dihedral angles \\
TORSION LIST  1  5 11 13 SIGMA 0.314 \\
TORSION LIST 11 13 15 21 SIGMA 0.314 \\
\# in this replica we bias only CV 2: \\
NOHILLS CV 1\\
\# end of the input \\
ENDMETA  }
}

In bias-exchange runs, the plugin will produce one {\tt COLVAR} file and one {\tt
HILLS} file for each replica. What is actually exchanged between replicas are the atomic
coordinates, not the bias, so that replica 0 (printing out {\tt COLVAR0}) will always be biased by {\tt HILLS0}, 
replica 1 (printing out {\tt COLVAR1}) will always be biased by {\tt HILLS1},) and likewise for the other replicas.

In the example above, note the use of the keyword \keyword{HILLS\_LABEL}: this will print a comment line 
"{\tt \#! ACTIVE 1 1 A}" in files {\tt COLVAR0} and {\tt HILLS0} and "{\tt \#! ACTIVE 1 2 B}" in files {\tt COLVAR1} and {\tt HILLS1},
indicating the number of active CVs, their index, and the chosen label. 
This facilitates the reconstruction of the multidimensional free-energy landscape from the bias-exchange simulation
according to the weighted-histogram algorithm in Ref.~\cite{marinelli-trpc09},
e.g. employing the METAGUI program~\cite{METAGUI} downloadable from \\ \url{http://www.plumed-code.org/contributions}

See also Section~\ref{sec.betool} for informations on bias-exchange simulations with MD codes different from GROMACS.

\newpage

\section{Umbrella sampling}
 The directive \keyword{UMBRELLA} \index{Umbrella sampling} allows
 umbrella sampling calculations to be performed on
 the CV specified by the parameter keyword {\tt CV}.
 The position $s_0$ of the umbrella restraint is determined by
 the keyword {\tt AT}, and the spring constant $k$ -
  in internal unit of the main code -  by the keyword {\tt KAPPA}.
Optionally, also a constant force $m$ can be specified
by the keyword {\tt SLOPE}.
 This turns on a potential of the following functional form:
 \begin{equation}
 V_{\rm umb}(s)=\frac{1}{2} {k}(s- s_0)^2 + m(s-s_0).
\label{umbrella_potential}
 \end{equation} 

 \esempio{The following input file defines an umbrella potential acting on the 
 first CV and centered in $s_0=-1.0$.  \vspace{10pt}\\
{\tt TORSION LIST 13 15 17 1 \\
UMBRELLA CV 1 KAPPA  200 AT -1.0 \\
PRINT W\_STRIDE 100 \\
ENDMETA}}

In the case of umbrella sampling runs, 
 the value of the CVs is printed on the {\tt COLVAR} file with a stride
 fixed by the directive {\tt PRINT} and the keyword {\tt W\_STRIDE}. 
The file contains the time, the CV values, the
metadynamics potential, the harmonic potential of umbrella sampling,
the CV on which the restraint acts and the position of the restraint.
The final calculation of the free energy as a function of this 
CV can be done using the weighted histogram analysis method,
 choosing one of the many possible 
 implementations, for instance the wham code by Alan Grossfield 
 \cite{grossfield}.
 The directive {\tt UMBRELLA} can be used multiple times in the case of  multidimensional
 umbrella sampling calculations (one directive for each CV).

The keyword {\tt RESTART} can be used when restarting an umbrella
sampling calculation to append the value of the CVs on the {\tt COLVAR} file.

\newpage
\section{Steered MD}
\index{Steered MD}
\plumed \ can be used to drag a system to a target value in
CV space using an harmonic potential moving at constant speed.
If the process is reversible, \emph{i.e.} for velocities tending to zero, the work
done in the dragging corresponds to the free energy.  In the case of finite
velocity it is possible to obtain an estimate of the free-energy from the work
distribution using Jarzynski \cite{jarzynski} or Crooks \cite{Crooks98}
relations.

The directive \keyword{STEER} activates the steering on the collective
variable specified by the keyword {\tt CV}. The target value is determined by
the keyword {\tt TO}, the velocity, in the same unit as the corresponding CV every 1000 steps,
by {\tt VEL} and the spring constant by {\tt KAPPA}. An additional keyword {\tt FROM} 
can be used to specify the starting point of the dragging, otherwise the CVs values at the first step
are taken as the starting point.
The functional form of the dragging potential is the same as the one of
formula \ref{umbrella_potential}, with the reference position $s_0$ moving 
at the specified velocity.

The keyword {\tt RESTART} can be used when restarting a steered MD
calculation to append the value of the CVs on the {\tt COLVAR} file.

\esempio{The following input file defines a steered MD on the angle CV to a
target value of 3.0 rad.  \vspace{10pt}\\
{\tt
ANGLE LIST 13 15 17 \\
STEER CV 1 TO 3.0 VEL 0.5 KAPPA 500.0 \\
PRINT W\_STRIDE 100 \\
ENDMETA}

}


\subsection{Steerplan}
\index{Steerplan}
\plumed \ can be used to drag a system on a pathway that is 
composed of successive steering runs on chosen degrees of freedom
in a planned fashion. While the directive \keyword{STEER} is rather easy and intuitive, 
\keyword{STEERPLAN} is more flexible and it allows to avoid a lot of scripting 
whenever you plan to simulate complex transitions by means of out-of-equilibrium runs.

The directive \keyword{STEERPLAN} activates this option and reads the name 
of a file that contains the plan as an input. 

%calculation to append the value of the CVs on the {\tt COLVAR} file.

\esempio{The following input file contains many collective variables and a steerplan directive. 
 \vspace{10pt}\\
{\tt

PRINT W\_STRIDE 5 \\
\#\\
\# these CVs are put here just to show that you may use more variable than the\\ 
\# ones defined in the steerplan \\
\#\\
TARGETED TYPE MSD FRAMESET restrained.pdb \\ 
TARGETED TYPE MSD FRAMESET waters.pdb \\
TARGETED TYPE MSD FRAMESET ref\_H.pdb \\
\#\\
\# difference of distances for proton transfers \\
\#\\
DISTANCE LIST  53 703 DIFFDIST 703 702 \\ 
DISTANCE LIST 702 913 DIFFDIST 913 912 \\
\#\\
\# steer plan read from file myplan\\
\#\\
STEERPLAN myplan\\
ENDMETA\\
}}

The steerplan file {\tt myplan} contains the control sequence of the steerplan and it is composed as follows.
The first column contains the time (in timeunits of the program) at which a particular action is planned. The following columns are 
composed in blocks of five. Each block contains: the {\tt CV} keyword, the index of the CV to be used, the spring
constant at a given time, the value of the center of the spring potential and a keyword among  {\tt CENTRAL} (the potential is
a normal parabolic shape), {\tt POSITIVE} (the potential with parabolic shape is applied only when the CV value is higher than the center of the 
spring) and {\tt NEGATIVE} (the potential with parabolic shape is applied only when the CV value is lower than the center of the   
spring). A Block of four columns without the latter keyword is also accepted. In this case {\tt CENTRAL} is assumed.   
Wildcards ({\tt *}) are also accepted for the position. Their meaning is much clearer in the following example. 
Comments are allowed  ({\tt \#}) .
\esempio{An example of steerplan file.
 \vspace{10pt}\\
{\tt
\# Bring CV 4 from wherever it is (*) to -0.5 at 2 ps by increasing \\ 
\# gradually the spring constant from 0 to 800. CV 5 does the same and goes to -0.64.  \\ 
\begin{tabular} {r  r r r r r  r r r r r  r r r r r }
      0.00 & CV &  4 &  000.0 &   *    &   POSITIVE &  CV &  5 &  000.0 &   *     &   POSITIVE   \\
   2000.00 & CV &  4 &  800.0 &  -0.5  &   POSITIVE &  CV &  5 &  800.0 &   -0.64 &   POSITIVE   \\
\end{tabular}\\
\# Now in 400 fs keep CV 4 at its value while releasing the other \\
\# slowly to 0 spring constant.    \\ 
\begin{tabular} {r  r r r r r  r r r r r  r r r r r }
   2400.00 & CV &  4 &  800.0 &  -0.5  &   POSITIVE &  CV &  5 &    0.0 &   -0.64 &   POSITIVE   \\
\end{tabular}\\
\# Now drag back only CV 4 to 1.4 with a value that acts only \\
\# on the negative part. The potential on CV 5 is off.  \\
\begin{tabular} {r  r r r r r  r r r r r  r r r r r }
   8400.00 & CV &  4 &  800.0 &   1.4  &   NEGATIVE &  CV &  5 &    0.0 &   -0.64 &   POSITIVE   \\
\end{tabular}\\
}
} 

Please note that the wildcards have particular meaning:
\esempio{Here the dragging force starts from whatever value the system assumes at 0 fs 
and drags it to -0.5 at 2 ps. The spring constant is linearly increasing from 0.0 to 800.0. 
When the starting point is a wildcard this takes the current value. 
 \vspace{10pt}\\
{\tt
\begin{tabular} {r  r r r r r  }
      0.00 & CV &  4 &  000.0 &   *    &   POSITIVE    \\
   2000.00 & CV &  4 &  800.0 &  -0.5  &   POSITIVE    \\
\end{tabular}\\
}
 \vspace{10pt}\\
Here the center of the harmonic potential follows the system at each step from 0 to 2 ps. 
When the position of the ending point is a wildcard then the potential will 
follow the coordinates (this means that the potential is not applied as the force is zero everywhere). 
 \vspace{10pt}\\
{\tt
\begin{tabular} {r  r r r r r }
      0.00 & CV &  4 &  800.0 &   *    &   POSITIVE    \\
   2000.00 & CV &  4 &  800.0 &   *    &   POSITIVE    \\
\end{tabular}\\
}
 \vspace{10pt}\\
This case is considered identical as the one before \\
 \vspace{10pt}\\
{\tt
\begin{tabular} {r  r r r r r  }
      0.00 & CV &  4 &  800.0 &   0.5  &   POSITIVE    \\
   2000.00 & CV &  4 &  800.0 &   *    &   POSITIVE    \\
\end{tabular}\\
}
}


\section{Adiabatic Bias MD}
\index{Adiabatic bias molecular dynamics}
\plumed \ can be used to evolve a system towards a target value in
CV space using an harmonic potential moving with the thermal fluctuations of the CV\cite{ballone,provasi10abmd,camilloni11abmd}.

The directive \keyword{ABMD} activates the biasing on the collective
variable specified by the keyword {\tt CV}. The target value is determined by
the keyword {\tt TO}, the spring constant by {\tt KAPPA}. 
The biasing potential is implemented as
\begin{equation}
V(\rho(t)) = \left \{ \begin{array}{ll} \frac{\alpha}{2}\left(\rho(t)-\rho_m(t)\right)^2, &\rho(t)>\rho_m(t)\\
              0, & \rho(t)\le\rho_m(t), \end{array} \right .
\end{equation}
where
\begin{equation}
\rho(t)=\left(CV(t)-TO\right)^2
\end{equation}
and
\begin{equation}
\rho_m(t)=\min_{0\le\tau\le t}\rho(\tau).
\end{equation}
The method is based on the introduction of a biasing potential which is zero when
the system is moving towards the desired arrival point and which damps the
fluctuations when the system attempts at moving in the opposite direction. As in the
case of the ratchet and pawl system, propelled by thermal motion of the solvent
molecules, the biasing potential does not exert work on the system.

The keyword {\tt RESTART} can be used when restarting an adiabatic bias MD
calculation to append the value of the CVs on the {\tt COLVAR} file and
to set the best value previously reached.

\esempio{The following input file defines an adiabatic biased MD on the angle CV to a
target value of 3.0 rad.  \vspace{10pt}\\
{\tt
ANGLE LIST 13 15 17 \\
ABMD CV 1 TO 3.0 KAPPA 500.0 \\
PRINT W\_STRIDE 100 \\
ENDMETA}
\\
The corresponding {\t COLVAR} will look like:\\
\\
{\tt
\#! FIELDS time cv1 vbias vwall vext XX XX ABMD1\\
     0.000      1.884703657      0.000000000      0.000000000      0.000000000 ABMD 1    1.243885933\\
     0.200      2.353724409      0.000000000      0.007812004      0.000000000 ABMD 1    0.412082147\\
     0.400      2.433164746      0.000000000      0.000000000      0.000000000 ABMD 1    0.321302206\\
     0.600      2.608463835      0.000000000      0.004202352      0.000000000 ABMD 1    0.149200641\\
     0.800      2.818935636      0.000000000      0.009463992      0.000000000 ABMD 1    0.026631583\\
     1.000      2.831631093      0.000000000      0.147609095      0.000000000 ABMD 1    0.004049192\\
     1.200      2.583559043      0.000000000      7.171877688      0.000000000 ABMD 1    0.004049192\\
     1.400      2.841442463      0.000000000      0.121111568      0.000000000 ABMD 1    0.003130352\\
     1.600      2.794514101      0.000000000      0.382087213      0.000000000 ABMD 1    0.003130352\\
     1.800      2.634418597      0.000000000      4.258829096      0.000000000 ABMD 1    0.003130352\\
     2.000      2.861303394      0.000000000      0.086113847      0.000000000 ABMD 1    0.000677239\\
}
\\where the last column is $\rho_m(t)$.
}

\section{External potentials} \label{ext_pot}
\subsection{Walls} \label{restri} \index{Walls}
The \keyword{UWALL} and \keyword{LWALL} keywords define a wall for the value 
of the CV $s$ which limits the region of the phase space accessible during the simulation.
The restraining potential starts acting on the system when the value of 
the CV is greater (in the case of \keyword{UWALL}) or lower (in the case of \keyword{LWALL}) than a certain
limit {\verb LIMIT } minus an offset {\verb OFF }.

The functional form of this potential is the following:
\begin{equation}
V_{wall}(s)=\mathtt{KAPPA}  \left (\frac{s- \mathtt{LIMIT}+ \mathtt{OFF}}{\mathtt{EPS}} \right)^{\mathtt{EXP}}, 
\end{equation}
where {\verb KAPPA } is an energy constant in internal unit of the code,
{\verb EPS } a rescaling factor and {\verb EXP } the exponent determining the power law.
\\
By default: $\mathtt{EXP}=4, \; \mathtt{EPS}=1.0, \; \mathtt{OFF}=0$.

\esempio{
To run a well-tempered metadynamics simulation using as CV
the distance between one atom and the center of mass of a group of
atoms and limiting its value below $ 15 \,\AA$, you have to use the
following input file:  \vspace{10pt} \\ 
{\tt
HILLS HEIGHT 0.1 W\_STRIDE 100 \\ 
WELLTEMPERED SIMTEMP 300 BIASFACTOR 10 \\
PRINT W\_STRIDE 50 \\
DISTANCE LIST 13 <g1> SIGMA 0.35\\
g1->\\
17 20 22 30\\
g1<-\\
UWALL CV 1 LIMIT 15.0 KAPPA 100.0 \\
ENDMETA
}
}

\subsection{Tabulated potentials} \label{tabu} \index{Tabulated potentials}

An external potential of generic form can be added 
to any collective variables using the directive \keyword{EXTERNAL}. 
The user must specify the total number of
CVs on which the potential acts with the keyword {\tt NCV}
and the variables with {\tt CV}.
The external potential must be provided in a tabulated
form in the file specified by {\tt FILENAME}. 
The format used for the external potential is the same as the one described in \ref{grid}
for the case of metadynamics potential on a {\tt GRID}. 

\esempio{
The following input file controls a simulation with an external potential acting on 
two collective variables. The tabulated potential is provided in the
file {\tt external.dat}.  \vspace{10pt} \\
{\tt
PRINT W\_STRIDE 100 \\
DISTANCE LIST 13 20 \\
TORSION  LIST 5 8 10 12 \\
EXTERNAL NCV 2 CV 1 2 FILENAME external.dat
}}

Every collective variables implemented in \plumed \ has a unique ID. 
This number must be used to define the type of CV in the header of the external
potential file and must match the CV activated in the \plumed \ input file. 
In Tab. \ref{id_cv} we provide a legend. 

\esempio{
The header of the external potential file of the previous example
looks like this: \vspace{10pt} \\
{\tt
\#! FORCE 1 \\
\#! NVAR  2 \\
\#! TYPE   1    5 \\
\#! BIN  100  100 \\
\#! MIN  0.0  -3.14159 \\
\#! MAX 10.0   3.14159 \\
\#! PBC    0    1 
}
}

\begin{table}
\begin{center}
\begin{tabular}{ c | c }
  {\bf ID}    & {\bf CV}  \\
\hline
1   & Distance\\
2   & Minimum distance\\
3   & Coordination number\\
4   & Angle\\
5   & Torsion\\
6   & Alpha-beta similarity\\
7   & Hydrogen bonds\\
8   & Dipole\\ 
11  & Radius of gyration\\
%14  & Torsional RMSD\\
16  & Dihedral correlation\\
20  & Interfacial water\\
30  & Path collective variable S\\
31  & Path collective variable Z\\
32  & Absolute position\\
33  & Electrostatic potential\\
34  & Puckering coordinates\\
35  & Energy\\
36  & Helix loops \\
37  & Alpha helix rmsd \\
38  & Antiparallel beta rmsd \\
39  & Parallel beta rmsd \\
42  & PCA projection\\
45  & Contact Map\\
55  & SPRINT \\
\hline
\end{tabular}
\caption{ID of the collective variables implemented in \plumed.}\label{id_cv}
\end{center}
\end{table}

If force data are not present on file, they will be
calculated from the tabulated potential using finite differences.

\clearpage

\section{Commitment analysis}
The \keyword{COMMITMENT} directive is used to run commitment analysis. 
The keyword {\tt NCV} determines the total number of CVs for
the analysis, while {\tt CV} must be used to specify the variable id. 
Following this line,  {\tt NCV} lines must be provided, each of which containing 
the upper and lower limits of the ${\cal A}$ and  ${\cal B}$ basins for the $i$-th variable.
\esempio{The following line defines a commitment analysis on the first two collective
variables $s_1$ and $s_2$, while the third is only monitored. The commitment basins are defined as 
   ${\cal A} = \{(s_1, s_2) | s_1 \in (0,1), s_2 \in (-1,1)\}$, and 
   ${\cal B} = \{(s_1, s_2) | s_1 \in (1,2), s_2 \in (-1,1)\}$.
 \vspace{10pt} \\
  { \tt 
COMMITMENT NCV 2 CV 1 2\\
     0.0 1.0 1.0 2.0  \\
    -1.0 1.0 -1.0 1.0 \\
\\
DISTANCE LIST 12 30 \\
DISTANCE LIST 20 24 \\
TORSION  LIST 14 16 20 22 \\
ENDMETA }}
%\section{Commitment analysis}
%The \keyword{STOPWHEN} directive can be useful whenever one runs interface sampling methods.
%It looks like committment and probably will be extended and fused with it in the near future.


\section{Projection of gradients}
The keyword \keyword{PROJ\_GRAD} is enable one to calculate the projection of the gradient in this way:
\begin{equation}
P_{ij}(R)=\sum_n^{NAT} \nabla_{n} s_{i}(R) \cdot \nabla_{n} s_{j}(R)
\end{equation}
where $ s_i(R) $ and $  s_j(R) $ are two collective coordinates defined in input.
The output is stored into the COLVAR file in the form of upper diagonal matrix.
The initial line of  \keyword{COLVAR} file describes the output in detail. 
\esempio{The following line instructs plumed to calculate the projection of two collective 
variables $s_1$ and $s_2$.
 \vspace{10pt} \\
  { \tt 
DISTANCE LIST 1 2 \\ 
TORSION  LIST 1 2  3 4 \\
TORSION  LIST  2  3 4 8 \\
PROJ\_GRAD CV <g1> \\
g1-> \\
2 3 \\
g1<- \\
ENDMETA }}
In the  \keyword{COLVAR} file you find a  \keyword{PROJ\_GRAD} keyword followed by $N(N+1)/2$ elements that constitute the upper
diagonal matrix (diagonal elements included) of the projection.
It is important to know that  \keyword{PROJ\_GRAD}  is {\underline NOT} supporting the  \keyword{LIST} keyword therefore all the variables
that one needs to use must be included in a "group" as reported in the example above. 

\section{Reconnaissance metadynamics}
\label{sec:recon_metad}

Reconnaissance metadynamics is a self-learning algorithm for accelerated dynamics that is able to work with a very large number of collective coordinates.  Acceleration of the dynamics is achieved by constructing a bias potential in terms of a patchwork of one-dimensional, locally valid collective coordinates.  To understand the details of the methodology please read reference \cite{tribello-recon}.  Furthermore, as detailed in section \ref{sec:recon_compile}, by default \plumed\ compiles without reconnaissance metadynamics and so some slight modifications in compilation are required if you wish to run this type of simulation.

For a reconnaissance metadynamics simulation the \plumed\ input must contain the definition of at least one CV (see chapter \ref{ch.generalcv} for the required syntax) and the \keyword{RECONNAISSANCE}, \keyword{BASINS}, \keyword{ONIONS} and \keyword{CLUSTER} keywords.  These keywords provide the parameters for reconnaissance metadynamics simulations and are described in sections \ref{sec.recon_cluster} and \ref{sec.recon_onions}. 

\subsection{Typical output}

A reconnaissance metadynamics simulation produces at least four output files.  These files are called {\tt BASINS}, {\tt ONIONS}, {\tt PPCA\_DIAGNOSTICS} and {\tt CLUSTER\_DATA.0}.  The first of these files contains the locations of the various basins found during cluster analysis and their shapes.  This data is formatted as follows:

\begin{itemize}
\item{The first column contains the basin number}
\item{The second column contains the timestep at which the basin was created}
\item{The following $d$ columns contain the values of the collective coordinates at the basin center}
\item{The next $d^2$ columns contain the covariance of the collective coordinates inside the basin}
\item{The final column contains the value of the ``diffusion constant" in this particular basin.}
\end{itemize} 

The second file, {\tt ONIONS}, gives the details of the bias.  In reconnaissance metadynamics the bias consists of Gaussian functions that are added to the distance from basin centers.  These Gaussian functions are only added when one is within a certain distance of the basin center, the basin size.  This basin size changes as the simulation progresses and the {\tt ONIONS} file stores the details of all basin expansion events as well as the details of the hill addition events.  Each event is recorded on a single line of the ONIONS file, which is formatted as follows:      

\begin{itemize}
\item{The first column contains the time step at which the contribution to the bias potential was added.}
\item{The second column contains the number of the basin to which this particular Gaussian was added}
\item{The third column contains the distance from the basin center specified in column 2 at which the centroid of the Gaussian is to be found}
\item{The fourth column contains the width of the Gaussian}
\item{The fifth column contains the height of the Gaussian}
\item{The final column contains the current size of the basin specified in column 2.}
\end{itemize}

Basin expansion events can be distinguished from hill addition events as in the former case the height of the ``Gaussian" in column 4 is zero.

The {\tt PPCA\_DIAGNOSTICS} file contains a information on how successful the PPCA algorithm, that is used for clustering, has been.  By default fits of the data are performed with 1 to 10 Gaussian functions.  Bayesian information criteria ( $-\log(L) + n_p \log(M)$, $\log(L)$, $n_p$ and $M$ being the log likelyhood, number of parameters and number of datapoints respectively) and fuzzy volumes ( $\sum_{n=1}^{N} \sqrt{ | \mathbf{\Sigma}_n | }$ ) are reported for each fit.  The best fit to the data is the one with the minimum value for the Bayesian Criterion and it is from this fit that the bias is constructed.  In our early applications of the methodology we have found that the if a large number of Gaussian is regularly needed to fit the collected data the method will not be successful.  However, a change of cvs can often resolve this problem.  After the {\tt PPCA\_DIAGNOSTICS} file has reported on the quality of the fits with different numbers of Gaussians the weights of the various Gaussians in the best fit to the data are reported along with information on which Gaussians have been accepted into the biassing strategy.

All the remaining files produced during a reconnaissance metadynamics simulation are only required to restart simulations.    
    
Unlike in normal metadynamics one cannot obtain the free energy surface from an examination of the bias potential at the end of the simulation.  However, a graphical tool that can be added to vmd is available from the \plumed\ website.  With this tool one can visualize the results from from reconnaissance metadynamics simulation.

\subsection{Controlling the clustering}
\label{sec.recon_cluster}

As discussed in reference \cite{tribello-recon} in reconnaissance metadynamics one periodically analyses the trajectory in order to obtain collective coordinates to bias.  The frequency of these analyses are controlled by the \keyword{CLUSTER} keyword.  The line containing this keyword should also contain an integer, which controls the frequency at which the values of the cvs are stored for subsequent cluster analysis and a integer than controls the frequency of clustering.  The first of these quantities should be proceeded by \keyword{STORE\_FREQ} and the second by \keyword{RUN\_FREQ}.  One may cluster over multiple timescales separately by including this line multiple times.

\esempio{
The following lines tell \plumed\ to run cluster analyses every 10000 steps and every 100000 steps    \vspace{10pt} \\
{\tt
RECONNAISSANCE \\
CLUSTER RUN\_FREQ 100000 STORE\_FREQ 100 \\
CLUSTER RUN\_FREQ 1000000 STORE\_FREQ 1000
}}
  
Once the clustering has finished \plumed\ must make certain decisions on how these basins should be incorporated into the biassing strategy.  The parameters controlling these decisions are given on the line starting with the \keyword{BASINS} keyword.  This line provides three parameters:

\begin{itemize}
\item{ \keyword{BASIN\_TOL} - the fraction of the trajectory that must be within a cluster in order for it to be used for biassing.}
\item{\keyword{INITIAL\_SIZE} - the initial size of the basin} 
\item{\keyword{EXPAND\_PARAM} - the parameter that controls the basin expansion} 
\end{itemize} 

Guidance on how to set the value of  \keyword{BASIN\_TOL} is provided in the supplementary information of reference \cite{tribello-recon}.  The supplementary information also shows how, when the angular variables of a multivariate Gaussian are integrated out, the resulting probability distribution as a function of $r$ (the distance from the center of the basin in the metric of the basin) is equal to a Gaussian with standard deviation $\frac{1}{\sqrt{2}}$ centered at $\sqrt{d-1}$.  With this in mind the initial size of basins is set equal to $\sqrt{d-1}$ + the value of the initial size parameter.  Therefore a sensible value for this parameter is around 1.5 - i.e. approximately two standard deviations.  

The final parameter described above \keyword{EXPAND\_PARAM} controls the way basins change size as a function of time.  As discussed in the supplementary information of reference \cite{tribello-recon} we introduce a probabilistic criterion for basin expansion that is derived using ideas from the theory of diffusion.  Since writing that paper we have slightly changed the way that this parameter functions.  In particular we realized that the diffusion constant in a given basin could be also be fit from the data using:

\begin{equation}
D = \frac{1}{ d \Delta t (M-1)} \sum_{i=2}^{M} | \mathbf{s}_i - \mathbf{s}_{i-1} |^2
\end{equation}
 
where $M$ is the number of vectors used for clustering, $d$ is the dimensionality, $\Delta t$ is the \keyword{STORE\_FREQ} parameter and the squared differences between vectors of CVs in the sum are calculated in the metric of the basin.  The \keyword{EXPAND\_PARAM} provided  in the input should be a number between 0 and 1.  It is used to scale this estimate value and expresses the level of confidence a user has in the fitting.  In practice this parameter gives some control over the rate of exploration of phase space.  If it is large phase space will be explored rapidly, while if it is small phase space will be explored more slowly.   
 
\subsection{Controlling the bias}
\label{sec.recon_onions}

The bias in reconnaissance metadynamics consists of a number of Gaussian functions that are periodically added to the potential.  These Gaussians are different from those added in normal metadynamics because they act added on the distance from the center of a basin.  Hence, the keyword that controls the hill addition in reconnaissance metadynamics is \keyword{ONIONS} and NOT \keyword{HILLS}.  However, the syntax for \keyword{ONIONS} is the same as the syntax for \keyword{HILLS}.  That is to say one must provide:

\begin{itemize}
\item{\keyword{HEIGHT} - The height of the Gaussian functions. }
\item{\keyword{W\_STRIDE} -  The frequency with which to attempt hill addition.}
\item{\keyword{WIDTH} - The width of the Gaussian functions. }
\end{itemize}

Once again guidelines as to how to set these parameters are provided in the supplementary information of reference \cite{tribello-recon}. 

Bringing all of the above together a typical input file for a reconnaissance metadynamics simulation should look like this:

\esempio{
Typical input for a reconnaissance metadynamics simulation    \vspace{10pt} \\
{\tt
RECONNAISSANCE \\
ONIONS HEIGHT 1.0 W\_STRIDE 1000 WIDTH 1.5 \\
BASINS BASIN\_TOL 0.2 EXPAND\_PARAM 0.3 INITIAL\_SIZE 3.0 \\
CLUSTER RUN\_FREQ 500000 STORE\_FREQ 250 \\
TORSION LIST 13 15 17 19 \\
TORSION LIST 15 17 19 21 \\
} 
}
 
\subsection{Restarting a simulation}

To restart a reconnaissance metadynamics simulation, the flag \keyword{RESTART} must be added on the line containing the directive \keyword{RECONNAISSANCE}.  This allows one to restart the simulation after an interruption or after a run has terminated.  At the start of the simulation the {\tt BASINS}, {\tt ONIONS} and {\tt CLUSTER\_DATA.*} files are read in so the bias can be restarted.  Data from the restarted simulation will then be appended to the {\tt BASINS} and {\tt ONIONS} files.  Please note that the \keyword{RESTART} flag is a directive to \plumed\ to restart the reconnaissance metadynamics simulation.  It not an instruction to restart the MD run.  Hence, to restart the simulation one must also follow the usual procedure for restarting the MD run.  This will depend on the particular MD engine you are using - details can be found in the documentation for the particular MD code you are using.        

\esempio{
A sample input file for restarting a reconnaissance metadynamics simulation    \vspace{10pt} \\
{\tt
RECONNAISSANCE RESTART \\
ONIONS HEIGHT 1.0 W\_STRIDE 1000 WIDTH 1.5 \\
BASINS BASIN\_TOL 0.2 EXPAND\_PARAM 0.3 INITIAL\_SIZE 3.0 \\
CLUSTER RUN\_FREQ 500000 STORE\_FREQ 250 \\
TORSION LIST 13 15 17 19 \\
TORSION LIST 15 17 19 21 \\
} 
}

\subsection{Using a subset of the defined cvs}

By default reconnaissance metadynamics is performed using all the collective coordinates defined in the \plumed\ input file.  However, there may be occasions where it is desirable to include some constraints to prevent exploration of the uninteresting parts of phase space.  In these cases it would also be desirable to not include these cvs in the reconnaissance metadynamics.  Consequentially, a keyword exists which allows one to explicitly specify which of the CVs are to be used in the reconnaissance metadynamics.  This keyword is \keyword{CV\_LIST} and it should be added to the line containing the directive \keyword{RECONNAISSANCE}.  It should be followed by a tag which gives the name of a list of the cvs to be used for reconnaissance.  This list should be specified using the syntax described in section \ref{sc.list}.

\esempio{
A sample input file for a reconnaissance metadynamics simulation using only three of the four collective coordinates specified in the \plumed\ input file    \vspace{10pt} \\
{\tt
RECONNAISSANCE CV\_LIST <cvlist> \\
ONIONS HEIGHT 1.0 W\_STRIDE 1000 WIDTH 1.5 \\
BASINS BASIN\_TOL 0.2 EXPAND\_PARAM 0.3 INITIAL\_SIZE 3.0 \\
CLUSTER RUN\_FREQ 500000 STORE\_FREQ 250 \\
TORSION LIST 13 15 17 19 \\
TORSION LIST 15 17 19 21 \\
TORSION LIST 17 19 21 23 \\
TORSION LIST 19 21 23 25 \\
cvlist->\\
1 2 3\\
cvlist<-
} 
}

\section{Driven Adiabatic Free Energy Dynamics (d-AFED)}

The driven adiabatic free energy (d-AFED) algorithm~\cite{Abrams08} is a variant of the earlier AFED method~\cite{Rosso02a,Rosso02}, which required cumbersome coordinates transformations. In the d-AFED method, an extra dynamical variable $S$ is coupled to a collective variable $s({\bf r})$, where $\bf{r}$ represents the coordinates of a number of atoms in the system. The coupling is mediated by a potential energy function with harmonic constant $\kappa$, 
\begin{equation}
V(S, s({\bf r})) = \frac{1}{2} \kappa \left( S - s({\bf r}) \right)^2.
\end{equation}
The dynamics of the $S$ meta-variable is adiabatically decoupled from the dynamics of the underlying physical system by choosing a large mass $m_S \gg \bar{m}$, where $\bar{m}$ is a typical mass of the physical system. Thanks to the adiabatic separation, a temperature $T_S > T$ can be assigned to the $S$ meta-variable. With this choice of $m_s$ and $T_s$,  the physical system will evolve fast at room temperature $T$ around the instantaneous value of $s({\bf r}) = S$ . On the other hand, $S$  will evolve slowly, but have a temperature large enough to drive the system over high free energy barriers.

 In the limit of $\kappa \rightarrow \infty$, it can be shown that the free energy surface at temperature $T$ can be recovered from the density $\rho^\mathrm{adb}(S)$ sampled at temperature $T_S$ during the adiabatic d-AFED simulation using 
 \begin{equation}
 \Delta G(S) = -k_B T_S \log\left( \rho^\mathrm{adb}(S) \right).
 \end{equation}
 This result generalizes well to the case where more than one collective variable is used and $\Delta G(S)$ is a multi-dimensional free energy surface. Note that the d-AFED method is similar to the Temperature Accelerated Molecular Dynamics (TAMD) method devised by other authors\cite{Maragliano06}.
 
 The d-AFED method requires very efficient thermostatting of the meta-variable $S$. Therefore, in the present implementation, $S$ is coupled to a Generalized Gaussian Moment Thermostat (GGMT) \cite{Liu00}. The meta-variable is coupled to two thermostatting variables $p_\eta$ and $p_\zeta$, with associated masses $Q_\eta$ and $Q_\zeta$, respectively. Given a typical time scale $\tau$ of the thermostated system, optimal masses are $Q_\eta=k_BT_S \tau^2$ and $Q_\zeta=\frac{8}{3}(k_BT_S)^3 \tau^2$. The order-2 GGMT dynamics for one degree of freedom is
\begin{eqnarray} \label{ggmt1}
\dot{p}_S &=& V(S, s({\bf r})) -  \frac{p_\eta}{Q_\eta} p_S  - \frac{p_\zeta}{Q_\zeta} \left[ k_B T_S + \frac{1}{3} \frac{p_S^2}{m_S} \right] p_S, \\\dot{p}_\eta &=&  \frac{p_S^2}{m_S} - k_B T_S,  \\
\dot{p}_\zeta &=&     \frac{1}{3} \left[ \frac{p_S^2}{m_S} \right]^2 - (k_B T_S)^2, \\
\dot{S} &=& \frac{p_S}{m_S},  \quad  \dot{\eta} = \frac{p_\eta}{Q_\eta}, \quad    \dot{\zeta} = \frac{p_\zeta}{Q_\zeta}  \label{ggmt2}.
\end {eqnarray}

If multiple reaction coordinates are used, one separate GGMT thermostat is associated to each of them. The implemented integrator for the dynamics above is based on a Trotter decomposition of the corresponding Liouville operator \cite{Abrams08}. The quality of the integration can be monitored using the quantity $H_S$, which would be conserved if the dynamics of $S$ was decoupled from the physical system,
 \begin{equation} \label{Hs}
 H_S(S,p_S,\eta,p_\eta, \zeta, p_\zeta) =  \frac{p_S^2} {2m_S} + V(S) +  \frac{p_\eta^2} {2Q_\eta}  
                                                                      +  \frac{p_\zeta^2} {2Q_\zeta} + k_B T_S \left( \eta + \zeta \right) 
\end{equation}
The heat transfer $W_S$ from the meta-variable to the physical system can be calculated,
\begin{equation}\label{Ws}
W_S = \int_0^t dt' \kappa  \left(S-s({\bf r}) \right) \frac{p_S}{m_S}
\end{equation}
 The effective adiabaticity of the coupling can thus be asserted. In addition, for each collective variable, the quantity $H_S - W_S$ should be strictly conserved and provides a quality check for the simulation. 

\subsection{Input for d-AFED}

For each CV, a \keyword{DAFED} directive is used to define the parameters of the corresponding dynamics.  On the same line, the number of the CV to which the directive applies is specified after the keyword \keyword{CV}. The temperature $T_S$ in K is given after keyword  \keyword{TEMPERATURE}. The  thermostat time constant $\tau$ is given in ps after keyword \keyword{TAUTHERMO}. The  mass $m_S$ and harmonic constant $\kappa$, are given after the keywords \keyword{MASS}  and \keyword{KAPPA}, respectively. The units of $\kappa$ and $m_S$ depend on the nature of the CV. They should always be such that $\kappa S^2$ and $m\dot{S}^2$ are both in units of energy (kJ/mol= amu nm$^2$ / ps$^2$), see the example below.  

In addition, tow optional keywords can be used with the  \keyword{DAFED} directive. First, for periodic CVs such as torsion angles, the $S$ variable should also evolve on a periodic interval. This is specified by the keyword  \keyword{PERIODIC}, followed by two numbers for the lower and upper bounds. The numbers can be replaced by  \keyword{MINUS\_PI},  \keyword{PLUS\_PI}, or \keyword{PLUS\_2PI} to specify $-\pi$, $+\pi$, or $+2\pi$, respectively. 

The optional keyword  \keyword{JACOBIAN\_FORCE} causes a bias force $F = - 2k_B T / S$ to be applied to the dynamics of $S$. This is useful with distance CVs in order to counterbalance the effect of the Jacobian factor and sample a more uniform distribution along the CV. 
\esempio{The following lines couple CV 1 (a distance in nm) to a meta-variable of mass $10^5$ amu with a harmonic constant of $10^6$ kJ/mol/nm$^2$ and CV 2 (a unitless number) to a meta-variable with mass $10^3$ amu*nm$^2$ with a harmonic constant of $10^4$ kJ/mol. For both CV, the d-AFED temperature is 600 K and the GGMT thermostat time constant is 0.2 ps. See text for the optional keywords \ \keyword{JACOBIAN\_FORCE} and  \keyword{PERIODIC}. \vspace{10pt} \\ 
{ \tt
DISTANCE LIST 1 34   \\
TORSION LIST 5 15 29 36 �\\ 
\\
DAFED CV 1 TEMPERATURE 600 MASS 1e5  KAPPA 1e6 TAUTHERMO 0.2 JACOBIAN\_FORCE\\
DAFED CV 2 TEMPERATURE 600 MASS 1e3  KAPPA 1e4 TAUTHERMO 0.2 PERIODIC MINUS\_PI PLUS\_PI \\
\\
DAFED\_CONTROL RESTART checkpoint\_file   WRITE\_STATE -1 N\_RESPA 1 \\ 
\\
PRINT W\_STRIDE 100 \\
ENDMETA\\}}

A separate \keyword{DAFED\_CONTROL} directive contains general controls for the d-AFED simulation. The d-AFED dynamics, including all variables described in Eqs.~(\ref{ggmt1} -  \ref{ggmt2}) can be restarted exactly from a previous run using a checkpoint file. Following the \keyword{WRITE\_STATE} keyword appears the number of steps after which a checkpoint file is saved. A value of $-1$ implies that a checkpoint file is written only when GROMACS saves its own checkpoint file, i.e. at regular wall clock time intervals. The checkpoint file is saved in the current directory with default name  \keyword{DAFED\_STATE}. The optional keyword  \keyword{RESTART} is used to specify the path to the checkpoint file from which to restart. 

\subsection{Typical output for d-AFED}

WIth the d-AFED method, the \keyword{COLVAR} file will contain the following data, if $d$ collective variables are used:

\begin{itemize}
\item time step
\item value of the collective variable $s_1({\bf r}) ... s_d({\bf r})$
\end{itemize} 

Then for each of the $S_j, \  j=1...d$, appears a set of  4 columns with :

\begin{itemize}
\item the meta-variable $S$
\item the instantaneous temperature of $S$ in K
\item the conserved quantity $H_S$, see Eq.~\ref{Hs}, in kJ/mol
\item the work $W_S$ from $S$ to the physical system, see Eq.~\ref{Ws}, in kJ/mol
\end{itemize}

These fields are labeled  \keyword{sj},  \keyword{T\_sj},  \keyword{E\_sj}, and  \keyword{W\_sj}, respectively, in the  \keyword{COLVAR} header line, $j=1...d$. Additional collective variables can be monitored during a d-AFED run, in which case more columns will appear before the first d-AFED keyword.

Note that for each extended variable $S_j$, the quantity $H_{S_j} + W_{S_j}$ should be conserved, $j=1...d$. Let $\mathcal{H}$ be the pseudo-energy of the physical system including the associated thermostats and barostats. Then the total energy of the simulation, $\mathcal{H} + \sum_{j=1}^d H_{S_j}$, should be conserved as well. In addition, considering the physical system only,  the quantity  $\mathcal{H} - \sum_{j=1}^d W_{S_j}$ should be conserved.


 %================================================================================   
\chapter{Collective variables}
\label{ch.generalcv}
\plumed \ contains implementations of a large number of CVs so one can properly describe 
the processes involved in a wide variety of interesting problems.  In the following chapter
we describe all the CVs, which are implemented in \plumed\ together with their analytic first derivative.
% The implementation of many CVs is required to deal
% with the different problems of interest and to give a proper description
% of the processes involved. In the following we describe all the possibilities 
% that the current package offers, implemented with their own analytic first derivative.

In general to instruct \plumed\ to use a particular CV a line starting with the keyword indicating
the CV type must be included in the input file. 
This keyword should then be followed by the various pieces of CV-specific information required to
calculate the variable along with the keywords that tell \plumed\ how this particular CV is to be employed.
(N.B. unless stated explicitly these pieces of data can be specified in any order)
%define the variable can be added in any order.  
\\
For metadynamics the lines defining the variables on which the user desires there to be a biasing potential 
should contain the \keyword{SIGMA} keyword.  This keyword should then be followed by the width of the 
Gaussian hills (in the units of the CV) on that particular CV.  This keyword serves two functions; namely, it 
instructs \plumed\ to use metadynamics and tells it the widths of the hills.  Obviously the \keyword{SIGMA} keyword
is not only required if you are running metadynamics and is not required with other methods.
% that defines the Gaussian width in CV units.
% This keyword is not needed when performing other types of free energy calculations.

\subsection*{Specifying lists of atoms}
\index{Atom lists} \label{sc.list}
Most collective variables require the user to specify one or several groups of atoms in their
definition. Whenever a set of atom groups is required, the \keyword{LIST}
keyword must be used. This keyword is then followed a set of 
tags which specify the number and order of the groups of atoms to be employed. The atoms involved in each
of the groups invoked must be specified somewhere in the input file.
These group specifications must then start with the the name of the group followed by the {\tt ->} sign and 
finish with the same name followed by the {\tt <-} sign.  Between these two delimiters 
the indices of the atoms which comprise the group must then be listed, separated by spaces or line feeds.
% be defined in the following lines as follows: for each group, a
% block starting with the name of the group followed by the {\tt ->} sign and
% terminating with the same name followed by the {\tt <-} sign is expected. 
% In the lines between the two delimiters, the atoms comprising the group must be
% indicated, separated by spaces or line feeds. 
\esempio{The following
syntax instructs \plumed\ to use the distance between the center of mass of
atoms 6 and 10 and the center of mass of atoms 8, 15 and 21 as a CV: \vspace{10pt} \\
{\tt DISTANCE LIST <g1> <g2> SIGMA 1.0  \\
 g1->    \\
 6 10    \\
 g1<-    \\
         \\
 g2->    \\
 8 15 21 \\
 g2<-    }
}

Inside the group definition
blocks, one can either specify the atom numbers explicitly, or one can use the {\tt LOOP} keyword
to define a regular sequence of indices with a given starting number, end
number and stride.
\esempio{The following two commands are equivalent definitions of the group {\tt g1}:
 \vspace{10pt} \\
{\tt 
g1-> \\
 10 12 14 16 18 20 \\
g1<- \\
 \\
g1-> \\
LOOP 10 20 2\\
g1<-  } 
}

For the special, and rather common, case of a group composed of a single atom
the user can specify the number of the atom of interest rather than the corresponding {\tt <g>} tag.
% can be specified by indicating the number of that atom instead of the
% corresponding {\tt <g>} tag.
\esempio{The following two commands are equivalent ways instruct \plumed\ to use 
the distance between atom 5 and group {\tt <g1>} as a CV:  \vspace{10pt} \\
{\tt DISTANCE LIST <g1> <g2> SIGMA 1.0\\
 g1->    \\ 
  10 12 14 16 18 20 \\
 g1<- \\
 \\
g2-> \\
5\\
g2<-\\
 \\
 DISTANCE LIST <g1> 5  SIGMA 1.0\\
 g1->    \\
  10 12 14 16 18 20 \\
  g1<-}
}

% -------------------------------------------------------------------------

\section{Absolute position} \cv{Absolute position}

The \keyword{POSITION} keyword instructs \plumed\ to use the absolute position of an atom
or a group of atoms, specified by using the LIST keyword, as a CV.
This CV accepts several options that allow the user to restrict the bias to a given direction, e.g. {\it z}, 
to bias the position of the particle as projected on a selected line segment or, in analogy with the path CV, to bias the atoms' distance from a line segment. 
The keyword DIR accepts as input X, Y or Z and limits the restraint to the chosen direction.
\esempio{The following line instructs \plumed\ to use the {\it y} coordinate of atom 13 as a CV.  \vspace{10pt} \\
{\tt POSITION LIST 13 SIGMA 0.35 DIR Y}}

The keyword LINE\_POS  instructs \plumed\ to use the projection of the atoms position on a line as a CV, while the keyword LINE\_DIST instructs \plumed\
to use the distance from the line as a CV.  In both these cases the line is defined by stating its start and end points.  The line can then 
either be in constrained to be in the XY, XZ or YZ planes (keywords (XY, XZ or YZ respectively) or it can have an arbitrary orientation in space (keyword XYZ).  
Obviously if the line is constrained to the XY, XZ or YZ planes then the start and end points are two dimensional vectors whereas if it has an arbitrary orientation these vectors have three components.

%The keywords LINE\_POS and LINE\_DIST define the CV to be the position of the atoms as projected 
%on a line and as the distance from the line respectively, the line being defined by its starting and final points.
%These keywords accept as input XY, XZ, YZ and XYZ and define whether the atom position 
%should be projected on a Cartesian plane (XY, XZ or YZ) or if the line has a generic orientation in space (XYZ).

\esempio{The following lines instruct \plumed\ to use the projection and distance of the coordinates of atom 13 
on a generic line segment defined by the start and end points (0,0,0) and (2,3,4) respectively 
as the two collective coordinates.  \vspace{10pt} \\
{\tt POSITION LIST 13 SIGMA 0.35 LINE\_POS XYZ 0. 0. 0. 2. 3. 4. \\ \\
POSITION LIST 13 SIGMA 0.35 LINE\_DIST XYZ 0. 0. 0. 2. 3. 4. \\
}}


% -------------------------------------------------------------------------
\section{Distance} \cv{Distance} 
The \keyword{DISTANCE} keyword instructs \plumed\ to use the 
distance between the center of mass of two groups of atoms as a CV.  Two groups
must be defined using the LIST keyword and the syntax described in
section \ref{sc.list}.

\esempio{The following lines instruct \plumed\ to use the distance 
between atom number 13 and the center of mass of the four atoms in list {\tt <g1>} as a CV.  \vspace{10pt} \\
{\tt DISTANCE LIST 13 <g1> SIGMA 0.35  \\
 g1->\\
 17 20 22 30\\
 g1<-}
}

The optional flag \keyword{NOPBC} can be used to calculate the distance without applying periodic
boundary conditions. This should be done only if all the atoms in the groups are part of the same molecule. See also
Sec.~\ref{PBC}.

The keyword \keyword{DIR} can be used to calculate the component of this 
distance along the cartesian axes (X, Y or Z) and or the component in the planes (XY, XZ or YZ).

\esempio{The following line instruct \plumed\ to use the X-component of the distance
between atom number 20 and 25 as a CV.  \vspace{10pt} \\
{\tt DISTANCE LIST 20 25 DIR X SIGMA 0.35
}}

Whenever one wants to use the difference between two distances one may use the keyword {DIFFDIST}.
This may turn to be useful in bond breaking/ bond formation. \cv{Difference of distances}

\esempio{The following line instruct \plumed\ to use the difference between the distance between 20 and 25 and 
the distance between 30 and 31 as CV.  
\vspace{10pt} \\
{\tt DISTANCE LIST 20 25 DIFFDIST 30 31
}}
 
Groups are also accepted as input instead of two atoms.

Other two useful variants are the following: the distance of a point from an axis and the projection of the point on the axis.
The first is introduced by the additional keyword  \keyword{POINT\_FROM\_AXIS} followed by the atom or the group defining the point
respect to which the distance has to be calculated. The axis is defined through the standard two groups appearing in the 
definition of the distance collective variable. \cv{Distance from an axis} 

\esempio{The following line instruct \plumed\ to use the distance between one atom, 26 and the axis defined by the two atoms 20 and 25 as CV. 
\vspace{10pt} \\
{\tt DISTANCE LIST 20 25 POINT\_FROM\_AXIS 26 
}}

In a similar way it is possible to calculate the projection of this point on an axis to be used as a CV by using the keyword 
\keyword{PROJ\_ON\_AXIS}. \cv{Projection on an axis}    

\esempio{The following line instruct \plumed\ to use the projection of the coordinate if atom 26 on  the axis defined by the two atoms 20 and 25 as CV. 
\vspace{10pt} \\
{\tt DISTANCE LIST 20 25 PROJ\_ON\_AXIS 26 
}}

Similarly to all the other variables, these two keywords may accept groups instead of atom indexes.

 

% -------------------------------------------------------------------------
\section{Minimum distance} \cv{Minimal distance} 
The \keyword{MINDIST} keyword instructs \plumed\ to use the minimum distance between two
groups of atoms as a CV.  To ensure differentiability, this quantity is implemented as: $$ s =
\frac{\beta}{\log \sum_{ij} \exp ( \beta/||r_{ij}|| ) },$$ where by default 
$\beta = 500$. The value of $\beta$ can however be tuned if needed by using the 
optional keyword \keyword{BETA}.
Much like the distance variable when calculating the minimum distance one must define
two groups using the LIST keyword and the syntax described in section
\ref{sc.list}.

\esempio{The following lines instruct \plumed\ to use the minimum distance 
between atom number 13 and the set of atoms in list {\tt <g1>} as a CV.  \vspace{10pt} \\
{\tt MINDIST LIST 13 <g1> SIGMA 0.35 BETA 500.\\
 g1->\\
 17 20 22 30\\
 g1<-
}}
  
The optional flag \keyword{NOPBC} can be used to calculate the distance without applying periodic
boundary conditions. This should be only be done if all the atoms in the groups are part of the same molecule. See also
Sec.~\ref{PBC}.

% -------------------------------------------------------------------------
\section{Angles} \cv{Angle} 
The \keyword{ANGLE} keyword instructs \plumed\ to use the angle defined by the centers of mass of
three groups of atoms as a CV.  The compulsory {\tt LIST} keyword must be followed by
three properly defined groups (see Section \ref{sc.list}).  

\esempio{The following lines instruct \plumed\ to use the angle 
defined by atom number 102 and the centers of mass of the atoms in groups {\tt g1} 
 and {\tt g2} as a CV.  \vspace{10pt} \\
{\tt ANGLE LIST <g1> <g2> 102 SIGMA 0.05  \\
g1->  \\
13 15  \\
g1<-  \\
 \\
g2-> \\ 
LOOP 1000 3000 3 \\ 
g2<- } }

It is also possible to use the sine or cosine of the angle as a collective coordinate by including the \keyword{SIN} or \keyword{COS} keywords respectively similarly to what is done for \keyword{TORSION}.

% -------------------------------------------------------------------------
\section{Torsion} \cv{Torsion}
The \keyword{TORSION} keyword instructs \plumed\ to use a dihedral angle as the CV. This angle can either be defined 
by four atoms or, more generally, by the positions of the centers of mass of four groups of atoms.
 The compulsory {\tt LIST} keyword must be followed by
four properly defined groups (see Section \ref{sc.list}).
\esempio{The following lines instruct \plumed\ to use to the torsion angle  
about the centers of mass of the four groups {\tt <g1>}, {\tt <g2>}, {\tt <g3>}, {\tt <g4>} as a CV.  \vspace{10pt} \\
{\tt TORSION LIST <g1>  <g2> <g3> <g4> SIGMA 0.35 }
}

It is also possible to use the sine or cosine of the torsional angle as a collective coordinate by including the \keyword{SIN} or \keyword{COS} keywords respectively.
\esempio{The following line instructs \plumed\ to use the cosine 
of the torsion angle about the centers of mass of the four 
groups {\tt <g1>}, {\tt <g2>}, {\tt <g3>}, {\tt <g4>} as a CV.  \vspace{10pt} \\
{\tt TORSION LIST <g1>  <g2> <g3> <g4> COS SIGMA 0.35 }
}

% -------------------------------------------------------------------------
\section{Coordination number} \cv{Coordination number} \label{coordnum}

The {\tt COORD} keyword instructs \plumed\ to use the total number of 
contacts between the atoms in group ${\cal G}_1$ and those in group ${\cal G}_2$ 
- the coordination number between these two groups.   To
ensure differentiability, this is implemented as the sum: $$s = \sum_{i\in {\cal
G}_1} \sum_{j\in {\cal G}_2} s_{ij},$$ where this sum is extended to all pairs of
atoms with $i\in {\cal G}_1$ and $j\in {\cal G}_2$.  The individual
contributions $s_{ij}$ are defined using a switching function, which, in the present
case, is given by:
$$ s_{ij} = \left \{
\begin{array}{ll}
 1 & {\rm for}~ r_{ij} \leq 0\\
 \frac{1-(\frac{r_{ij}}{r_0})^n}{1-(\frac{r_{ij}}{r_0})^m} & {\rm for}~ r_{ij} > 0\\
\end{array} \right. $$
where $r_{ij} = |r_i - r_j| - d_0$.  The user must supply the $r_0$, $d_0$, $n$
and $m$ parameters, using the additional keywords {\tt R\_0}, {\tt D\_0}, {\tt
NN} and {\tt MM} respectively and thus has a great deal of control over the definition
of the switching function.  In general a good first guess for these parameters can be 
achieved by looking at the pair distribution function and setting $d_0$ 
equal to the position of the first peak in the pair distribution function,
$r_0$ as the full width at half maximum of the peak and $n$ and $m$ to force
$s_{ij}\simeq0$ at the first minimum of the pair distribution function. However, 
oftentimes different choices for these parameters will lead to better results 
because of certain specific properties of the system of interest.
An optional keyword {\tt PAIR} treats the atoms in a pairwise fashion so that 
instead of simply counting the number of bonds between two groups of atoms one can define
which precise bonds between the two groups should be monitored.  In this case the groups, {\tt
<g1>} and {\tt <g2>} must have the same number of atoms as the switching functions are on the  
distances between the $i$th atom of group {\tt <g1>} and the $i$th atom of group {\tt <g2>}.

\esempio{The following lines instruct \plumed\ to use the coordination 
of the atoms in group {\tt g1} -- 13 and 15 -- 
with the atoms in group {\tt solvent} as the CV.  \vspace{10pt} \\
{\tt COORD LIST <g1> <solvent> NN 6 MM 12 D\_0 2.5 R\_0 0.5 SIGMA 0.35  \\ 
g1->  \\ 
13 15  \\ 
g1<-  \\ 
 \\ 
solvent-> \\ 
LOOP 1000 3000 3 \\ 
solvent<-} 
}

The optional flag \keyword{NOPBC} can be used to calculate the distance without applying periodic
boundary conditions. This should only be done when all the atoms are part of the same molecule. See also
Sec.~\ref{PBC}.

% -------------------------------------------------------------------------
\section{Hydrogen bonds} \cv{Hydrogen bonds}

\keyword{HBONDS} is the keyword for a variable that counts the number of intra-molecular
hydrogen bonds between a group of hydrogen bond donors and a group of hydrogen bond acceptors.  This is defined as: 
$$ s = \sum_{ij} \frac{1-(\frac{d_{ij}}{r_0})^n}{1-(\frac{d_{ij}}{r_0})^m},$$ 
where $i \in {\cal D}$ is the group of donors and $j\in {\cal A}$ are the acceptors.

The two groups must be defined using the compulsory {\tt LIST} keyword followed
by two groups (see Section \ref{sc.list}).  \plumed \ then assumes 
that  there is only one donor/acceptor per residue and, that within the list,
% each of these in the list of atoms that composes the group of donor/acceptor atoms are there is only one donor/acceptor per residue and that 
the donor/acceptor atoms on neighboring residues are consecutive.   
The values of $r_0$, $n$ and $m$ can be specified using the {\tt R\_0}, {\tt NN} and {\tt MM} keywords.
If no value is given for $r_0$, $n$ and $m$ the default values of $r_0=2.5$, $n=6$ and $m=12$ are assumed. 

The {\tt TYPE} keyword selects which residues to include in the count:
\begin{itemize}
\item With {\tt TYPE 0}, all donor-acceptor pairs are included;
\item If {\tt TYPE 1} is specified only those donor-acceptor pairs separated by an odd number of
residues greater than 4 are counted.  This allows one to monitor parallel $\beta$-sheet formations.
\item If {\tt TYPE 2} is specified only those donor-acceptor pairs that are separated by exactly 4 
residues are included.  This allows the formation of $\alpha$-helical conformations
($\alpha$ type) to be monitored; 
\item If {\tt TYPE 3} is specified,  only those donor-acceptor pairs that are separated by
an even number of residues greater than 4 are counted.
 This allows anti-parallel $\beta$-sheet formations ($\beta$-even type) to be monitored.
\item If {\tt TYPE 4} is specified,  only the first  donor and the first acceptor and so on are counted.
 This allows to monitor a set of native hydrogen bonds.
\item If {\tt TYPE 5} is specified, only hydrogen bonds between atoms belonging to different residues
are counted. Moreover, pairs with an index difference less than 5 are also discarded, so as 
to avoid counting H and a O that are in the same peptide group (NH-C=O) (contributed by M. Cuendet).
This option requires the user to specify to which residue each atom is belonging using
the RESLIST keyword (see example below).
\end{itemize}


\esempio{
The following lines instruct \plumed\ to use the count of the 
total number of hydrogen bonds between the pairs in groups {\tt H} and {\tt O} as a CV. 
The default switching function with  $r_0=2.5$, $n=6$ and $m=12$ is implied.  \vspace{10pt} \\
{\tt HBONDS LIST <H> <O> TYPE 0 SIGMA 0.1 \\
H->\\
 6 10  \\
H<- \\
 \\
O-> \\
8 12 \\
O<- }   \vspace{15pt} \\
In the following example, a modified switching function is employed.  \vspace{10pt} \\
{\tt HBONDS LIST <H> <O> TYPE 0 SIGMA 0.1 NN 8 MM 20 R\_0 2.5}  \\ \vspace{10pt} \\
In the following example, inter-residue hydrogen bond are counted \\ \vspace{10pt} \\
{\tt HBONDS LIST <H> <O> RESLIST <resH> <resO> TYPE 5\\
H->
1 9 17
H<- \\
O->
5 13 21
O<- \\
resH->
1 2 3
resH<- \\
resO->
1 2 3
resO<- \\
}
}

The optional flag \keyword{NOPBC} can be used to calculate the distance without applying periodic
boundary conditions. This should be done only when the atoms are part of the same molecule. See also
Sec.~\ref{PBC}.

% -------------------------------------------------------------------------
\section{Interfacial water} \cv{Interfacial water}

\keyword{WATERBRIDGE} is the keyword for a CV that counts the number of interfacial contacts.  
This variable does this by calculating the number of atoms from group ${\cal G}_0$ that are 
simultaneously in contact with atoms from both groups ${\cal G}_1$ and ${\cal G}_2$.  A typical
application of this CV is to count the number of water molecules at the interface of two surfaces.
This is calculated using:
$$ s_{\rm WatBr}=   \sum_{i}^{n0}  \left( \sum_{j}^{n1} 
 \frac{1-(\frac{\vert {\bf r}_i - {\bf r}_j \vert}{r_0})^n}{ 1-(\frac{\vert {\bf r}_i - {\bf r}_j \vert}{r_0})^m}  \right ) 
 \left( \sum_{j}^{n2} \frac{1-(\frac{\vert {\bf r}_i - {\bf r}_j \vert}{r_0})^n}{ 1-(\frac{\vert {\bf r}_i - {\bf r}_j \vert}{r_0})^m}  \right ). $$
The syntax of the command requires the user to specify three groups of atoms after the keyword {\tt LIST} starting with the ${\cal G}_1$ and  ${\cal G}_2$ groups and closing with the ${\cal G}_0$ group. 
(see Section \ref{sc.list}).
The parameters of the switching function are then defined using the usual {\tt NN}, {\tt MM} and {\tt R\_0} keywords. 

\esempio{ The following command instructs \plumed\ to use the number of atoms in the 
{\tt solvent} group that are simultaneously in contact with either atom 6 or 10 
and either atom 8, 15 or 21 as a CV. \vspace{10pt} \\
{\tt WATERBRIDGE LIST <type1> <type2> <solvent> NN 8 MM 12 R\_0 4.0 SIGMA  0.1 \\
type1-> \\
6 10   \\
type1<-\\
       \\
type2->\\
8 15 21 \\
type2<- \\
        \\
solvent->\\
LOOP 100 1000 3\\
solvent<-   
}}

% -------------------------------------------------------------------------
\section{Radius of gyration} \cv{Radius of Gyration} 

One can employ the radius of gyration of a group of atoms defined with the
compulsory additional keyword {\tt LIST} by using the \keyword{RGYR} directive.  
 The {\tt LIST} keyword (see Section \ref{sc.list}) must be followed by only one properly defined group.
This CV is calculated using :
$$ s_{\rm Gyr}=\Big ( \frac{\sum_i^{n}  
 \vert {r}_i -{r}_{\rm COM} \vert ^2 }{\sum_i^{n} m_i} \Big)^{1/2}, $$
where the sums are over the $n$ atoms in group ${\cal G}$ and the center of mass is defined using:
$${r}_{\rm COM}=\frac{\sum_i^{n} {r}_i\ m_i }{\sum_i^{n} m_i}.$$

\esempio{The following lines instruct \plumed\ to use the radius of gyration  
of the group {\tt <g1>} as a CV.  \vspace{10pt} \\
{\tt RGYR LIST <g1> SIGMA 0.35 }
}

N.B. The radius of gyration is calculated without applying periodic boundary conditions so
the atoms in group {\tt <g1>} should all be part of the same molecule. 
See also Sec.~\ref{PBC}.

\subsection{Gyration tensor based CVs}
\cv{Gyration tensor based CVs}

The \keyword{RGYR} directive can be used also to access to a number of CVs 
based on gyration tensor \cite{Jiris2011}:

\begin{equation}
S=\frac{1}{N}\left(\begin{array}{ccc}
\sum x_{i}^{2} & \sum x_{i}y_{i} & \sum x_{i}z_{i}\\
\sum x_{i}y_{i} & \sum y_{i}^{2} & \sum y_{i}z_{i}\\
\sum x_{i}z_{i} & \sum y_{i}z_{i} & \sum z_{i}^{2}\end{array}\right),
\end{equation}

which describes the spatial distribution of mass in a molecule or complex
of molecules. Alternatively, tensor of inertia may be used for the
same purpose:

\begin{equation}
\mathbf{I}=\left(\begin{array}{ccc}
\sum m_{i}(y_{i}^{2}+z_{i}^{2}) & \sum-m_{i}x_{i}y_{i} & \sum-m_{i}x_{i}z_{i}\\
\sum-m_{i}x_{i}y_{i} & \sum m_{i}(x_{i}^{2}+z_{i}^{2}) & \sum-m_{i}x_{i}z_{i}\\
\sum-m_{i}x_{i}z_{i} & \sum-m_{i}y_{i}z_{i} & \sum m_{i}(x_{i}^{2}+y_{i}^{2})\end{array}\right).
\end{equation}

%When the same weighting is used for atomic contribution, gyration
%and inertia tensors store the same information and share the same
%eigenvectors. Therefore only gyration tensor is evaluated. In both
%cases, coordinates are related to the center of object (geometric
%center or center of mass).

The weighting of atomic contribution is controlled by keyword \keyword{MASS-WEIGHTED}
(default) and \keyword{NO-WEIGHTED}.
When \keyword{MASS-WEIGHTED} directive is applied, the individual atomic contribution
to the gyration tensor is weighted by $\frac{m_{i}N}{\sum m_{i}}$
and the center of mass is used as origin of the system.
If \keyword{NO-WEIGHTED} option is chosen, coordinates are related to the geometrical
center of object and no mass-weighting is performed in calculation
of CV. 

Diagonalizing of gyration tensor provides the principal moments of gyration
- $S_{1}$,$S_{2}$,$S_{3}$ and three eigenvectors corresponding
to the principal axes of inertia. 
The individual gyration tensor based CVs are available by specifying the 
directive \keyword{RGYR} and the keyword \keyword{TYPE} followed by one of the following:
\begin{itemize}
\item \keyword{TRACE}: trace of inertia tensor

\item \keyword{GTPC\_1}: the largest eigenvalue of gyration
tensor. The square root $S_{1}^{'}=\sqrt{S_{1}}$ of principal moment
$S_{1}$ is used as CV. If we approximate the shape and mass distribution of the
object with an ellipsoid, $S_{1}^{'}$ corresponds to the largest radius
of this ellipsoid.

\item \keyword{GTPC\_2}: the middle principal moment of gyration tensor, $S_{2}^{'}=\sqrt{S_{2}}$.

\item \keyword{GTPC\_3}: the smallest principal moment of gyration tensor, $S_{3}^{'}=\sqrt{S_{3}}$.

\item \keyword{RGYR\_1}: the largest radius of gyration around principal axes
of inertia, $r_{g1}=\sqrt{S_{1}+S_{2}}$.

\item \keyword{RGYR\_2}: the middle radius of gyration around principal axes
of inertia, $r_{g2}=\sqrt{S_{1}+S_{3}}$.

\item \keyword{RGYR\_3}: the smallest radius of gyration around principal axes
of inertia, $r_{g3}=\sqrt{S_{2}+S_{3}}$.

\item \keyword{ASPHERICITY}: the deviation of mass distribution
from spherical symmetry. The modified version of asphericity was implemented
as a CV: $b'=\sqrt{b}$, where original Suter's \cite{doi:10.1021/ma00148a028}
asphericity is given by $b=S_{1}-\frac{1}{2}(S_{2}+S_{3})$.

\item \keyword{ACYLINDRICITY}: the deviation
from cylindrical symmetry. Again the modified version $c'=\sqrt{c}$
is used as a CV rather then original form \cite{doi:10.1021/ma00148a028}
$c=S_{2}-S_{3}$.

\item \keyword{KAPPA2}: the relative shape anisotropy\cite{doi:10.1021/ma00148a028}
$\kappa^{2}=1-3\frac{S_{1}S_{2}+S_{1}S_{3}+S_{2}S_{3}}{\left(S_{1}+S_{2}+S_{3}\right)^{2}}$
reflects both symmetry and dimensionality of an object. Limited between
values of 0 and 1, $\kappa^{2}$ reaches 1 for an ideal linear arrangement
and drops to zero in case of highly symmetric configurations (at least
tetrahedral symmetry).
\end{itemize}

\esempio{The following lines instruct \plumed\ to use the trace of the 
inertia tensor of the group {\tt <g1>} as a CV.  \vspace{10pt} \\
{\tt RGYR TYPE INERTIA LIST <g1> SIGMA 0.35 }
}
\esempio{The following lines instruct \plumed\ to use the asphericity 
of the group {\tt <g1>} as a CV.  \vspace{10pt} \\
{\tt RGYR TYPE ASPHERICITY LIST <g1> SIGMA 0.35 }
}

\section{Dipole} \cv{Dipole}
\keyword{DIPOLE} instructs \plumed\ to use the electrical dipole
generated by a group of atoms as a CV:
$$ s_{dipole}=\vert \sum_i^{n}  {\bf r}_i\ q_i \vert. $$
Only the {\tt LIST} keyword followed by one properly defined group is required 
to define this CV (see Section \ref{sc.list}).
\esempio{The following lines instruct \plumed \ to use the dipole  
of the group {\tt <g1>} as a CV. \vspace{10pt} \\
{\tt DIPOLE LIST <g1> SIGMA 0.35 }
}

% -------------------------------------------------------------------------
\section{Dihedral correlation} \cv{Dihedral correlation}

% come back here

\keyword{DIHCOR} is the keyword for a CV that measures 
the similarity between adjacent dihedral angles:
$$s_{\rm DC}=\sum_{i=2}^{N_D}\frac{1}{2}\left( 1 + \cos\left( \phi_i-\phi_{i-1}\right)\right).$$
The syntax for this CV requires the user to specify the number of dihedrals $N_D$ and, in the subsequent $N_D$ lines,
the indices of the four atoms defining each dihedral $\phi_i$.
\esempio{The following lines instruct \plumed\ to use the dihedral correlation for the 3 dihedrals listed as a CV. \vspace{10pt} \\
{\tt DIHCOR NDIH 3 SIGMA 0.1 \\
 168 170 172 188 \\
 170 172 188 190 \\
 172 188 190 197 }
}

% -------------------------------------------------------------------------
\section{Alpha-beta similarity} \cv{Alpha-beta similarity}
\keyword{ALPHABETA} is a keyword for a CV that measures the similarity of dihedral angles  
to a reference value (see also \ref{sc.tormds}). It is calculated using:
$$s_{\alpha\beta}=\frac{1}{2}{\sum_{i=1}^{N_D}\left( 1 + \cos\left( \phi_i-\phi_{i}^{\rm Ref}\right)\right)}.$$
The syntax for this CV requires the user to specify the number of dihedrals $N_D$ and, in the subsequent $N_D$ lines,
the indices of the four atoms defining each dihedral followed by the value of the dihedral in the 
reference conformation $\phi_i^{\rm Ref}$.
\esempio{The following lines instruct \plumed\ to use the
Alpha-beta similarity of three dihedrals as a CV. \vspace{10pt} \\
{\tt ALPHABETA NDIH 3 SIGMA 0.1 \\
168 170 172 188 3.14 \\
170 172 188 190 .56 \\
 188 190 192 230 3.14 }}

% -------------------------------------------------------------------------
\section{Alpharmsd} \cv{Alpharmsd}
\keyword{ALPHARMSD} is the keyword for a CV that counts the number of 6-residue segments in the protein chain 
that resemble an ideal alpha helix (i.e. the average experimental structure).
This CV is calculated using:
$$s_{\alpha\,{\rm rmsd}}=\sum_{\alpha}n\left[{\rm RMSD}\left(\left\{{\bf R}_{i}\right\}_{i\in\Omega_{\alpha}},\ \left\{ {\bf R}^{0}\right\} \right)\right],$$
where $n$ is the coordination (switching) function defined in Sec. \ref{coordnum}. 
The sum over $\alpha$ in the above runs over all the 6-residue segments in the protein chain, where
each of the residues is defined based on the positions of the backbone atoms N, CA, C, O and CB, $\Omega_{\alpha}$.   
The distance used in the switching function is then the root mean square difference between the distance matrix of the
atoms in the set $\Omega_{\alpha}$ with the corresponding distances between these atoms in the ideal alpha helix 
$\left\{{\bf R}^{0}\right\}$:
$${\rm RMSD}\left(\left\{{\bf R}_{i}\right\}_{i\in\Omega_{\alpha}},\ \left\{ {\bf R}^{0}\right\} \right)=
\sqrt{\frac{1}{N_{\rm pairs}}\sum_{i,j\in\Omega_{\alpha}}\left(d_{ij}-d_{ij}^0\right)^2}. $$
See Ref.\cite{pietrucci09jctc} for more details (although note that in \plumed\ the RMSD between distance matrices is used rather than the
cartesian RMSD. However, these two measures are essentially equivalent).
 
% The sum over $\alpha$ in the above runs over all the 6-residue segments in the protein chain, 
%$\Omega_{\alpha}$ is the set of backbone atoms N, CA, C, O, plus CB in one of these segments, 
%and $\left\{{\bf R}^{0}\right\}$ are the corresponding ideal atomic positions in a alpha helix.
%RMSD is defined as the root mean square difference of the distance matrix of atoms in the set $\Omega_{\alpha}$ 
%with respect to the distance matrix of the ideal positions:
%$${\rm RMSD}\left(\left\{{\bf R}_{i}\right\}_{i\in\Omega_{\alpha}},\ \left\{ {\bf R}^{0}\right\} \right)=
% \sqrt{\frac{1}{N_{\rm pairs}}\sum_{i,j\in\Omega_{\alpha}}\left(d_{ij}-d_{ij}^0\right)^2}. $$
%See Ref.\cite{pietrucci09jctc} for details (here the RMSD among distance matrices is used instead of the
%cartesian RMSD, but the two are basically equivalent).

To use this CV the syntax requires the user to specify the coordination function parameters 
{\tt R\_0}, {\tt D\_0}, {\tt NN} and {\tt MM} (see Sec. \ref{coordnum})
(we suggest values of {\tt R\_0}=0.8 Angstrom, {\tt NN}=8, {\tt MM}=12, {\tt D\_0}=0),
% N.B. In the relsease I think the rest of this sentence can be deleted as this will
% be hard coded
and a conversion factor {\tt ANGSTROM\_SCALE} which converts the ideal alpha positions
from Angstrom to the length units of the MD code 
(e.g. in Gromacs units are nanometers therefore {\tt ANGSTROM\_SCALE} is 0.1).
%No new paragraph here
In addition a list of list of atom indices for the N, CA, C, O and CB (in this order) for each residue 
must be provided for all the consecutive residues (in ascending order) which form the chain.  For those 
residues, such as glycine, which do not have a CB the atom index for the corresponding hydrogen should
be used.
% A list of atom indices must be provided representing the N, CA, C, O, CB atoms (in this order)
% of each residue, for all consecutive residues (in ascending order) which form the chain.
% If CB does not exist, like in glycine, the corresponding hydrogen must be used.

Important note: for those MD codes which, like Gromacs4, do not keep the protein whole but instead split it by the PBC, 
 {\tt ALPHARMSD} requires the additional option {\tt NOPBC}, together
with the {\tt ALIGN\_ATOMS} command for all the atoms employed in the {\tt ALPHARMSD}.

\esempio{The following lines define an Alpharmsd CV for all 16 consecutive residues of a protein in the MD code Gromacs4
(which uses nanometers as the length unit).
 \vspace{10pt} \\
{\tt ALPHARMSD LIST <ncacocb> SIGMA 0.5 R\_0 0.08 NN 8 MM 12 ANGSTROM\_SCALE 0.1 NOPBC \\
ncacocb-> \\
    1     5    23    24     7\\
   25    27    42    43    29\\
   44    54    56    57    51\\
   58    68    70    71    65\\
   72    74    77    78    76\\
   79    81   101   102    83\\
  103   105   116   117   107\\
  118   120   138   139   122\\
  140   142   162   163   144\\
  164   166   179   180   168\\
  181   183   190   191   185\\
  192   194   214   215   196\\
  216   218   225   226   220\\
  227   229   236   237   231\\
  238   240   243   244   242\\
  245   247   267   268   249\\
ncacocb-> }}

% -------------------------------------------------------------------------
\section{Antibetarmsd} \cv{Antibetarmsd}
\keyword{ANTIBETARMSD} counts the number of pairs of 3-residue segments in the protein chain 
which are similar to the ideal antiparallel beta (i.e. the average experimental structure).
See Ref.\cite{pietrucci09jctc} for details (but remember that here the RMSD among distance matrices is used instead of the
cartesian RMSD as the two are basically equivalent).
The definition, input parameters and syntax for this CV are the same as for the {\tt ALPHARMSD} and again we suggest values of {\tt R\_0}=0.8 Angstrom, {\tt NN}=8, {\tt MM}=12, {\tt D\_0}=0 for the parameters.
The only difference in the implementation is the additional option {\tt STRANDS\_CUTOFF} which allows the user to specify a threshold distance beyond which
pairs of 3-residue segments are considered far.  This option considerably speeds up the computation (often 1 nm is good choice for this quantity).

Important note: for those MD codes which, like Gromacs4, do not keep the protein whole but instead split it by the PBC,
 {\tt ANTIBETARMSD} requires the additional option {\tt NOPBC}, together
with the {\tt ALIGN\_ATOMS} command for all the atoms employed in the {\tt ANTIBETARMSD}.

% Important note: with some MD codes which do not keep the protein as a whole through the PBC, 
% like Gromacs4, {\tt ANTIBETARMSD} requires the additional option {\tt NOPBC}, together
% with the {\tt ALIGN\_ATOMS} command for all the atoms employed in {\tt ANTIBETARMSD}.

\esempio{The following lines define an Antibetarmsd CV for all 16 consecutive residues of a protein in the MD code Gromacs4
(which uses nanometers as the length unit).
 \vspace{10pt} \\
{\tt ANTIBETARMSD LIST <ncacocb> SIGMA 0.5 R\_0 0.08 NN 8 MM 12 ANGSTROM\_SCALE 0.1 STRANDS\_CUTOFF 1. NOPBC \\
ncacocb-> \\
    1     5    23    24     7\\
   25    27    42    43    29\\
   44    54    56    57    51\\
   58    68    70    71    65\\
   72    74    77    78    76\\
   79    81   101   102    83\\
  103   105   116   117   107\\
  118   120   138   139   122\\
  140   142   162   163   144\\
  164   166   179   180   168\\
  181   183   190   191   185\\
  192   194   214   215   196\\
  216   218   225   226   220\\
  227   229   236   237   231\\
  238   240   243   244   242\\
  245   247   267   268   249\\
ncacocb-> }}

% -------------------------------------------------------------------------
\section{Parabetarmsd} \cv{Parabetarmsd}
\keyword{PARABETARMSD} counts the number of pairs of 3-residue segments in the protein chain 
which are similar to the ideal parallel beta (i.e. the average experimental structure).
See Ref.\cite{pietrucci09jctc} for details (but remember that here the RMSD among distance matrices is used instead of the
cartesian RMSD as the two are basically equivalent).
The definition, input parameters, and syntax are the same as for the{\tt ALPHARMSD}
and again we suggest values of {\tt R\_0}=0.8 Angstrom, {\tt NN}=8, {\tt MM}=12, {\tt D\_0}=0 for the parameters.
The only difference in the implementation is the additional option {\tt STRANDS\_CUTOFF} which allows the user to specify a threshold distance beyond which
pairs of 3-residue segments are considered far. This option considerably speeds up the computation (often 1 nm is good choice for this quantity).

Important note: for those MD codes which, like Gromacs4, do not keep the protein whole but instead split it by the PBC,
 {\tt PARABETARMSD} requires the additional option {\tt NOPBC}, together
with the {\tt ALIGN\_ATOMS} command for all the atoms employed in the {\tt PARABETARMSD}.

% Important note: with some MD codes which do not keep the protein as a whole through the PBC, 
% like Gromacs4, {\tt PARABETARMSD} requires the additional option {\tt NOPBC}, together
% with the {\tt ALIGN\_ATOMS} command for all the atoms employed in {\tt PARABETARMSD}.

\esempio{The following lines define a Parabetarmsd CV for all 16 consecutive residues of a protein in the MD code Gromacs4
(which uses nanometers as the length unit).
 \vspace{10pt} \\
{\tt PARABETARMSD LIST <ncacocb> SIGMA 0.5 R\_0 0.08 NN 8 MM 12 ANGSTROM\_SCALE 0.1 STRANDS\_CUTOFF 1. NOPBC \\
ncacocb-> \\
    1     5    23    24     7\\
   25    27    42    43    29\\
   44    54    56    57    51\\
   58    68    70    71    65\\
   72    74    77    78    76\\
   79    81   101   102    83\\
  103   105   116   117   107\\
  118   120   138   139   122\\
  140   142   162   163   144\\
  164   166   179   180   168\\
  181   183   190   191   185\\
  192   194   214   215   196\\
  216   218   225   226   220\\
  227   229   236   237   231\\
  238   240   243   244   242\\
  245   247   267   268   249\\
ncacocb-> }}

% -------------------------------------------------------------------------
%\section{Torsional RMSD} \cv{Torsional Root mean square deviation}
%\label{sc.tormds}
%\keyword{RMSDTOR} is the keyword for a CV whose value is given by the root mean square deviation 
%of a set of dihedral angles from a reference configuration. This is calculated using:
%$$ s_{\rm TR}=\sqrt{ 
% \frac{ \sum_i^{N_D}( \theta_i - \theta_i^{\rm Ref} )^2 }{N_D}} .$$
%The syntax for this CV, like that for \keyword{ALPHABETA}, requires the user to specify the number of 
%dihedrals $N_D$ and, in the subsequent $N_D$ lines,
%the indices of the four atoms defining each dihedral followed by the value of the dihedral in the
%reference conformation $\theta_i^{\rm Ref}$.
%
%\esempio{The following lines instruct \plumed\ to use the  
%torsional root mean square deviation of two dihedrals as a CV. \vspace{10pt} \\
%{\tt RMSDTOR NDIH 2 SIGMA 0.1 \\
%168 170 172 188 0.4 \\
%178 180 182 188 1.4}
%}

% -------------------------------------------------------------------------
\section{Electrostatic potential} \cv{Electrostatic potential}
\label{sc.elstpot}
Using the \keyword{ELSTPOT} keyword one can instruct \plumed\ to use the electrostatic potential
exerted by a group of atoms on the center of mass of a second group of atoms (or single atom) as a CV. 
This CV is calculated using:
$$ s_{\rm ELST}=\sum_{i}^{N_A}\frac{q_i}{\vert {\bf r}_i-{\bf r}_{\rm COM}\vert}*f(\vert {\bf r}_i-{\bf r}_{\rm COM}\vert, R_0, CUT )
  $$
where 
$${\bf r}_{\rm COM}=\frac{\sum_i^{N_B} {\bf r}_i\ m_i }{\sum_i^{N_B} m_i}.$$
Here the sum in the first equation above is over the $N_A$ atoms in the group whose charges exert the electric potential, while
in the second equation the sum is over the $N_B$ atoms in the group that defines the point at which this potential is felt.
$f(x)$ is a smoothing function defined as:
%\begin{eqnarray}
%x < R_0 && f( x, R_0, CUT) = 1.0 \\ 
%R_0  < x < CUT  &&   f( x, R_0, CUT) =   cos\left( \frac{\pi x}{2 \left( { CUT - R_0 } \right)}  \right) \\ 
%\verb R_0 <x< \verb CUT &&     cos( \frac{\pi x}{2 (  \verb CUT  - R_0 ) } ) \\ 
%x>  CUT &&    f( x, R_0, CUT) = 0 
%\end{eqnarray}
\\
$$ f( x, R_0, CUT) = \left \{
\begin{array}{ll}
1.0 & x < R_0  \\
cos\left( \frac{\pi x}{2 \left( { CUT - R_0 } \right)}  \right) & R_0  \leq x \leq CUT  \\
0 & x>  CUT  \\
\end{array} 
\right . $$
\\
where $R_0$ is the onset and $CUT$ is a cutoff distance.

\esempio{The following lines instruct \plumed\ to use the electrostatic potential exerted by the atoms in
group2 on the center of mass of the atoms in group1 as a CV: \vspace{10pt}  \\
{\tt 
ELSTPOT LIST <group1> <group2>  R\_0 4.0 CUT 12.0 SIGMA 0.01 \\
group1-> \\
LOOP 1 8 1  \\
group1<- \\
group2-> \\
LOOP  9 16 1  \\ 
group2<- 
}
}

% -------------------------------------------------------------------------
\section{Puckering coordinates} \cv{Puckering coordinates}
\label{sc.puckering}
\keyword{PUCKERING} refers to the set of collective coordinates for 6-membered rings in polar coordinates~\cite{Puckering}.
Given the coordinates $z_j$, that  represent the displacements of the $j-$th atom from the mean ring plane,
three variables $Q,\theta,\phi$ can be obtained starting from the general definition for 6-membered rings

$$
q_2 \cos \phi_2 =  \sqrt{\frac{1}{3}} \sum_{j=1}^6 z_j \cos\left[\frac{2\pi}{3} (j-1)\right]
$$
$$
q_2 \sin \phi_2 = -\sqrt{\frac{1}{3}} \sum_{j=1}^6 z_j \sin\left[\frac{2\pi}{3} (j-1)\right] 
$$
$$
q_{3} = \sqrt{\frac{1}{6}} \sum_{j=1}^6 (-1)^{j-1} z_j 
$$
as
$$
Q = \sqrt{q_2^2+q_3^2} \quad \geq 0 
$$
$$
\theta = \arctan\left({q_2/q_3}\right) \quad \in [0,\pi]
$$
$$
\phi= \phi_2 \quad \in [0,2\pi)
$$

Each one can be selected as a TYPE of PUCKERING. The CV accept the \verb!LIST! and \verb!SIGMA! keywords. 
As of now \verb!LIST! accept only 6 atoms. The atoms in the list have to be enumerated following the chemical sequence (although the
numbering scheme in the topology  does not have to be sequential). In order to fulfill IUPAC convention
for sugar hexopyranose rings the first atom in the list has to be the ring oxygen, followed by the
anomeric carbon.
$Q$ is in general a fast degree of freedom.

\esempio{The following lines define a PUCKERING CV: \vspace{10pt} \\
{\tt 
PUCKERING LIST <group1> TYPE PHI SIGMA 0.1 \\
group1-> \\
1 2 3 4 5 6  \\
group1<-
}
}

% -------------------------------------------------------------------------
\section{Path collective variables} \cv{Path variables}

One can instruct \plumed\ to use path collective variables \cite{brand07} 
using the \keyword{S\_PATH} and \keyword{Z\_PATH} keywords.  In this scheme
one defines a path as a set of $N$ reference conformations that
define the path in configuration space ${\cal X}$ from some initial state to some final 
state.  The $s$ variable (defined with {\tt S\_PATH}) then
measures the position {\it along} the path, and is defined as:
$$ s = {\cal Z}^{-1} \sum_{i=1}^{N} i e^{-\lambda d(X_i,X(t))}, $$
where $X(t)$ is the configuration of the system at any given time, 
$d:{\cal X}\times {\cal X} \rightarrow {\mathbf R}^+_0$ is a metric on ${\cal X}$, and
$ {\cal Z} =  \sum_{i=1}^{N} e^{-\lambda d(X_i,X(t))}$ is a normalization factor in which
the prefactor, $\lambda$, should be chosen so as to have 
$ \lambda d(X_i,X_{i\pm 1}) \simeq 2.3 $ on average.

%The $s$ variable (defined with {\tt S\_PATH}) 
%measures the position {\it along} the path, and is defined as:
%$$ s = {\cal Z}^{-1} \sum_{i=1}^{N} i e^{-\lambda d(X_i,X(t))}, $$
%where $X(t)$ is the configuration of the system at any given time,
%$X_i$ is an ordered set of $N$ reference conformations that 
%define the path in the configuration space ${\cal X}$, 
%$d:{\cal X}\times {\cal X} \rightarrow {\mathbf R}^+_0$ is a metric on ${\cal X}$, and 
%$ {\cal Z} =  \sum_{i=1}^{N}   e^{-\lambda d(X_i,X(t))}$ 
%is a normalization factor.
%The prefactor $\lambda$ should be chosen so as to have on average
%$ \lambda d(X_i,X_{i\pm 1}) \simeq 2.3 $.
  
By contrast the $z$ variable (defined with {\tt Z\_PATH}) measures the position
{\it off} the path, and is defined as:
$$ z = - \lambda^{-1} \log {\cal Z}. $$

For both \keyword{S\_PATH} and \keyword{Z\_PATH} the following keywords must be used:
\begin{itemize}
\item {\tt TYPE}, which defines the metric used to calculate distances 
 in configuration space.  The following sections provide more details
 on this but, suffice to say, currently one can use either {\tt MSD} 
 (mean square deviation see section \ref{ssc.msd}), 
 {\tt DMSD} (distance root mean squared deviation see section \ref{ssc.dmsd}) 
 or {\tt CMAP} (the distance between contact matrices see section \ref{ssc.cmap}) \cite{Bo.Bra:08}. 
\item {\tt NFRAMES} which sets the  number of reference structures, $N$,
that are used in the definition of the path.
\item {\tt LAMBDA}, the prefactor $\lambda$ in the exponential 
  term of the equation that defines both $s$ and $z$. 
\item Optionally, you can also use {\tt NEIGHLIST} which defines a neighbor list on the closest 
      frames to speed up the calculation followed by the number of steps in which the list must be calculated
and the number of elements it must contain (e.g. {\tt NEIGHLIST 50 10 } means that each 50 steps the neighbour list must be calculated 
and it will contain the 10 closest elements to the current molecular dynamic snapshots. All the other are discarded up to the next neighbor list calculation.).
\end{itemize}
 
Other keywords are specific to the type of path variable being defined
(\emph{i.e.} mean square displacement in Cartesian coordinates, 
MSD in distances or contact map distance). 

\underline{An important change occurred from the version 1.2 to 1.3} : the convention for the names has changed the former \keyword{RMSD} has been replaced by \keyword{MSD}. Similarly \keyword{DRMS} has been changed into  \keyword{DMSD}. This was done since many users noticed the inconsistencybetween the definitions and the related keywords. 
 \\ 

N.B. before using this feature please ensure that the parameters in {\tt metadyn.h} (common files directory) are 
set properly. In particular look for the section containing: 
\\
{\tt
// path dimensions\\
\#define MAXATOMS\_PATH 230 \\
\#define NMAX\_PATH 8 \\
\#define MAXFRAMES\_PATH 22  \\
\#define MAXATOMS\_RMSD 230 \\
\#define MAXCHARS\_PATH 40 \\
// cmap \\
\#define MAXDIM\_CMAP 3800 \\
\#define MAXNUM\_GROUP 10 \\
\#define MAXATOM\_GROUP 30 \\
}

and ensure that {\tt MAXATOMS\_PATH} is greater than or equal to the number of atoms per frame involved in your 
path and {\tt MAXFRAMES\_PATH} is greater than or equal to the number of frames you are using to define your path.
In addition {\tt NMAX\_PATH } maximum number of path variables that you can use simultaneously.  If you change any
of these values you must subsequently recompile all the MD codes in which you have implemented \plumed\ in order for your 
changes to take effect.  

% {\tt NMAX\_PATH } is the maximum number of path variables simulaneously used, {\tt MAXFRAMES\_PATH} is the 
% maximum number of frames in your path. After having properly set this numbers then you need to compile once again 
% all the programs in which you make use of them.  

\subsection{Mean square deviation}
\label{ssc.msd}
 As already briefly mentioned, when using the path collective coordinates \keyword{S\_PATH} and \keyword{Z\_PATH},
 the command {\tt MSD} after the {\tt TYPE} keyword instructs \plumed\ to the the mean square deviation of a 
 subset of the atoms in the system (the {\it displacement} set {\cal B}) calculated after the system has been aligned 
 to another subset of the atoms (the {\it alignment} set {\cal A}) as the metric used in the definition of the path.  
 This quantity is calculated using 
 $$ d(X_j,X_i) = \sum_{a=1}^{N_{\cal B}} 
  \frac{w_{j}}{w_{TOT}}(X^{(j)}_a - M_{ij}(\{z\}) X^{(i)}_a)^2,$$
 where  $M_{ij}(\{z\})$ is the roto-translation matrix calculated using the Kearsley \cite{kearsley} algorithm and $w_{j}$ is 
the number specified (called the {\it displacement} parameter) for the beta column in the provided input PDB.
If this is one then $w_j$=1 and $w_{TOT}$=$N$ (while differently $w_{TOT}=\sum_i^N w_i$).
The weights for the  alignment  are actually taken into account when forming the alignment matrix that compose the 
$M_{ij}(\{z\})$ and are denoted here with the set $\{z\}$ defined by the beta column in the reference PDB.  

 To use path CVs the user must specify the coordinates of the atoms in each of the reference frames 
 of the path in a set of supplementary input files.  The user should specify the basename of these 
 files (<basename>) after the keyword {\tt FRAMESET} and be aware that \plumed\ then will expect to find $N$ files named
 <basename>.$i$.pdb, which contain the coordinates of the atoms in both the {\it displacement} and {\it alignment} sets
 for the $i$th frame in the path.  The particular set/s each atom is involved in is specified using the the last two 
 numerical fields of the frameset file.  Values of 1.0 and 0.0 indicate that the atom is to be used in the 
 {\it alignment} set only, while values of 0.0 and 1.0 indicates that the atom is to be used in the {\it displacement} set
 only.  Values of 1.0 and 1.0 indicate that the atom is to be used in both the {\it alignment} and {\it displacement} sets.
% containing the coordinates 
%  of the reference frames in the PDB format after the keyword {\tt FRAMESET}. $N$ files, 
%  whose name is obtained by appending to the basename a consecutive integer  and  the
%  extension {\tt .pdb},  are expected.
%\\
%The atoms involved in the  {\it alignment} and in the {\it displacement} sets must be specified using 
%the last two numerical fields of the frameset files respectively ($1.00$ for used, $0.00$ for not used).
\\
Version 1.1 and higher allow these alignment and displacement indicators to be non-integer.
This allows users to perform a weighted alignment in cases where the alignment of one region of system is considered more important 
%(in case you need to align better one portion than another)
and a weighted displacement evaluation in cases where the displacement of a particular atom/s is though to be of particular importance. 
%(in case you need to increase the weight of the displacement fo one atom). 
Be aware however that this feature can produce strange numbers that are not trivial to interpret and is thus perhaps best left to experienced users. 
%is for experienced users as it can produce strange numbers that are not trivial to be interpreted.   
\\
N.B. \plumed \ contains a hard coded limit on the number of atoms that can be used in the alignment so, like in the previous section, users should
ensure that the value of this hard limit ({\tt MAXATOMS\_RMSD} in {\tt metadyn.h}) is sufficiently large for their needs prior to compilation.   
% Note that the number of atoms you may use in the alignment depends strictly on the {\tt MAXATOMS\_RMSD} keyword   
% in {\tt metadyn.h}. You should adjust that so that the maximum number of atoms in the alignment suits your needs 
% and recompile the package. In case of wrong choice a warning should appear anyway.
\\
The unit of distance in the PDB files is {\AA}ngstrom. Engines like GROMACS, whose internal units are different, will perform appropriate conversions automatically.  
\\
Clearly, the atom indices in these PDB files must be the same as the indices of the atoms they refer to in the system topology.  It is therefore likely that the atom indices in the frameset files may be non-consecutive.   
\\
Additional keywords are supported: \keyword{NO\_ROT} and  \keyword{NO\_CENTER}. These two keywords prevent the rotation and the center of 
mass alignment respectively whenever one uses {\tt MSD}.  
\\
 \esempio{The following command defines a path 
 using MSD metrics. 2 frames are used to 
 define the path, which has $\lambda=9.0$.  \vspace{10pt}

{\tt
S\_PATH TYPE MSD FRAMESET frame\_ NFRAMES 2 LAMBDA 9.0 SIGMA 0.1 \\
Z\_PATH TYPE MSD FRAMESET frame\_ NFRAMES 2 LAMBDA 9.0 SIGMA 0.1 \\
}

Two PDB files must be provided: frame\_1.pdb and frame\_2.pdb.
The last two columns of these files specify which atoms are to be
used for alignment and which are to be used to calculate the MSD.   \vspace{10pt}


{\tt
ATOM       1  C   ALA     2      -0.186  -1.490  -0.181  1.00  0.00 \\
ATOM       2  O   ALA     2      -0.926  -2.447  -0.497  1.00  1.00 \\
ATOM     15  N   ALA     2       0.756   0.780  -0.955  1.00  0.00 \\
ATOM     17  CA  ALA     2       0.634  -0.653  -1.283  1.00  1.00 \\
ATOM     19  CB  ALA     2       2.063  -1.233  -1.286  1.00  1.00\\
END }

}


\subsection{Distance mean square deviation}
\label{ssc.dmsd}

 Instead of using the MSD in the definition of the metric for the path 
 in the \keyword{S\_PATH} and \keyword{Z\_PATH} CVs the command {\tt DMSD} after the {\tt TYPE} keyword  
 allows one to use a metric based on the
% The word {\tt DRMS} after the {\tt TYPE} keyword sets the metric for the path
    mean square deviation of the distances between a subset of the atoms in the system:  
 $$
 d(X_j,X_i) = \frac{2}{N_{\cal A}(N_{\cal A}-1)}\sum_{a=1}^{N_{\cal A}-1}  \sum_{b=a+1}^{N_{\cal A}}  
  (r^{(j)}_{ab} - r^{(i)}_{ab})^2,$$
 where $r^{(j)}_{ab}$ is the distance of atoms $a$ and $b$ in the 
 $j$-th reference frame.
 For this metric the coordinates of the atoms in the reference frames of the path
 are specified using the keyword {\tt FRAMESET} along with a set of pdb files
 containing the atom coordinates. This is the same way that they are specified when the root mean 
 square deviation metric is used (see section \ref{ssc.msd}). 

%  As is the case for the root mean square deviation metric (see section \ref{ssc.msd}),
%  the frameset must be specified using the 
%  keyword {\tt FRAMESET}, and the PDB format must be used to define the 
%  reference structures.
 

\subsection{Contact map distances}
\label{ssc.cmap}
 The final choice one can employ to define the metric for the path in the \keyword{S\_PATH} and \keyword{Z\_PATH} CVs 
 is to use the command {\tt CMAP} after the {\tt TYPE} keyword.  This sets the metric 
 for the path to be the distance between the contact matrices for a given subset of the atoms in the system:
 $$ d(X_j,X_i) = ||D^{(j)}_{ab} - D^{(i)}_{ab}||.$$
 Given two sets of atoms $a,b \in {\cal J}$, this contact matrix $D_{ab}$ 
 is calculated using:
 $$ D_{ab}(X) = \theta(c_{ab} - r_{ab}) w_{ab} \frac
{\Big ( 1 - (r_{ab}/r_{ab}^{(0)})^n_{ab} \Big ) }
{\Big ( 1 - (r_{ab}/r_{ab}^{(0)})^m_{ab} \Big ) }, $$
where $\theta(x)$ is a step function which vanishes if $x<0$.

 As with other variables, the parameters $r_{ab}^{(0)}$, $n_{ab}$ and $m_{ab}$
 allow for a great freedom in the definition of the switching function.
 What is more the parameter $c_{ab}$ allows one to set the values of the switching function to
 zero at large separations.  Finally, if the formation of particular contacts is deemed to be of great importance   
 the weights $w_{ab}$ can be used to change their relative importance.

 Like the other metrics for the path collective variables to use path collective variables with a contact map metric
 information must be provided in supplementary files about the frames that make up the path.  
 The syntax requires the user to specify a file name for the indices of the atoms and the parameters defining 
 the calculation of the contact matrix (after the keyword  {\tt INDEX}) and another filename for the values of the 
 reference matrices $D_{ab}^{(i)}$, after the keyword {\tt MAP}.  
 \\
 The {\it index} file, specified after the  {\tt INDEX} keyword, must contain one line for each of the elements in the 
 contact matrix $D_{ab}$, which specified how that particular contact should be calculated.  
   Each of these lines should begin with the {\tt CONTACT} keyword
 followed by a numerical label for the contact.  The next two fields are the indices $a,b$  of
 the two atoms that make up the contact, which are followed by the values of 
 $r_{ab}^{(0)}$, $n_{ab}$, $m_{ab}$, $c_{ab}$ and $w_{ab}$ in the switching function.
% the values of the pairs of indices $a,b$ that define each contact, and the
%corresponding values of $r_{ab}^{(0)}$, $n_{ab}$, $m_{ab}$, $c_{ab}$ 
%and $w_{ab}$.  
% Each contact is on a separate line, starting with the {\tt CONTACT} keyword.
\\
The {\it values} file, specified  after the keyword {\tt MAP},
contains the values of the reference matrices $D_{ab}^{(i)}$ used in the definition of the path. 
Each line in this file corresponds to one element in the contact matrix. These lines are formatted 
with the numerical label for the contact as the first field.  This is followed by the indices of the two atoms
that make up the contact $a,b$ and finally the value of this particular switching function in 
the reference frame.  Unlike the MSD and DMSD metrics all the reference frames are placed in a single file
with each reference frame separated by the {\tt END} keyword.

The optional flag \keyword{NOPBC} can be used to calculate the distance without applying periodic
boundary conditions. This should be done only if all the atoms in the groups are part of the same molecule. See also
Sec.~\ref{PBC}.

% Each reference matrix must be separated by the {\tt END} keyword, and consists of lines
% with the contact number, the values $a$ and $b$ of the involved atoms, 
% and the value of the contact map.

 \esempio{The following command defines a path 
 using the contact map metric. 2 frames are used to 
 define the path with $\lambda$ set equal to 0.1. 
 \vspace{10pt} \\
 {\tt S\_PATH TYPE CMAP NFRAMES 2 INDEX fr.ndx MAP fr.mps LAMBDA 0.1 SIGMA 1. NOPBC}

 \vspace{5pt}
 The {\tt  fr.ndx} file should contain the details on how each of the switching functions 
 in the contact matrix should calculated: \vspace{10pt} \\ 
% indices of the atoms that 
% define the contact matrix:\vspace{10pt} \\
 {\tt 
CONTACT    1    1    2   3.0  6 10 100.0    1 \\
CONTACT    2    1   15   3.0  6 10 100.0    1 \\
CONTACT    3   17    2   3.0  6 10 100.0    1 \\
CONTACT    4   15   19   3.0  6 10 100.0    1} \\

The {\tt  fr.mps} file contains the values of the reference matrices:  \vspace{10pt} \\ 
{\tt 
   1    1    2      0.99491645\\
   2    1   15      0.76586085\\
   3   17    2      0.79183088\\
   4   15   19      0.81924184\\
END\\
   1    1    2      0.99369661\\
   2    1   15      0.76748693\\
   3   17    2      0.76454272\\
   4   15   19      0.72917217\\
END }
 }

One can also use contact matrices that involve
contacts between the centers of mass of groups of atoms.
However, in this case, a further additional file containing the definition of the groups must
be provided.  The name of this file is specified in the \plumed \  input file, using the {\tt GROUP} keyword.
%The file itself must contain the definitions of the groups, as follows.
Then within the specified file each group is defined on a single line starting with the {\tt GROUP} keyword.
This keyword is followed by a numerical label for the group, the number of atoms in the group and then a list
of the indices of the various atoms that make up the group.  To instruct \plumed \ to use these group contacts 
rather than atomic contacts one must use lines starting with the keyword {\tt GROUP} in the index file.  The 
remainder of the format of the lines in the index file and the format of the corresponding lines in the 
map file are identical to the lines used to specify atomic contacts.  However, the indices that would have 
specified the indices of the atoms involved in the atomic contact must be replaced with the indices from the
group file of the two groups that make up the contact.
% the first number labels the group,
% the second number specifies how many atoms  are in the group, and  the 
% following numbers are the indices of the atoms forming the group under consideration.

\esempio{
The following command defines a path
 using the contact matrix metrics. 2 frames are used to
 define the path, and a value of $\lambda=0.1$ is set.
 \vspace{10pt} \\
 {\tt S\_PATH TYPE CMAP NFRAMES 2 INDEX fr.ndx \
   MAP fr.mps GROUP fr.grp LAMBDA 0.1 SIGMA 1.}  \vspace{10pt} \\
  The file {\tt fr.grp} defines three groups of atoms:   \vspace{10pt} \\
 {\tt  GROUP 1 4 23 43 56 457 \\
 GROUP 2 5 76 47 97 322 695 \\
 GROUP 3 4 17 15 19 2}  \vspace{10pt} \\
 The {\tt  fr.ndx} file contains the parameters of the contacts between groups:  \vspace{10pt} \\
 {\tt
GROUP    1    1    2   3.0  6 10 100.0    1 \\
GROUP    2    1    3   3.0  6 10 100.0    1 \\
GROUP    3    2    3   3.0  6 10 100.0    1 } \vspace{10pt} \\
Finally, the {\tt fr.mps} has the values of the  contact matrix in the reference positions: \vspace{10pt} \\
{\tt
   1    1    2      0.9949\\
   2    1    3      0.7658\\
   3    2    3      0.7918\\
END\\
   1    1    2      0.9936\\
   2    1    3      0.6674\\
   3    2    3      0.8645\\
END }
}

It is possible to have maps in which there are both atomic and group contacts. However,
be aware that, in the index files in which the switching functions are specified, 
the definitions of the group contacts MUST follow the definitions of the atomic {\tt CONTACT} functions.
\esempio{An example of an index file containing both atom-atom contacts and
 group contacts:  \vspace{10pt} \\
{ \tt
CONTACT  1    123  545   7.0  6 10  100.0  0.50 \\
CONTACT  2    224  244   8.5  6 10  100.0  0.50 \\
GROUP    3      1    2   3.0  6 10  100.0  1.00 
}
}


\subsection{Using path variables as MSD, DMSD and CMAP and the {\tt TARGETED} statement}
\cv{MSD} \cv{DMSD} \cv{Contact Matrix distance} \cv{Targeted}

If one defines a $z$ path collective variable with a single frame it is clear from the
definition $$ z = - \lambda^{-1} \log {\cal Z} = - \lambda^{-1} \log \sum_f e^{-\lambda d_f} $$
that this is equivalent to using to the squared distance of the current
configuration from a reference structure in the chosen metric.  This CV can thus be
used in simulations which employ the standard MSD, distance DMSD or CMAP distance from a single frame as
a collective coordinate.  However,  in this case we recommend use of the alias \keyword{TARGETED} instead, which
defines the exact same CV but with a far simpler syntax.   

% The definition of the $z$ path variable, 
% $$ z = - \lambda^{-1} \log {\cal Z} = - \lambda^{-1} \log \sum_f e^{-\lambda d_f} $$
% makes it clear that, for $f=1$, it coincides with the  distance between a configuration
% and the reference structure, measured in the chosen metric (squared). Therefore,
% this CV should be used to reproduce the standard RMSD, distance RMSD or CMAP distance.
% There is also an useful alias for doing 
% \keyword{Z\_PATH} with only one frame wich is \keyword{TARGETED} .
% In this way you may do umbrella sampling, metadynamics or steered md by taking only one reference frame
% with simplified and user-friendly syntax.

\esempio{The following command instructs \plumed \ to do steered MD towards a target frame 
 using MSD metrics. 
 \vspace{10pt}

{\tt
PRINT W\_STRIDE 10\\
TARGETED TYPE MSD FRAMESET ref\_frame.pdb  \\
STEER CV 1 TO 3.0 VEL 0.5 KAPPA 500.0\\
ENDMETA\\
}
 \vspace{10pt}

A single PDB file must be provided: ref\_frame.pdb, in which the 
last two columns specify which of the atoms are to be
used for alignment and which are to be used to calculate the MSD/DMSD (see section \ref{ssc.msd}).   \vspace{10pt}


{\tt
ATOM       1  C   ALA     2      -0.186  -1.490  -0.181  1.00  0.00 \\
ATOM       2  O   ALA     2      -0.926  -2.447  -0.497  1.00  1.00 \\
ATOM     15  N   ALA     2       0.756   0.780  -0.955  1.00  0.00 \\
ATOM     17  CA  ALA     2       0.634  -0.653  -1.283  1.00  1.00 \\
ATOM     19  CB  ALA     2       2.063  -1.233  -1.286  1.00  1.00\\
END }

 \vspace{10pt}
In the case of CMAP the input for \plumed \ is
 \vspace{10pt}

{\tt
PRINT W\_STRIDE 10\\
TARGETED TYPE CMAP INDEX CMAPINDEX MAP CMAPVALUES  \\
STEER CV 1 TO 3.0 VEL 0.5 KAPPA 500.0\\
ENDMETA\\
}
 \vspace{10pt}

where the CMAPVALUES file contains only one map (for details on the format of the CMAPINDEX and CMPAVALUES file see section \ref{ssc.cmap}).
}

Additionally one can specify another keyword \keyword{SQRT} that 
transform the metrics in its square root, namely the mean square deviation would be
changed into the more commonly used root mean square deviation. 
Similarly it applies to \keyword{DMSD} and  \keyword{CMAP}.
It should be stressed that the use of this keyword is \underline{particularly dangerous} especially
for values close to zero since the square root has a cusp there. 
The former example would look like: 
\esempio{
 \vspace{10pt}

{\tt
PRINT W\_STRIDE 10\\
TARGETED TYPE MSD SQRT FRAMESET ref\_frame.pdb  \\
STEER CV 1 TO 3.0 VEL 0.5 KAPPA 500.0\\
ENDMETA\\
}
 \vspace{10pt}
}

\section{Contact Map}
The Contact Map is defined as the sum of the contacts between a number 
of atom pairs specified by the user. A contact is defined in terms of the switching functions introduced for the 
coordination number CV in section \ref{coordnum}.
Each contact can have its own set of parameters, which are defined in a file specified by
the keyword \keyword{INDEX}.  See the documentation in paragraph \ref{ssc.cmap} for a detailed 
explanation of the index file syntax.

As for PCV in contact map space (see paragraph \ref{ssc.cmap}), one can define contacts between the centers of mass of groups of atoms.
The name of the file in which groups are defined is specified by using the {\tt GROUP} keyword.
See the documentation in paragraph \ref{ssc.cmap} for additional details.

The optional flag \keyword{NOPBC} can be used to calculate distances without applying periodic
boundary conditions.

\esempio{The following command defines the CV Contact Map. Distances are calculated
without periodic boundary conditions. 
 \vspace{10pt} \\
 {\tt CMAP INDEX fr.ndx SIGMA 1. NOPBC}

 \vspace{5pt}
 The {\tt  fr.ndx} file should contain the details on how each of the switching functions
 in the contact matrix should calculated: \vspace{10pt} \\
% indices of the atoms that 
% define the contact matrix:\vspace{10pt} \\
 {\tt
CONTACT    1    1    2   3.0  6 10 100.0    1 \\
CONTACT    2    1   15   3.0  6 10 100.0    1 \\
CONTACT    3   17    2   3.0  6 10 100.0    1 \\
CONTACT    4   15   19   3.0  6 10 100.0    1} \\
}


\section{Energy}
The \keyword{ENERGY} keyword instructs \plumed \ to use the total potential energy of the system
\cite{li:094101,Michel:2009p17713,Donadio:2005p17962,Bonomi:2009p17935} as a CV. 
Currently this CV is available only in GROMACS4, AMBER and DL\_POLY.  

\esempio{The following lines instruct \plumed \ to use 
 the potential energy of the system as a CV.  \vspace{10pt} \\
{\tt ENERGY SIGMA 100.0 } }

\section{Helix loops}

The \keyword{HELIX} keyword instructs \plumed \ to use the number of $\alpha$-helix loops as a CV. 
A helix loop is formed when the pair of dihedral angles $(\Phi,\Psi)$ for three consecutive 
residues along the chain all adopt a particular pair of reference values $(\bar{\Phi},\bar{\Psi})$. 
%Typical values can for $(\bar{\Phi},\bar{\Psi})\simeq(-1.200,-0.785)$.
Typically in an alpha helix $\bar{\Phi}$ and $\bar{\Psi}$ have values of -1.200 and -0.785 respectively.  
Having specified reference values for the dihedrals the total number of loops is calculated using: 
\begin{equation}
s=\sum_{i=2}^{N-1} \prod_{j=i-1}^{i+1} \frac{1}{4} \left [ cos(\Phi_j-\bar{\Phi}_i)+1 \right ] \left [ cos(\Psi_j-\bar{\Psi}_i)+1 \right ],
\end{equation}
where $N$ is the total number of residues.
This CV requires the user to specify the number of loops with the \keyword{NLOOP} keyword.
Then for each loop the user should provide three sets of four atoms that define
the dihedrals $\Phi_{i-1},\Phi_i,\Phi_{i+i}$ and a reference value for the dihedral $\bar{\Phi}_i$ 
along with three sets of four atoms that
define the dihedrals $\Psi_{i-1},\Psi_i,\Psi_{i+i}$ and a reference value the dihedral $\bar{\Psi}_i$.

\esempio{The following lines instruct \plumed \ to use the number of 
$\alpha$-helix loops as a CV.  \vspace{10pt} \\
{\tt HELIX NLOOP 2 SIGMA 0.1  \\
10 12 14 20  20 22 24 35   35 37 39 49   -1.200 12 14 20 22   22 24 35 37   37 39 49 51  -0.785\\ 
20 22 24 35  35 37 39 49   49 51 53 59   -1.200 22 24 35 37   37 39 49 51   51 53 59 61  -0.785
}}


\section{PCA projection}

The \keyword{PCA} keyword instructs \plumed \ to use as CV the projection of a set of atoms on a previously calculated Principal Components Analysis (PCA) eigenvector\cite{Amadei:1993p412}. A typical application of this CV is to explore the system along its principal directions of fluctuations \cite{Sutto:2010}.

The current conformation $X$ is first aligned to a reference structure $X^{ref}$ (except if the optional \keyword{NOALIGN} keyword is used) and then projected on the specified eigenvector $e^0$. The set of atoms used for alignment must be the same as the one the eigenvector refers to.
The user should specify the filename of the reference structure after the keyword \keyword{FRAME} and the filename of the eigenvector after the keyword \keyword{EIGENVEC}.

If the optional \keyword{DIFF} keyword is used, the difference between the current (aligned) conformation and the centered reference structure is projected on the eigenvector. 

The CV is calculated as :

$$
s_{PCA}(X,X^{ref},e^0)=\sum_{i=1}^N<\mathcal{R}^{ref} (X_i-X^{CM}),e^0_i>
$$

or, when using the \keyword{DIFF} keyword, as :

$$
s_{PCA}(X,X^{ref},e^0)=\sum_{i=1}^N<\mathcal{R}^{ref} (X_i-X^{CM})-X^{ref}_0,e^0_i>
$$

 where $N$ is the number of atoms used to align and project, $\mathcal{R}^{ref}$ is the $3\times 3$ rotation matrix, calculated using the Kearsley \cite{kearsley} algorithm, that optimally overlaps the current centered set of atoms $(X_i-X^{CM})$ into the centered reference set of atoms $X^{ref}_0$, $X^{CM}$ is the current centroid of conformation $X$ (i.e. identical masses are assumed) and $<,>$ denotes the usual inner product.

 If the optional \keyword{NOALIGN} keyword is used, no alignment is performed (i.e. the above rotation matrix $\mathcal{R}$ is the identity). This latter keyword is supported mainly for debug purposes.

\esempio{The following command defines a CV as projection, along the eigenvector listed in {\tt egv0.dat}, of the current conformation aligned to the reference structure specified in {\tt ref.dat}:

 

\vspace{10pt}
 {\tt
PCA FRAME ref.dat EIGENVEC egv0.dat SIGMA 0.1\\}

\vspace{5pt}

where {\tt ref.dat} contains the reference structure in the format ATOMID X Y Z:

{\tt 
\# my reference structure\\
2   1.324   1.045   1.550\\
5   1.316   1.174   1.469\\
6   1.345   1.174   1.350\\
7   1.287   1.286   1.538}

\vspace{5pt}

and {\tt egv0.dat} contains the eigenvector in the same format:

{\tt 
\# my first PCA eigenvector\\
2  0.12 3.28 0.19\\
5  0.53 1.37 1.10\\
6 -0.23 1.93 0.33\\
7 5.32 -2.56 1.44}\\
}

More eigenvectors can be used with additional \keyword{PCA} keywords as additional CVs, nevertheless, the set of atoms and the reference structure must be the same for all of them.
 \esempio{The following command defines two CVs as projections, along two distinct eigenvectors listed in {\tt egv0.dat} and {\tt egv1.dat}, of the difference between the current aligned conformation and the reference structure specified in {\tt ref.dat}:

\vspace{10pt}
 {\tt
PCA FRAME ref.dat EIGENVEC egv0.dat DIFF SIGMA 0.1\\
PCA FRAME ref.dat EIGENVEC egv1.dat DIFF SIGMA 0.1\\}

}

Some final notes : 
\begin{itemize}

\item{the atom and eigenvector coordinates should be in engine units (e.g. in nm for GROMACS)}

\item{the alignment is not mass-weighted (identical masses for all the involved atoms are assumed), and since the routine used for the alignment is the same as the one used to calculate the RMSD, the same cautions hold. In particular, (i)
users should ensure that the value of the hardcoded limit ({\tt MAXATOMS\_RMSD} in {\tt metadyn.h}) is sufficiently large for their needs prior to compilation, (ii) the use of double precision code is recommended as well as (iii) the use of the directive \keyword{ALIGN\_ATOMS}. See also \ref{PBC}.}

\item{starting the simulation from the identical conformation used for alignment should be avoided since it could generate numerical instabilities in the calculation of the rotation matrix, possibly leading to a crash of the simulation.}

\item{comments line beginning with '\#' are allowed but blank lines must be avoided from the input files.}

\end{itemize}


\section{SPRINT topological variables} \cv{SPRINT}

The \keyword{SPRINT} keyword instructs \plumed \ to use as CV the ``Social PeRmutation INvarianT'' coordinates described in Ref.~\cite{Pietrucci:2011}:
$$
S_i = \sqrt{N}\, \lambda^{\textrm{max}}\, v_i^{\textrm{max,sorted}}
$$
where $\lambda^{\textrm{max}}$ and $v_i^{\textrm{max}}$ are the largest eigenvalue and the corresponding eigenvector of the $N\times N$ contact matrix among $N$ atoms:
$$
\sum_{j=1}^N C_{ij}\, v_j^{\textrm{max}} = \lambda^{\textrm{max}}\, v_i^{\textrm{max}}
$$
and $C_{ij}$ is the same coordination function defined for the CV \keyword{COORD} in Section~\ref{coordnum},
with the corresponding parameters specified by the keywords \keyword{R\_0}, \keyword{D\_0} (both in units of the host MD code, e.g. Bohr for CPMD),
\keyword{NN}, and \keyword{MM}.
In $S_i$ the eigenvector components are sorted from the smallest to the largest, so that $S_1<S_2<S_3<...<S_N$.
Which $S_i$ has to be used as CV is specified by the keyword \keyword{INDEX}: e.g. \keyword{INDEX 2} means that the 2$^{\textrm{nd}}$
one has to be used, namely $S_2$.
Note that the sorting of the eigenvector is performed only within sets of atoms of the same element (automatically identified in \plumed~ by the atomic mass):
if the list of $N$ atoms includes e.g. two carbon and four hydrogen atoms, the \keyword{SPRINT} coordinates are sorted only within carbon atoms
and within hydrogen atoms, without mixing the two elements (since atoms of different elements are not indistinguishable).
Due to this, after keyword \keyword{R\_0} it must be specified the number of different element pairs and the \keyword{R\_0}-value for each possible pair of
elements. The same holds for \keyword{D\_0}. See below for an example.

\esempio{
The following command defines a SPRINT CV $S_2$ from a set of two carbon (1,2) and four hydrogen atoms (3,4,5,6):\\ 

\vspace{5pt}

{\tt
SPRINT LIST <all>  NN 6 MM 12 R\_0 3 5.0 4.2 4.2 INDEX 2 SIGMA 1.0\\
all->\\
 1 2 3 4 5 6\\
all<-
}}

In the example above \keyword{ R\_0 } {\tt 3 5.0 4.2 4.2} refers to the three pairs of elements C-C, C-H, and H-H.

Note that it is desirable to keep a tail of the coordination function long enough so that the system appears formally as a single connected cluster of atoms.
If a part is disconnected from the rest, the Perron-Frobenius theorem does not hold anymore and several $S_i$ components will go to zero (see Ref.~\cite{Pietrucci:2011}).

\section{Radial distribution function} \cv{RDF}

The keyword \keyword{RDF} instructs plumed to use the number of distances between atoms in a given range as a collective variable.  A recent paper \cite{tribello-rdfrecon} showed how a number of such collective variables could be used to define the instantaneous radial distribution function and how this description of the structure of small clusters could be used to enhance sampling in reconnaissance metadynamics simulations.

The number of distances within a given range is calculated using:

\begin{equation}
s = \sum_{i,j} \int_a^b w( r-r_{ij}) \textrm{d}r
\end{equation}

where $r_{ij}$ is the distance between atoms $i$ and $j$ and $w$ is a Gaussian window function with width $\sigma$.  When using multiple such CVs calculating the set of distances separately for each bead in the RDF would be computational expensive and counterproductive.  As such all RDF collective coordinates that involve the same atoms are calculated at the same time.  In order that multiple RDF collective coordinates can be used in a single calculation RDF beads are labeled using the RDF\_LABEL keyword.  Obviously if all your RDF CVs come from the same  RDF the RDF\_LABEL should be 1 for all your RDF coordinates.  An example input for this type of CV is as follows:

\esempio{
The following command instructs plumed to calculate two collective coordinates the number of distances between 3.0 and 3.5 and the number of distances between 3.5 and 4.0.  The value of $\sigma$ in this calculation is 0.25.\\

{\tt
RDF RDF\_LABEL 1 LIST <all> <all> RANGE 3.0 3.5 WIDTH 0.25 \\
RDF RDF\_LABEL 1 LIST <all> <all> RANGE 3.5 4.0 WIDTH 0.25 \\
all->\\
1 2 3 4 ...\\
all<-
}}

\esempio{
The following command instructs plumed to calculate two collective coordinates the number of distances between atoms in group <all> between 3.0 and 3.5 and the number of distances in group <all2> between 3.5 and 4.0.  The value of $\sigma$ in this calculation is 0.25.\\

{\tt
RDF RDF\_LABEL 1 LIST <all> <all> RANGE 3.0 3.5 WIDTH 0.25 \\
RDF RDF\_LABEL 2 LIST <all2> <all2> RANGE 3.5 4.0 WIDTH 0.25 \\
all->\\
1 2 3 4 ...\\
all<- \\
all2->\\
5 6 7 8 ...\\
all2-<
}}


\section{Angular distribution function} \cv{ADF}

Rather than using the radial distribution function as a collective variable one can use the distribution of angles between central atoms and the atoms in the first hydration sphere \cite{tribello-rdfrecon}.  Once again angular distribution functions of this type have been combined employed successfully with the reconnaissance metadynamics algorithm.  The number of angles within a given range is calculated using:

\begin{eqnarray}
w &=& \sum_{i=1}^N \sum_{j=1}^N \sum_{k=1}^N \sigma(r_{ij}) \sigma(r_{ik}) \int_a^b w(\theta - \theta_{jik} ) \textrm{d}\theta \\
\textrm{where} \qquad \sigma(r) &=& \frac{1 - \left( \frac{r-d_0}{r_0} \right)^n }{1 - \left( \frac{r-d_0}{r_0} \right)^m}
\end{eqnarray}

where $w$ is once again a Gaussian window function with width $\sigma$.  The two switching functions $\sigma(r_{ij})$ and $\sigma(r_{ik})$ are used to make sure that one only takes angles in the first hydration sphere.  A detailed description of how to set the parameters in these functions can be found in the section of this manual on coordination numbers.  Much as was described above for the RDF the RDF\_LABEL keyword ensures that the code does not waste time calculating all the angles involved in these ADF beads multiple times.  If all your ADF CVs are based on the positions of the same atoms then RDF\_LABEL should be 1 for all your ADF coordinates. In other words this keyword should be used to differentiate between the various angular distributions functions you are calculating in input.

\esempio{
The following command instructs plumed to use the number of angles in the first coordination sphere that are between 0.5 and 1.0 radians as a CV.  The value of $\sigma$ in this calculation is 0.25.\\

{\tt
ADF RDF\_LABEL 1 LIST <all> <all> <all> RANGE 0.5 1.0 WIDTH 0.25 R\_0 3.0 NN 6 MM 12 D\_0 1.5 \\
all-> \\
1 2 3 4 ... \\
all<- 
}
}

\section{Polynomial combination of CVs} \cv{POLY} 

Polynomial combinations of collective variables, given by the functional form
$$
\sum_{i=1}^{k} c_i ({\rm CV}_i - s_i)^{n_i}
$$
 can be defined using the 
\keyword{POLY} directive, followed by the \keyword{TERMS} keyword that specifies the number $k$ 
of terms to be included in the sum.
In the next $k$ lines, 
the compulsory keyword \keyword{CV} defines the number ${\rm CV}_i$ of the collective variable 
in the $i$-th term; optionally the values of $c_i$, $s_i$ and $n_i$ can be specified
preceded respectively by the keywords \keyword{COEFF}, \keyword{SHIFT} and \keyword{EXP}.
If not specified they assume the default values of $c_i=1$ $s_i=0$ $n_i=1$.
Fractional and negative values of the exponent are allowed, but 
singularities should be carefully avoided by imposing restraints on the regions
explored by the collective variables involved.  

\vspace{10pt}
\begin{minipage}{13cm}
       \scriptsize{
{{\fontfamily{phv} \fontseries{b}
 \selectfont Warning! } To combine more than 20
collective variables ($k>20$), the hardset allocation for {\tt intpar}  and
{\tt vecpar} in metadyn.h should be increased accrdingly and the code
recompiled.
} }
\end{minipage}
\vspace{10pt}

\esempio{
The following command defines a new CV as ${\rm CV}_1 - {\rm CV}_2^2$:\\
{\tt
POLY TERMS 2 SIGMA 1.0\\
CV 1\\
CV 2 COEFF -1 EXP 2\\}
}

More complicated functions might need multiple auxiliary CVs to be defined.
\esempio{
The following command defines
a new CV as $\sqrt{3{\rm CV}_1 + {\rm CV}_2}$

{\tt
POLY TERMS 2 \\
CV 1 COEFF 3 \\
CV 2 \\
NOHILLS CV 3 
 \\
POLY TERMS 1 SIGMA 1.0\\
CV 3 EXP 0.5 \\
}
}

\section{Function of CVs} \cv{FUNCTION} 

General function of CVs with arbitrary form 
 can be defined using the 
\keyword{FUNCTION} directive. 
You should define this function AFTER you defined the variables you are combining.

This function, still experimental, makes use of libmatheval library that you should download from 
\url{http://www.gnu.org/software/libmatheval/} and compile along with them.  
You may find useful infos on the syntax in \url{http://www.gnu.org/software/libmatheval/manual/}.
It is still experimental so if you want to compile it you should hack your Makefile after patching.
I tried in NAMD2.7. You should change a couple of lines in the Makefile.  
\esempio{ in the Makefile (dots represent the arguments that you already find in the Makefile) \\
\tt{
CXXBASEFLAGS = \ldots \ \$(COPTI)/my/path/to/matheval/install\_dir\_x86\_64/include\\
}
and later\\
\tt{
namd2:  -L/my/path/to/matheval/install\_dir\_x86\_64/include/lib -lmatheval  \ \ldots\\
}
}
while in SANDER (v10) I had to change 
\esempio{ in \$AMBERHOME/src/config\_amber.h  I added the needed stuff: \\
\vspace{10pt}
\tt{
AMBERBUILDFLAGS=-DAMBER -DHAVE\_MATHEVAL -L/my/path/to/matheval/install\_dir\_x86\_64/lib \\
 -lmatheval -I/my/path/to/matheval/install\_dir\_x86\_64/include\\
}
}
\vspace{10pt}
 It is crucial to put the function of CVs after all the cvs you use to define it!!!! 
\vspace{10pt}

\esempio{
The following command defines a new CV as function of the two angles \\
\vspace{10pt}
{\tt
PRINT W\_STRIDE 1 \\
ANGLE LIST 7 9 15 \\
ANGLE LIST 9 15 17 \\
FUNCTION  " 2*CV\_1 + (CV\_2+CV\_1)\*0.2 " \\}
}
Additional note is that the function is not periodic.

\section{A note on periodic boundary conditions} \index{Periodic boundary conditions} \label{PBC}

\plumed \ is designed so that for the majority of the CVs implemented the periodic 
 boundary conditions are treated in the same manner as they would be treated in the host code.
%Most collective variables are designed so as to be applied taking into account the
%periodic boundary conditions as they are defined by the host code.
However, there are some exceptions; namely:
\begin{itemize}
\item Average coordinate of a group of atoms;
\item \keyword{RGYR};
\item \keyword{DISTANCE}, \keyword{MINDIST}, \keyword{COORD}, \keyword{HBONDS}, and path collective variables \keyword{S\_PATH} and \keyword{Z\_PATH} with contact map metrics ({\tt TYPE CMAP}), if the \keyword{NOPBC} flag is used;
\item Path Collective Variables \keyword{S\_PATH} and \keyword{Z\_PATH} with RMSD ({\tt TYPE RMSD});
\item \keyword{ALPHARMSD}, \keyword{ANTIBETARMSD}, \keyword{PARABETARMSD}.
\end{itemize}
In all these cases, it is essential that the atoms involved in the definition of the CV are all part of a single,
unbreakable object such as a molecule.
Furthermore, it is essential that in the coordinates passed to \plumed \ 
the molecules are kept intact. We are aware of at least two cases where this condition is not satisfied; namely,
 when using domain decomposition or the option for periodic molecules within the host code GROMACS4.

% for the host code GROMACS4, when using domain decomposition or when using the option
% for periodic molecules.

In these cases one must use the additional directive \keyword{ALIGN\_ATOMS}, which takes the \keyword{LIST}
keyword.  By using this command the user can define an ordered group of atoms, which are to be aligned such that
the distance between adjacent atoms in the list is minimized.  In the majority of cases, the atoms
in this list should be in the same order as they appear in the pdb file as usually these files are arranged in way that
reflects how close together atoms are in the molecular topology.  As for the number of atoms that must be specified in this
list it is often sufficient to just specify those atoms involved in the CVs.  However, in cases where the atoms involved are separated
by a large distances along a chain alignment of intermediate atoms will also be required.  

%For these cases one must take advantage of the additional directive \keyword{ALIGN\_ATOMS}.
%When this directive is included, the keyword \keyword{LIST} allows to define an ordered group of
%atoms which are aligned in such a way that the distance of atom with subsequent
%indexes in the sequence is the minimal one.
%Usually these atoms should be listed in the same order as they appear in the
%pdb file, which takes into account how close they are in the molecule topology.
%Sometimes it is sufficient
%to align the atoms involved in the CVs.
%However, if the atoms involved in the CVs are far away from each other, it is necessary to also
%align intermediate atoms.

\esempio{
Input for running a metadynamics simulation using the end-to-end distance
in a protein as a CV with GROMACS4 and domain decomposition:

{\tt
PRINT W\_STRIDE 10 \\
HILLS HEIGHT 2.0 W\_STRIDE 10 \\
DISTANCE LIST 9 238 SIGMA 0.35 NOPBC \\
ALIGN\_ATOMS LIST <C-alpha> \\
C-alpha-> \\
 9 16 31 55 69 90 102 114 124 138 160 174 194 208 224 238 \\
C-alpha<- \\
ENDMETA
}
}

%% ---------------------------------------------------------------------------------------
\chapter{Postprocessing}

\section{Estimating the free energy after a metadynamics run}
\label{sc.sumhills}
The program {\tt sum\_hills.f90} is a tool for summing up the Gaussians laid during the
metadynamics trajectory and obtaining the free energy surface.

\subsection{Installation instructions}
As {\tt sum\_hills.f90} is a simple fortran 90 program, the installation is straightforward
so long as you have a fortran compiler available on your machine.
As an example, with the gnu g95 compiler one would compile {\tt sum\_hills.f90} using the following
command:
\vspace {10pt} \\
 {\tt g95 -O3 sum\_hills.f90 serial.f90 -o sum\_hills.x}

For post processing of large {\tt HILLS} files we recommend that, if you have a multicore machine available,
 you use the parallel version, which is compiled thus:
\vspace {10pt} \\
 {\tt mpif90 -O3 sum\_hills.f90 parallel.f90 -o sum\_hills\_mpi.x}


\subsection{Usage}
The sum\_hills program takes its input parameters from the command line.
If run without options, this brief summary of options is printed out.
\esempio{{\tt
  USAGE:
  sum\_hills.x -file HILLS -out fes.dat -ndim 3 -ndw 1 2 -kt 0.6 -ngrid 100 100 100 \vspace {10pt} \\
   \begin{tabular}{ l l }
   -ndim 3              & number of collective variables NCV\\
   -ndw 1 ...           &  CVs for the free-energy surface\\
  -ngrid 50 ...        & mesh dimension. DEFAULT :: 100\\
  -dp ...                   & size of the mesh of the output free energy\\
  -fix 1.1 ...             & optional definition of the FES domain\\
  -stride 10             &how often the FES is written\\
  -cutoff\_e 1.e-6  & the hills are cut off at 1.e-6\\
  -cutoff\_s 6.25    & the hills are cut off at 6.25 std dev from the center\\
  -2pi x                    & $[0;2\pi]$ periodicity on the x CV, \\
  &                            if -fix is not used 2pi is used\\
  -pi x                      & $[-\pi;\pi]$ periodicity on the x CV, \\
  & if -fix is not used 2pi is used\\
  -kt 0.6                   & $kT$ in the energy units\\
  -grad                    & apply periodicity using degrees\\
  -bias $<$biasfact$>$  & writing output the bias for a well tempered mtd\\
  -file HILLLS                   &  input file\\
  -out  fes.dat          & output file\\
  -hills nhills    & number of Gaussians that are read\\
  -aver naver      & time-average the bias profile over the last naver Gaussians
  \end{tabular}
}}
The program works in the following way.

Using the {\tt -file} and {\tt -out} flags one tells the program the names
of the input file containing the Gaussian hills file and the name of the 
file in which the free-energy will be outputted.
  In the absence of any instruction sum\_hills assumes these files are
  to be called {\tt HILLS} and {\tt fes.dat}.

The number of CVs in the {\tt HILLS} file is specified using 
{\tt -ndim} whilst the number of CVs the output free energy surface is to be plotted as a function of is controlled by
{\tt -ndw} followed by the list of CVs in the desired order. Note that after
being read the CV are reordered in the code according to ndw. As a direct result EVERY output uses this new order. 
If the number of CVs requested using {\tt -ndw} is less than the number of CVs in 
in {\tt HILLS} file the CVs not specified for output will be
integrated out with the Boltzmann weight $K_bT$ specified by {\tt -kt} 
(kT must be given in the energy units that were used by in the code in which the simulation was performed) 
%(the energy units of the simulations are assumed).

The position, width and height of the Gaussians are read from the file
specified by the {\tt -file} option ({\tt HILLS} is the default) and the
free-energy surface is printed out on a grid in gnuplot format with a
blank line added after each block of data. Gnuplot is very
handy for quick visualization of 3D data (e.g. the FES as a function of 2 CVs).

For efficiency, the Gaussians are truncated at a certain distance from 
their center before being placed on the grid.
{\tt -cutoff\_s} and {\tt -cutoff\_e} allow one to tune this cutoff distance.
With {\tt -cutoff\_s} the value read specifies this truncation distance in terms of the number of
standard deviations from the center.  For consistency this should be set equal to the DP2CUTOFF
used in the metadynamics simulation (the default value for this parameter is 6.25 both in \plumed \ and in sum\_hills). 
  Alternatively, the user may specify this cutoff distance as the distance at which the energy of the Gaussian hill falls to less 
than some critical value specified using the {\tt -cutoff\_e} flag.  The energy specified after this flag must be in the
units used during the metadynamics simulation.  N.B. if the cutoff used in post-processesing 
is different to that used during the simulation the calculated free energy differs from the bias which was actually applied during
the metadynamics simulation.

% is the number of standard deviations from the center at
% which the Gaussian is truncated and it should be equal to DP2CUTOFF as used in the metadynamics run (the
% default value is 6.25 in both cases). Alternatively
% the user can specify the energy at which the Gaussian is truncated by using {\tt -cutoff\_e}; here the
% energy units are the same as in the simulation. Note that
% if the cutoff is different in the run and in this post-processing
% the calculated free energy differs from the bias which was actually applied during
% the metadynamics simulation.

The size of the output grid can be controlled either with {\tt -ngrid}
followed by the number of grid points in each CV direction
% (as per input) 
or by {\tt -dp} followed by the size of the voxel in each CV
direction. If neither of these options are specified, the grid
size is assumed to be 100 in each direction and the voxel size is
calculated such that all the input Gaussians fit into the grid. The {\tt -fix}
option allow one to fix the boundaries of the output free energy and after this flag. 
 two real numbers are expected for each CV.

{\tt -stride} allows one to print out the evolution of the free-energy as a
function of time, \emph{i.e.} as a function of the index of the Gaussian. 
{\tt -stride} expects an integer which specifies how many Gaussians
are added to the calculated free energy surface between print outs. The progressive free energy
is printed in files named {\tt fes.dat.XXX} where {\tt XXX} is an increasing
counter. The final free-energy is printed in {\tt fes.dat}.
This is useful for creating movies of the how the free-energy wells fill as a function of time.

{\tt -hills} is used to integrate only a part of the Gaussian files. It
expects an integer that specifies the maximum number of Gaussians to be read from input.

{\tt -naver} is used to plot the time average of the bias profile over the last given number of hills
(see Eq.~33 in Ref.~\cite{gerv-laio09review})

When periodic CVs like angles and dihedrals are used as CVs, periodicity options must be specified.
The flags {\tt -pi} and {\tt -2pi}, which are followed by the CV index, specify that
that the CV is periodic between [-pi;pi] or [0;2pi] respectively. The {\tt -grad} flag specifies that the CV values
are being given degrees rather than radians.

\section{Evaluating collective variables on MD trajectories}

The program {\tt driver} can be used to evaluate the value of any of the CVs
 implemented in \plumed \ for all the structures in a given trajectory file.

For GROMACS users: note that {\tt driver} has some limitation in the choice
of the simulation box shape. With GROMACS, it could be convenient to use
the {\tt -rerun} option of {\tt mdrun} together with \plumed.

\subsection{Installation instructions}
Before compiling {\tt driver}, the files
contained in the {\tt common\_files} directory must be linked to the 
{\tt utilities/driver} directory (i.e. this can be done using {\tt ./getlinks.sh}).

To compile using g95/gcc, type:
\vspace {5pt} \\
{\tt make arch=g95}
\vspace {5pt}\\
To compile with gfortran/gcc, type:
\vspace {5pt}\\
{\tt make arch=gfortran}
\vspace {5pt}\\
To compile with gfortran/gcc on a 64 bit machine, type:
\vspace {5pt}\\
{\tt make arch=gfortan\_64}
\vspace {5pt}\\
To compile with ifort/icc Intel compilers, type:
\vspace {5pt}\\
{\tt make arch=intel}
\vspace {5pt}\\

To use other compilers and/or compiler flags you will need to modify the Makefile.

\subsection{Usage}

This program takes its input parameters from the command line. If run 
without options, this brief summary of options is printed out. 

\esempio{Invoking {\tt driver} without arguments prints a list of the available options: \vspace{10pt} \\
{\tt USAGE : \\
 driver -pdb PDB\_FILE -dcd DCD\_FILE -plumed PLUMED\_FILE -ncv  \\
          (-interval min1 max1 min2 max2 -out OUT\_FILE -nopbc -cell CELLX CELLY CELLZ)\\
 \\
 \begin{tabular}{ l l }
 -pdb      &  pdb (one frame) connected to dcd file\\
 -dcd      &  trajectory file\\
 -plumed    & PLUMED-like input file\\
 -ncv        & number of collective variables\\
 -out        & pdb output filename for clustering format (optional)\\
 -interval   & extract frames with CV in this interval (optional)\\
  -nopbc      & don't apply pbc                                   (optional) \\
 -cell       & provide fixed box dimension in Angstrom \\
& for orthorhombic PBC  (optional)
\end{tabular}
 }
}

The user must provide a structure file in PDB format (only one frame, not a movie), within which the 
 atoms' masses and charges are inserted in the 
occupancy field and in the B-factor field respectively.
This data must be provided as it is required for the calculation of certain collective variables, such as
the center of masses or dipoles.

The trajectory file must be a CHARMM format DCD file (also NAMD 2.1 and later). 
To convert other trajectory files into this format, the user can download the utility {\tt CatDCD} from: \\ \url{http://www.ks.uiuc.edu/Development/MDTools/catdcd/}.\\

Within the input file driver the user can specify a printing stride (in number of DCD frames) using
the keyword {\tt W\_STRIDE} and the list CVs he/she wishes to calculate.  This file has the same syntax as the \plumed \ input file.
The CVs value will be printed on the {\tt COLVAR} file.
For a complete list of the CVs driver can calculate, please see section \ref{ch.generalcv}.

\esempio{If you wish to evaluate the distance between atom 13 and 17
for every frame in your trajectory file, the input file for driver would be as follows:
 \vspace{10pt} \\ 
{\tt
PRINT W\_STRIDE 1\\
DISTANCE LIST 13 17\\
ENDMETA
}}

By default, {\tt driver} uses orthorhombic periodic boundary conditions and looks for cell information in the DCD file. 
If these are not present, you must either provide the fixed dimensions of the box in Angstrom (only orthorhombic PBCs are allowed)
using the keyword {\tt -cell CELLX CELLY CELLZ} or switch off the pbc using the {\tt -nopbc} flag.
\\
Driver can also be used to extract frames in a specific window of the CVs space and write these extracted frames to a PDB file.
To use this functionality use the option {\tt -out output.pdb} and specify the interval with {\tt -interval CVmin CVmax}.

\esempio{To extract those frames in which the distance between 
two specified atoms is between 10.0 and 12.0 \AA  \ while the angle defined by three specified atoms is between 2.0 and 2.3 rad, 
the user should prepare the following input file (named plumed.dat):  \vspace{10pt} \\ 
{\tt
PRINT W\_STRIDE 1 \\
DISTANCE LIST 13 17\\
ANGLE LIST 19 20 22\\
ENDMETA \vspace{10pt}\\
}
and then invoke {\tt driver} using the following command:  \vspace{10pt} \\
{\tt  ./driver -pdb dialanine.pdb -dcd dialanine.dcd \/
  -plumed plumed.dat -ncv 2 -out window.pdb -interval 10.0 12.0 2.0 3.0 \\ }

The output file window.pdb is written in a format appropriate for use with the GROMACS tool {\tt g\_cluster}, which can be used to perform
a cluster analysis on your set of frames.

}

\section{Processing {\tt COLVAR} files}

\index{plumedat.sh}

Since \plumed\ 1.2, a more flexible format has been adopted for {\tt COLVAR} files.
These files can be parsed with the simple {\tt plumedat.sh} tool.
If run it without options, this brief summary of options is printed out.

\esempio{Invoking {\tt plumedat.sh} without arguments prints a list of the available options: \vspace{10pt} \\
{\tt
syntax:

plumedat.sh f1 f2 ... < file

or

plumedat.sh -l < file

file is the name of the COLVAR file

f1, f2, ... are the names of the required fields

if a required field is not available in the COLVAR file, "NA" is written in the output
with -l, the available choices are listed

example:

plumedat.sh time temp < COLVAR

prints a two-column file, with the time in the first column and the temperature
in the second column
}
}

\section{\plumed \ as a standalone program}

\index{Standalone}

\plumed\ can be run as a standalone program so that it can be easily integrated in a script. This is particularly
useful whenever the time between one \plumed \ call and the other is large (e.g. in ab initio programs) and one
aims at minimizing the time needed for implementation. {\tt plumed\_standalone} consists in a simple program, very much like
the  {\tt driver} tool, that reads the configuration and prints out the forces to be added and the other various output files 
typical of \plumed.  

\subsection{Installation instructions}
Before compiling {\tt plumed\_standalone}, the files
contained in the {\tt common\_files} directory must be linked to the
{\tt utilities/standalone} directory (i.e. this can be done using {\tt ./getlinks.sh}).
\vspace {5pt} \\
To compile using g95/gcc, type:
\vspace {5pt} \\
{\tt make arch=g95}
\vspace {5pt}\\
To compile with gfortran/gcc, type:
\vspace {5pt}\\
{\tt make arch=gfortran}
\vspace {5pt}\\
To compile with gfortran/gcc on a 64 bit machine, type:
\vspace {5pt}\\
{\tt make arch=gfortan\_64}
\vspace {5pt}\\
To compile with ifort/icc Intel compilers, type:
\vspace {5pt}\\
{\tt make arch=intel}
\vspace {5pt}\\

To use other compilers and/or compiler flags you will need to modify the Makefile.
In this way you may obtain an executable which is {\tt plumed\_standalone}.

\subsection{Usage}

\plumed\ standalone expects a standard \plumed\ input, a configuration file for your system (much in the format of 
xyz) and gives a screen output.
The configuration file has the following format:
 
\esempio{Example of {\tt plumed\_standalone}  input configuration  \vspace{10pt} \\
{\tt 
TIME 0.100000 100 \\
AMPLI 1.000000 \\
BOLTZ 0.001987 \\
BOX 1000.000000 1000.000000 1000.000000 \\
 \begin{tabular}{ r r r r r  }
     -13.523670417   &     0.140088464   &     4.263822219  &  131.207999  &   0.000000 \\
      -9.821878867   &    -0.695171589   &     5.042585958  &  101.119998  &   0.000000 \\
      -7.193416272   &    -2.392065561   &     2.830069705  &  163.190996  &   0.000000 \\
      -3.844793108   &    -3.737129154   &     4.056782255  &  129.197997  &   0.000000 \\
       \vdots
 \end{tabular}
} 
}

The configuration file must have the keyword {\tt TIME} followed by the step length (in the units  
of the program) and {\tt AMPLI} which is the conversion factor between \AA\ and the actual length unit in
the configuration file. This is needed when a RMSD calculation with respect to a pdb file in \AA \ must be performed. 
In this case, if the configuration is in Bohr, the {\tt AMPLI} factor must be set equal to 1.889725989.
{\tt BOLTZ} is the Boltzmann factor in the chosen units and {\tt BOX} followed by three numbers specifies the dimensions 
of the orthorombic simulation box.  

The following lines specify the configuration. The first three columns are 
the x, y and z coordinates, the fourth is the mass in units of the program (in the case above a coarse grained program takes the mass of 
all the single residues) and in the last column you may put the charge (in the code above it was not needed). 
\plumed \ invocation may be done with command line:

\esempio{Example of invocation of \plumed\ {\tt standalone}
\vspace{10pt} \\
{\tt
./plumed\_standalone -coord x.xyz -plumed plumed\_input.cfg 
}
}

{\tt plumed\_standalone} returns back a file named {\tt plumed\_forces.dat} which 
contains the additional energy and forces from the bias potential calculated by the 
\plumed \ module. Eventually, these must be summed to the ones of the original program. 

\esempio{Example of {\tt plumed\_standalone} output file. In the first line is the additional bias potential
calculated by \plumed, followed by the x, y and z components of the force for each atom.\\

\vspace{10pt}
{\tt
 2.35680 \\
 3.335998704981455 -3.684076849212145 -16.098665997131253 \\
 -0.9064447402311846 2.913567594563722 -1.0502785066801152 \\
 -0.7948355287665761 0.9861425018822194 0.6558434539267759 \\
 -1.9437110159112139 -0.6087819098852943 2.4119663558915296 \\
\vdots
}
}

%\chapter{Implementation details}
%\label{ch.implementation}
\section{Reweighting well-tempered metadynamics calculations}
\index{Well-tempered metadynamics Reweight}

From a converged metadynamics run we can calculate directly the canonical probability distribution of the
collective variables at a given temperature.
On the contrary, the statistics of other degrees of freedom is somehow distorted by the application, 
during the simulation, of a time-dependent external potential on the CVs.
Different possible techniques have been proposed to reconstruct the probability distribution of variables other 
than the CVs \cite{Tiana:2008p9224,marinelli-trpc09,Bonomi:2009p17105}.

In well-tempered metadynamics, the reconstruction of the distribution of variables different from the CVs
is particularly simple since for long times the amount of bias added decreases to zero and the
system becomes closer and closer to equilibrium.
The algorithm described in Ref. \cite{Bonomi:2009p17105} consists of three different steps:
\begin{enumerate}
\item Accumulate the histogram of the CVs plus the other variables of interest between
two updates of the bias potential;
\item When a new Gaussian is added, evolve the histogram following:
\begin{equation}
P(\bm{R},t+\Delta t)= e^{-\beta(\dot{V}(\bm{S}(\bm{R}),t)-\langle \dot{V}(\bm{S},t) \rangle )\Delta t} \, P(\bm{R},t),
\label{evol2}
\end{equation}
where $P(\bm{R},t)$ is the  biased probability distribution, $\dot{V}(\bm{S}(\bm{R}),t)$  
the time derivative of the bias potential and the average in the exponent is calculated in the biased ensemble;
\item At the end of the simulation, the unbiased distribution $P_B(\bm{R})$ can be recovered from the histogram collected 
by using a standard umbrella sampling reweighting:
\begin{equation}
P_B(\bm{R}) \propto e^{\beta V(\bm{S}(\bm{R}),t)} \cdot P(\bm{R},t). 
\end{equation}
\end{enumerate}

\subsection{Installation instructions}
As {\tt reweight.f90} is a simple fortran 90 program, the installation is straightforward
so long as you have a fortran compiler available on your machine.
As an example, with the gnu g95 compiler one would compile {\tt reweight.f90} using the following
command:
\vspace {10pt} \\
 {\tt g95 -O3 reweight.f90 -o reweight}

\subsection{Usage}

\esempio{We performed a metadynamics run with 2 CVs and we are interested
in reconstructing the distribution of a third variable. \vspace{0.25cm} \\

{\tt
 reweight -colvar COLVAR -hills HILLS -ncv 2 -nvar 3 -stride 1 -fes 3 -temp 300 -welltemp 
 \vspace{0.25cm} \\
   \begin{tabular}{ l l }
 -hills      & HILLS  filename \\                        
 -colvar     & COLVAR filename \\                        
 -out        & FES    filename  \\                       
 -ncv        & number of variables in HILLS \\                
 -nvar       & number of variables in COLVAR  \\              
 -stride     & ratio between COLVAR and HILLS stride\\
 -fes        & ID of the variables for FES in output \\       
 -temp       & temperature in Kelvin \\                  
 -ngrid      & histogram grid dimension \\               
 -nreject    & discard initial steps \\                
 -timeout    & stride for FES printout \\               
 -pi         & ID of the variables with $[-\pi;\pi]$ periodicity \\
 -welltemp   & control for well-tempered metadynamics \\
 -kjoule     & energy in kjoule/mol \\ 
  \end{tabular} 
}}

The code needs two files with the same format of the \plumed\ {\tt HILLS} and {\tt COLVAR} files.
In the latter, the metadynamics CVs should appear in the first $d$ column followed by the
variables whose distribution one wants to reweight. The ratio between the stride in {\tt COLVAR}
and {\tt HILLS} must be constant and greater than 1. The more data you have for the histogram, the better.

Some important things to keep in mind:
\begin{itemize}
\item For the choice of the bin size, please follow the suggestions described in section \ref{grid};
\item Eq. \ref{evol2} is exact. However, at the beginning of the simulation the
average of $\dot{V}(\bm{S}(\bm{R}),t)$ can be calculated only approximately.
Luckily, a possible initial error is recovered for long times. 
Alternatively, one could discard the first part of the trajectory using the {\tt -nreject} option.
Please always check that your results are robust to a discard of initial parts of the trajectory;
\item As for the calculation of the FES with {\tt sum\_hills}, remember to control the convergence
by plotting the reconstructed distribution at different times by using {\tt -timeout}; 
\end{itemize}

\section{Bias-exchange simulations via the linux shell}
\index{Bias-exchange simulations via the linux shell}
\label{sec.betool}

Bias-exchange simulations~\cite{piana} can be performed using every MD engine, employing the tool in the
directory {\tt utilities/bias-exchange}. Each replica (walker) runs independently in a different directory
for a duration $\tau_{\mathrm exch}$ equal to the time between two exchanges of the bias. When all runs are finished, 
a script called {\tt bias-exchange.sh} takes care of attempting exchanges of the bias and re-launching
the simulation for each replica. 

This procedure via the linux shell can be inefficient in terms of computer time (due to the need to stop and restart the simulations
to perform exchanges)
if $\tau_{\mathrm exch}$ is chosen very short, e.g. $<100$ timesteps. However in the literature times larger than 10~ps are typically employed
for classical simulations of biomolecules, while for ab-initio simulations this should not be an issue already for $\tau_{\mathrm exch}\sim 0.1$~ps.
Note that for GROMACS \plumed\  provides also bias-exchange within a single parallel run, which is computationally more efficient
(see Section~\ref{sec.beparallel}).

A simulation proceeds as follows (see also the directory {\tt example}).
First, the Fortran code {\tt exchange-tool.f90}
is compiled using the command {\tt make} (modify the Makefile if necessary).
The script {\tt bias-exchange.sh} and the
program {\tt exchange-tool.x} must be copied in a chosen directory {\tt BASE\_DIR},
and the following variables inside the script {\tt bias-exchange.sh} must be modified:
\begin{itemize}
\item {\tt BASE\_DIR}
\item {\tt NWALKER} is the number of replicas
\item {\tt KBT} is the Boltzmann factor $k_BT$ in units of the MD engine
\item {\tt NSIMULATIONS} is the maximum number of runs between exchanges for each walker
\end{itemize}
Then, inside {\tt BASE\_DIR} the user must create {\tt NWALKER} directories called {\tt walker1}, {\tt walker2}, etc.,
and in each one the following files must be present:
\begin{itemize}
\item an input for restarting an MD simulation of length $\tau_{\mathrm exch}$
\item a \plumed\ input file called {\tt plumed.dat} (see example below)
\item a script {\tt run-walker.sh} (see example below)
\end{itemize}
Finally, the bias-exchange simulation is started by launching the MD runs in each {\tt walker1}, {\tt walker2}, etc. directories using the command {\tt ./run-walker.sh \&} (the latter script must be executable: it can be made so e.g. using command {\tt chmod a+x run-walker.sh}).
This script must launch the MD run in foreground (no {\tt \&} character at the end of the line) and it must end 
with the following lines: \\
\\
{\tt \ touch READY} \\
{\tt \ ../bias-exchange.sh \&} \\
\\
After a time $\tau_{\mathrm exch}$ the runs finish and control passes to the {\tt bias-exchange.sh} script (in background), which attempts
to exchange the bias (i.e. files {\tt HILLS} and {\tt plumed.dat}, not atomic coordinates) between pairs of randomly chosen replicas.
Detailed output is printed in the file {\tt bias-exchange.log}.

It is required to employ the keyword \keyword{HILLS\_LABEL} in each {\tt plumed.dat} input file, with a different label for each walker: 
in this way \plumed\ includes the information (necessary for the bias-exchange tool)
about the active CVs in the {\tt COLVAR} and {\tt HILLS} files.

\esempio{Example of {\tt plumed.dat} input file in directory {\tt walker1}. Note the keyword {\tt HILLS\_LABEL}
and that only CV 1 is biased.\\ \\
{\tt
HILLS RESTART HEIGHT 0.001 W\_STRIDE 100 \\
PRINT W\_STRIDE 10 \\
HILLS\_LABEL A \\
\\
COORD LIST 1 <g234> NN 6 MM 12 R\_0 5.0 SIGMA 0.1 \\
g234-> \\
 2 3 4 \\
g234<- \\
 \\
COORD LIST 2 <g134> NN 6 MM 12 R\_0 5.0 SIGMA 0.1 \\
g134-> \\
 1 3 4 \\
g134<- \\
 \\
COORD LIST 3 <g124> NN 6 MM 12 R\_0 5.0 SIGMA 0.1 \\
g124-> \\
 1 2 4 \\
g124<- \\
 \\
COORD LIST 4 <g123> NN 6 MM 12 R\_0 5.0 SIGMA 0.1 \\
g123-> \\
 1 2 3 \\
g123<- \\
 \\
NOHILLS CV 2 \\
NOHILLS CV 3 \\
NOHILLS CV 4 \\
 \\
ENDMETA \\
}
}

The other walkers will have a similar {\tt plumed.dat} input file, but with different label 
(e.g. {\tt HILLS\_LABEL B} for {\tt walker2}, etc.) 
and different active CV (e.g. the second one in {\tt walker2}, etc.).

\esempio{Example of {\tt run-walker.sh } script file: the script must be executable.\\ \\
{\tt
\#!/bin/bash \\
\# note that CPMD is launched in foreground: \\
cpmd.x input . -plumed >> output \\
\# this creates and empty file called READY, \\
\# telling the bias-exchange tool that the MD run has finished:\\
touch READY \\
\# note that bias-exchange.sh is launched in background:\\
../bias-exchange.sh \&
}
}

At the end of the bias-exchange simulation, the multidimensional free-energy landscape as a function of several CVs
can be reconstructed according to the weighted-histogram algorithm in Ref.~\cite{marinelli-trpc09},
e.g. employing the METAGUI program~\cite{METAGUI} downloadable from \\ \url{http://www.plumed-code.org/contributions}

The bias-exchange simulation can be continued, after {\tt NSIMULATIONS} steps are reached and all replicas have finished their run, in the following way:
\begin{itemize}
\item enlarge {\tt NSIMULATIONS} in the script {\tt bias-exchange.sh}
\item in each walker-directory, delete the file {\tt READY}
\item in each walker-directory, restart the run using {\tt ./run-walker.sh \&}
\end{itemize}

\clearpage
\bibliographystyle{./elsarticle-num}
%\phantomsection
%\addcontentsline{toc}{chapter}{Bibliography}
\bibliography{./bibliography}
\printindex

\end{document}
